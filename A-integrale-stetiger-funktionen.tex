\appendix
\newpage
\section{Integrale stetiger Funktionen \small (einer reellen Variable)}
\marginpar{\tiny{23.10.2019}}
Wir unterscheiden zwischen
\begin{enumerate}[- ,topsep =-3pt]
	\item dem \emph{bestimmten Integral}
				$$ \int\limits_a^b f(x)\D x$$
	\item und dem \emph{unbestimmten Integral}, d.h. die Menge der Funktionen dieser Art
	$$x \mapsto \int\limits_a^x f(\xi) \D \xi + \underset{\text{\tiny const}}C$$
	Notation: $\int f \D x$.
\end{enumerate}
\paragraph{Hauptsatz der Differential- und Integralrechnung}
\begin{enumerate}[(\roman*), topsep = -1 pt]
	\item Ist $f$ von der Klasse $\mathcal{C}^0$ (d.h. stetig), so ist 
		$$\left(\int f \D x\right)' = f$$
		d.h. die Repräsentanten des unbestimmten Integrals sind Stammfunktionen.
	\item Ist $f$ von der Klasse $\mathcal{C}^1$ (d.h. stetig differenzierbar), so ist
	$$ \int F' \D x =F$$
	(zu lesen: $F$ repräsentiert $\int F' \D x)$ bzw.
	$$ \int\limits_a^x F'(\xi) \D x = F(x) -F(a)$$
\end{enumerate}
\paragraph{Rechenregeln für Differentialrechnung $\leadsto$ Rechenregeln für Integralrechnung}z.B.
\begin{equation*}
	\begin{split}
		(\ln |x|)' = \frac{1}{x} & \implies \int \frac{\D x}{x} = \ln |x| \text{ auf } \R\setminus\{0\}\\
		\arcsin' x {=\frac{1}{\sqrt{1-x^2}}} & \implies \int \frac{\D x}{\sqrt{1-x^2}} = \arcsin x \text{ auf } (-1,1) \\ 		
		\arctan' x = \frac{1}{1+x^2} & \implies \int \frac{\D x}{1+x^2} = \arctan x \text{ auf }\R
	\end{split}
\end{equation*}
\paragraph{Kettenregel $\leadsto$ Substitutionsregel} Aus der Kettenregel
$$(F \circ \varphi)' (u) = F'(\varphi (u)) \cdot \varphi'(u)$$
folgt mit dem Hauptsatz der Differential- und Integralrechnung: Sei $f := F'$ ($\mathcal{C}^0$),
$$ \int\limits_a^b f(\varphi(u)) \cdot \varphi'(u) \D u = \int\limits_a^b (F \circ \varphi) (u) \D u= F \circ \varphi \big\vert_a^b = F\big\vert^{\varphi(b)}_{\varphi(a)} = \int\limits_{\varphi(a)}^{\varphi(b)} f \D x
$$
Also die \textbf{Substitutionsregel} (Bezeichne $I:= (a,b), J = (\phi(a),\phi(b))$)
$$
\boxed{
\int\limits_a^b f(\varphi(u)) \varphi'(u) \D u = \int_{\phi(a)}^{\phi(b)} f(x) \D x 
}
$$
und die \textbf{Version für unbestimmtes Integral}
$$ \int f(\phi(u)) \phi'(u) \D u = \underbrace{\int f(x) \D x \Big \vert_{x =\phi(u)}}_{\substack{\text{die Komposition }\phi \\ \text{ mit } \int f(x)\D x}} $$
\begin{example} \
\begin{enumerate}
 \item \textbf{Lineare Substitution} mit $x = u+\alpha, \alpha \in \R$ 
 $$
	\int_a^b f(u+\alpha) \D x = \int_{a + \alpha}^{b+ \alpha} f(x) \D x
 $$
 bzw. 
 $$
 	\int f( u +\alpha) \D u = \int f(x) \D x \Big\vert_{x = u + \alpha}
 $$
 z.B. $f(x) = \frac{1}{x}$ auf $\R \setminus \{0\}$
 $$ \int_a^b \frac{\D u}{u+\alpha} = \int_{a+\alpha}^{b+\alpha} \frac{\D x}{x} = \ln |x| \big\vert^{b+\alpha}_{a+\alpha} = \ln \left| \frac{b+\alpha}{a+\alpha} \right|$$
 bzw.
 $$ \int \frac{\D u}{u+\alpha} = \int \frac{\D x}{x} \Big\vert_{x= u+\alpha}= \ln |x| \big\vert_{x = u + \alpha} = \ln |u + \alpha|$$
 \item[(i')] \textbf{(Multiplikative) lineare Substitution} mit $x= \lambda u (\lambda \in \R \setminus \{0\})$
 	$$ \int_a^b f(\lambda u) \D u = \frac{1}{\lambda} \int_{\lambda a}^{\lambda b} f(x) \D x$$ bzw. 
 	$$ \int f(\lambda u) \D u = \frac{1}{\lambda} \int f(x) \D x \Big\vert_{x = \lambda u}$$
 	z.B.
 $$\int \cos \lambda u \D u = \frac{1}{\lambda} \int \underbrace{\cos x}_{\sin' x} \D x \Big\vert_{x= \lambda u} = \frac{1}{\lambda} \sin \lambda u$$
 \item \textbf{Quadratische Substitution} mit $x = u^2$
 $$ \int_a^b f(u^2) u \D u =\frac{1}{2} \int_{a^2}^{b^2} f(x) \D x$$
 bzw.
 $$ f(u^2) u \D u = \frac{1}{2} f(x) \D x \Big\vert_{x=u^2}$$
 z.B. $f(x) = e^x$:
 $$ \int u  e^{u^2} \D u = \frac{1}{2} \int e^x \D x \Big\vert_{x=u^2} = \frac{1}{2} e^{u^2}$$
 \item Mit $f(x) = \frac{1}{x}$, (falls $\phi|_J$ keine Nullstelle hat)
 $$ \int_a^b \frac{\phi'(u)}{\phi(u)} \D u = \int_{\phi(a)}^{\phi(b)}\frac{\D x}{x} = \ln |x| \big\vert^{\phi(b)}_{\phi(a)} = \ln |\phi(u)| \big\vert_{a}^b$$
 bzw.
 $$ \int \frac{\phi'(u)}{ \phi(u)} \D u = \int \frac{\D x}{x} \Big\vert_{x=\phi(u)} = \ln | \phi(u) |$$
 z.B. $\phi(u)= \cos u$ auf $(-\frac{\pi}{2},\frac{\pi}{2})$
 $$\int \tan u \D u  = \int -\frac{\cos' u}{\cos u} \D u= -\ln |\cos u|$$
 Berechne $\int \frac{\D x}{\sqrt{1+x^2}}$ auf $\R (=I =J)$. Substituiere $x=\sinh u$ mit der Umkehrfunktion $u = \operatorname{arsinh} x$.
 $$\int \frac{\D x}{\sqrt{1+x^2}} = \int \frac{\sinh' u}{\sqrt{1+\sinh^2 u}} \D u \Bigg\vert_{u = \operatorname{arsinh}x}= \int \frac{\cosh' u}{\cosh' u} \D u\Big\vert_{u = \operatorname{arsinh}x} = \operatorname{arsinh}x$$
\end{enumerate}
\end{example}