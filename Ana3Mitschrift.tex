\documentclass[12pt,a4paper]{article}
\usepackage[left=28mm,top=28mm,right=28mm,bottom=28mm] {geometry}
\usepackage{amsfonts}
\usepackage{mathrsfs}
\usepackage{mathtools}
\usepackage{relsize}
\usepackage{etoolbox}
\usepackage[ngerman]{babel}
\usepackage[utf8]{inputenc}
\usepackage[T1]{fontenc}
\usepackage{marvosym}
\usepackage[shortlabels]{enumitem}
\usepackage{mathtools}
\usepackage{amssymb}
\usepackage{cancel}
\usepackage{mdframed}
\usepackage{framed}
\usepackage{mathtools}
\usepackage{tablefootnote} 
\usepackage{listings}
\usepackage{amsthm}
\usepackage{xcolor}
\usepackage{etoolbox}
\usepackage[all]{xy}
\usepackage{tikz}
\usetikzlibrary{cd}
\usetikzlibrary{calc}
\theoremstyle{definition}
\newtheorem*{example}{Beispiel}
\newtheorem*{korollar}{Korollar}
\newtheorem*{satz}{Satz}
\newtheorem*{proposition}{Proposition}
\newtheorem*{theorem}{Theorem}
\newtheorem*{lemma}{Lemma}
\newtheorem*{definition}{Definition}
\newtheorem{aufgabe}{Aufgabe}
\theoremstyle{remark}
\newtheorem*{remark}{Bemerkung}
\newtheorem*{remark'}{Nebenbemerkung}
\AfterEndEnvironment{lemma}{\noindent\ignorespaces}
\AfterEndEnvironment{definition}{\noindent\ignorespaces}
\AfterEndEnvironment{example}{\noindent\ignorespaces}
\AfterEndEnvironment{theorem}{\noindent\ignorespaces}
\AfterEndEnvironment{satz}{\noindent\ignorespaces}
\AfterEndEnvironment{korollar}{\noindent\ignorespaces}
\AfterEndEnvironment{remark}{\noindent\ignorespaces}
\AfterEndEnvironment{remark'}{\noindent\ignorespaces}
\AfterEndEnvironment{proposition}{\noindent\ignorespaces}
\AfterEndEnvironment{proof}{\noindent\ignorespaces}
\usepackage{thmtools}
 \usepackage[
   pdfpagelabels=true,
   pdftitle={Analysis III: Maßtheorie und Integralrechnung mehrerer Variablen},
   %pdfauthor={},
 ]{hyperref}
\usepackage{bookmark}
\let\existstemp\exists
\let\foralltemp\forall
\newcommand{\tikzmark}[1]{\tikz[overlay,remember picture] \node (#1) {};}
\newcommand{\vsubset}{\rotatebox[origin=c]{90}{$\subset$}}
\newcommand{\ol}{\overline}
\newcommand{\R}{\mathbb{R}}
\newcommand{\C}{\mathbb{C}}
\newcommand{\N}{\mathbb{N}}
\newcommand{\T}{\mathcal{T}}
\newcommand{\F}{\mathcal{F}}
\newcommand{\D}{\, \mathrm{d}}
\newcommand{\E}{\mathbb{E}}
\newcommand{\Hom}{\operatorname{Hom}}
\newcommand{\del}{\partial}
\newcommand{\vol}{\operatorname{vol}}
\newcommand{\Var}{\operatorname{Var}} 
\newcommand{\Cov}{\operatorname{Cov}}
\newcommand{\End}{\operatorname{End}}
\newcommand{\SL}{\operatorname{SL}}
\newcommand{\Bild}{\begin{tiny}(Bild hier)\end{tiny}}
\newcommand{\vldate}{\flushright\scriptsize}
\renewcommand*{\exists}{\existstemp\mkern2mu}
\renewcommand*{\forall}{\foralltemp\mkern2mu}
\renewcommand{\emptyset}{\varnothing}
\renewcommand{\qedsymbol}{$\blacksquare$}
\makeatletter 
\AfterEndEnvironment{mdframed}{%
 \tfn@tablefootnoteprintout% 
 \gdef\tfn@fnt{0}% 
}
%\setcounter{tocdepth}{3}

\title{Analysis III: Maßtheorie und Integralrechnung mehrerer Variablen}
%\date{}
\begin{document}
\maketitle
\tableofcontents
\addtocontents{toc}{\protect\setcounter{tocdepth}{3}}
\newpage
\section{Maßtheorie}
\subsection{Maßproblem und Paradoxien}
\marginpar{\tiny{14.10.2019}}
Maßtheorie ist die Theorie des Volumens. Motivierende Beispiele sind:
\begin{enumerate}[label=\roman*),topsep=3pt, itemsep=0pt]
\item Volumina von Teilmengen des euklidischen Raums
\item Wahrscheinlichkeiten (= ``Volumina von Ereignissen'')
\end{enumerate}
Wir konzentrieren uns im Rest des Abschnitts auf $\R^d$. Wir wollen leistungsfähigen Volumenbegriff haben, sodass die Volumina von möglich vielen Teilmengen flexibel gemessen werden können. Unser erster ``naiver'' Ansatz wäre, dass wir Volumenmessung für \emph{alle} Teilmengen verlangen, also eine Funktion
\begin{equation}
 \vol: \mathcal{P}(\R^d) \longrightarrow [0.\infty]
\end{equation}
Unsere grundlegende Forderung ist die Additivität von Volumina bei Zerlegungen, also
\begin{enumerate}
\item[(i)] \textbf{(endliche) Additivität}: Sind $M_1,...,M_n\subset \R^d$ paarweise disjunkt, so gilt
\begin{equation}
\vol (M_1\cup ... \cup M_n) = \vol(M_1)+...+\vol (M_n)
\end{equation}
\end{enumerate}
Volumina als geometrische Größen sollten durch die metrische Struktur (Längenmessung) bestimmt sein, also invariant unter Symmetrien der metrischen Struktur:
\begin{enumerate}
\item[(ii)] \textbf{Bewegungsinvarianz}: Für jede Bewegung $\phi: \R^d \longrightarrow \R^d$ und jede Teilmenge $A\subset \R^d$ gilt
\begin{equation}
\vol(\phi(A)) = \vol (A)
\end{equation}
\item[(iii)] \textbf{Normierung}: $\vol([0,1]^d)=1$.
\end{enumerate}
Verstärke Forderung (i): (Borel, Lebesgue)
\begin{enumerate}
\item[(i')]\textbf{$\sigma$-Additivität\footnote{$\sigma$: abzählbar, unendlich oft.}}: Für Folgen $(M_n)_{n\in \N}$ paarweise disjunkter Teilmengen $M_n \subset \R^d$ gilt:
\begin{equation}
\vol\bigg( \bigcup_{n\in \N} M_n \bigg) = \sum_{n\in \N} \underbrace{\vol (M_n)}_{\in [0,\infty]} 
\end{equation}
\begin{remark}
Wegen des Umordnungssatzes spielt die Reihenfolge der Summanden keine Rolle, da sie alle positiv sind.
\end{remark}
$\leadsto$ flexibilisiert Volumenmessung entscheidend, wir können also komplizierte Figuren durch einfach Figuren approximieren.
\end{enumerate}
Cantons Mengenlehre $\leadsto$ Existenz von ``naiver'' Volumenfunktion wurde hinterfragt: %\vspace{-10pt}
\paragraph{Maßproblem (naiv)} Existiert eine Volumenfunktion $\vol: \mathcal{P}(\R^d) \longrightarrow [0,\infty]$ mit (i') + (ii) + (iii)?
\begin{satz}[Vitali, 1905] Nein, das naive Maßproblem ist unlösbar.
\begin{proof}
Aus dem Auswahlaxiom folgt die Existenz ``verrückter'' (d.h. geometrisch unvorstellbarer) Teilmengen des $\R^d$. Hier existiert $M \subset \R^d$, ein \emph{Vertretersystem} für Nebenklassen von $\mathbb{Q}^d$ (Untergruppe von $\R^d$) in $\R^d$. Der Quotient abelscher Gruppen $\R^d/\mathbb{Q}^d$ ist also die Menge der Nebenklassen. Die Nebenklassen $a+\mathbb{Q}^d$ für $a\in \R^d$ partitionieren (d.h. zerlegen disjunkt) $\R^d$ (überabzählbar viele). Für alle $a,b \in \R^d$ besteht Dichotomie:
\begin{enumerate}[label=\roman*),topsep=3pt, itemsep=0pt]
\item entweder $a+\mathbb{Q}^d = b + \mathbb{Q}^d$ (nämlich wenn $a-b \in \mathbb{Q}^d$),
\item oder $(a+\mathbb{Q}^d)\cap (b+\mathbb{Q}^d) = \emptyset$ (nämlich wenn $a-b \notin \mathbb{Q}^d$).
\end{enumerate}
D.h. für alle $a\in \R^d$ besteht $M \cap (a+\mathbb{Q}^d)$ aus genau einem Element. Daraus folgt, die Translate $q+M$ (abzählbar viele) für $q \in \mathbb{Q}^d$ partitionieren $\R^d$. Aus der $\sigma$-Additivität von Volumen folgt
\begin{equation}
\underbrace{\vol (\R^d)}_{>0}= \sum_{q \in \mathbb{Q}^d} \underbrace{\vol (q+M)}_{\overset{\text{Bew Inv}}{=} \vol(M)}
\end{equation} 
und somit also $\vol (M)>0$.\\
Jetzt wähle $M$ spezieller, nämlich beschränkt, z.B. für $O\subset \R^d$ offen können wir $M$ so wählen, dass $M\subset O$, weil $a+\mathbb{Q}^d$ dicht in $\R^d$, also $(a+\mathbb{Q}^d)\cap O \neq \emptyset$. Z.B. wähle $M \subset (0,\frac{1}{2})^d$, so enthält $[0,1]^d$ abzählbar unendlich viele paarweise disjunkte Translate $q+M$, nämlich für alle $q\in \mathbb{Q}^d \cap (0,\frac{1}{2})^d$ gilt
\begin{equation}
V:= \bigcup_{q\in(0,\frac{1}{2})^d \cap \mathbb{Q}^d} (q+M) \subset [0,1]^d
\end{equation}
weil $\vol (V) + \underbrace{\vol ([0,1]^d-V)}_{\geq 0} = \underbrace{\vol ([0,1]^d)}_{=1}$. Daraus folgt $\vol (V) \leq 1 < \infty$ und 
\begin{equation}
\vol(V) = \sum_{q\in(0,\frac{1}{2})^d \cap \mathbb{Q}^d} \underbrace{\vol (q+M)}_{=\vol(M)}
\end{equation}
Somit muss gelten $\vol(M)=0$. \Lightning
\end{proof}
\end{satz}
Noch dramatischer: In $\dim \geq 3$ kann man je zwei Teilmengen (unter sehr allgemeinen Annahmen) aus demselben (abzählbaren, oft sogar endlichen) ``Bausatz'' zusammensetzen.
\begin{satz}[Banach-Tarski, 1924] Seien $A,B\subset \R^d$ Teilmengen mit nichtleerem Inneren.
\begin{enumerate}[(\roman*)]
\item Sei $d \geq 3$ und seien $A,B$ beschränkt. Dann existieren endlich viele Teilmengen $M_k \subset \R^d$ und Bewegungen $\phi_k$ des $\R^d$, so dass \emph{disjunkte Zerlegungen} $A=\bigsqcup_k M_k$ und $B = \bigsqcup_k \phi(M_k)$ bestehen.
\item Jetzt $d\geq 1$ beliebig und $A,B$ nicht notwendig beschränkt. Dann existieren abzählbar viele Teilmengen $M_k \subset \R^d$ und Bewegungen $\phi_k$, sodass  \emph{disjunkte Zerlegungen} $A=\bigsqcup_k M_k$ und $B = \bigsqcup_k \phi(M_k)$ bestehen.
\end{enumerate}
Der Beweis verwendet Gruppentheorie, Struktur von orthogonalen Gruppen $\mathrm{O}(d)$. (nicht mehr auflösbar für $d\geq 3$.)
\end{satz}
Das naive \emph{Inhaltsproblem}, also eine Volumenfunktion mit Eigenschaften (i), (ii) und (iii), ist lösbar in $d\leq 2$, aber nicht eindeutig, nicht lösbar in $d\geq 3$. (Banach 1923, Hausdorff 1914) Dies führt zu:
\vspace{-10pt}
\paragraph{Maßproblem (post-paradox)}: Man definiere eine Volumenfunktion $\vol: \mathcal{F} \longrightarrow [0,\infty]$ mit Eigenschaften (i'), (ii) und (iii) auf einer möglich großen und flexiblen Familie $\mathcal{F} \subset \mathcal{P}(\R^d)$, die die geometrisch wichtigen Teilmengen umfasst und abgeschlossen ist unter grundlegenden mengentheoretischen Operationen (Vereinigung, Schnitt, Differenz und Komplement). 
\marginpar{\tiny{17.10.2019}}
\end{document}
