\newpage
\appendix
%\subsection*{Appendix}
\subsubsection*{Integrale stetiger Funktionen einer Variable} %\small (einer reellen Variable)}
\marginpar{\tiny{23.10.2019}}
Wir unterscheiden zwischen
\begin{enumerate}[- ,topsep =-3pt]
	\item dem \emph{bestimmten Integral}
				$$ \int\limits_a^b f(x)\D x$$
	\item und dem \emph{unbestimmten Integral}, d.h. die Menge der Funktionen dieser Art
	$$x \mapsto \int\limits_a^x f(\xi) \D \xi + \underset{\text{\tiny const}}C$$
	Notation: $\int f \D x$.
\end{enumerate}

\begin{satz}[\textbf{Hauptsatz der Differential- und Integralrechnung}]  \begin{mdframed} \
\begin{enumerate}[(\roman*), topsep = -1 pt]
	\item Ist $f$ von der Klasse $\mathcal{C}^0$ (d.h. stetig), so ist 
		$$\left(\int f \D x\right)' = f$$
		d.h. die Repräsentanten des unbestimmten Integrals sind Stammfunktionen.
	\item Ist $f$ von der Klasse $\mathcal{C}^1$ (d.h. stetig differenzierbar), so ist
	$$ \int F' \D x =F$$
	(zu lesen: $F$ repräsentiert $\int F' \D x)$ bzw.
	$$ \int\limits_a^x F'(\xi) \D x = F(x) -F(a)$$
\end{enumerate}
\end{mdframed}
\end{satz}

\paragraph{Rechenregeln für Differentialrechnung $\leadsto$ Rechenregeln für Integralrechnung}z.B.
\begin{equation*}
	\begin{split}
		(\ln |x|)' = \frac{1}{x} & \implies \int \frac{\D x}{x} = \ln |x| \text{ auf } \R\setminus\{0\}\\
		\arcsin' x {=\frac{1}{\sqrt{1-x^2}}} & \implies \int \frac{\D x}{\sqrt{1-x^2}} = \arcsin x \text{ auf } (-1,1) \\ 		
		\arctan' x = \frac{1}{1+x^2} & \implies \int \frac{\D x}{1+x^2} = \arctan x \text{ auf }\R
	\end{split}
\end{equation*}
\paragraph{Kettenregel $\leadsto$ Substitutionsregel} Aus der Kettenregel
$$(F \circ \varphi)' (u) = F'(\varphi (u)) \cdot \varphi'(u)$$
folgt mit dem Hauptsatz der Differential- und Integralrechnung: Sei $f := F'$ ($\mathcal{C}^0$),
$$ \int\limits_a^b f(\varphi(u)) \cdot \varphi'(u) \D u = \int\limits_a^b (F \circ \varphi) (u) \D u= F \circ \varphi \big\vert_a^b = F\big\vert^{\varphi(b)}_{\varphi(a)} = \int\limits_{\varphi(a)}^{\varphi(b)} f \D x
$$
Also die \textbf{Substitutionsregel} (Bezeichne $I:= (a,b), J = (\phi(a),\phi(b))$)
$$
\boxed{
\int\limits_a^b f(\varphi(u)) \varphi'(u) \D u = \int_{\phi(a)}^{\phi(b)} f(x) \D x 
}
$$
und die \textbf{Version für unbestimmtes Integral}
$$ \int f(\phi(u)) \phi'(u) \D u = \underbrace{\int f(x) \D x \Big \vert_{x =\phi(u)}}_{\substack{\text{die Komposition }\phi \\ \text{ mit } \int f(x)\D x}} $$
\begin{example} \
\begin{enumerate}
 \item \textbf{Lineare Substitution} mit $x = u+\alpha, \alpha \in \R$ 
 $$
	\int_a^b f(u+\alpha) \D x = \int_{a + \alpha}^{b+ \alpha} f(x) \D x
 $$
 bzw. 
 $$
 	\int f( u +\alpha) \D u = \int f(x) \D x \Big\vert_{x = u + \alpha}
 $$
 z.B. $f(x) = \frac{1}{x}$ auf $\R \setminus \{0\}$
 $$ \int_a^b \frac{\D u}{u+\alpha} = \int_{a+\alpha}^{b+\alpha} \frac{\D x}{x} = \ln |x| \big\vert^{b+\alpha}_{a+\alpha} = \ln \left| \frac{b+\alpha}{a+\alpha} \right|$$
 bzw.
 $$ \int \frac{\D u}{u+\alpha} = \int \frac{\D x}{x} \Big\vert_{x= u+\alpha}= \ln |x| \big\vert_{x = u + \alpha} = \ln |u + \alpha|$$
 \item[(i')] \textbf{(Multiplikative) lineare Substitution} mit $x= \lambda u (\lambda \in \R \setminus \{0\})$
 	$$ \int_a^b f(\lambda u) \D u = \frac{1}{\lambda} \int_{\lambda a}^{\lambda b} f(x) \D x$$ bzw. 
 	$$ \int f(\lambda u) \D u = \frac{1}{\lambda} \int f(x) \D x \Big\vert_{x = \lambda u}$$
 	z.B.
 $$\int \cos \lambda u \D u = \frac{1}{\lambda} \int \underbrace{\cos x}_{\sin' x} \D x \Big\vert_{x= \lambda u} = \frac{1}{\lambda} \sin \lambda u$$
 \item \textbf{Quadratische Substitution} mit $x = u^2$
 $$ \int_a^b f(u^2) u \D u =\frac{1}{2} \int_{a^2}^{b^2} f(x) \D x$$
 bzw.
 $$ f(u^2) u \D u = \frac{1}{2} f(x) \D x \Big\vert_{x=u^2}$$
 z.B. $f(x) = e^x$:
 $$ \int u  e^{u^2} \D u = \frac{1}{2} \int e^x \D x \Big\vert_{x=u^2} = \frac{1}{2} e^{u^2}$$
 \item Mit $f(x) = \frac{1}{x}$, (falls $\phi|_J$ keine Nullstelle hat)
 $$ \int_a^b \frac{\phi'(u)}{\phi(u)} \D u = \int_{\phi(a)}^{\phi(b)}\frac{\D x}{x} = \ln |x| \big\vert^{\phi(b)}_{\phi(a)} = \ln |\phi(u)| \big\vert_{a}^b$$
 bzw.
 $$ \int \frac{\phi'(u)}{ \phi(u)} \D u = \int \frac{\D x}{x} \Big\vert_{x=\phi(u)} = \ln | \phi(u) |$$
 z.B. $\phi(u)= \cos u$ auf $(-\frac{\pi}{2},\frac{\pi}{2})$
 $$\int \tan u \D u  = \int -\frac{\cos' u}{\cos u} \D u= -\ln |\cos u|$$
 Berechne $\int \frac{\D x}{\sqrt{1+x^2}}$ auf $\R (=I =J)$. Substituiere $x=\sinh u$ mit der Umkehrfunktion $u = \operatorname{arsinh} x$.
 $$\int \frac{\D x}{\sqrt{1+x^2}} = \int \frac{\sinh' u}{\sqrt{1+\sinh^2 u}} \D u \Bigg\vert_{u = \operatorname{arsinh}x}= \int \frac{\cosh' u}{\cosh' u} \D u\Big\vert_{u = \operatorname{arsinh}x} = \operatorname{arsinh}x$$
\end{enumerate}
\rule{\textwidth}{0.4pt}
\marginpar{\tiny{30.10.2019}}
Die \emph{Produktregel} für die Ableitung führt zur Methode der \emph{partiellen} Integration.
\paragraph{Partielle Integration:} Für $\mathcal{C}^1$ Funktionen $f,g:I\to \C$ auf einem offenen Intervall $I \subset \R$ gilt:
\begin{equation*}
	\boxed{
	\int_a^b f' \cdot g \D x = f\cdot g \big\vert^b_a - \int^b_a f \cdot g' \D x	
	}
\end{equation*}
für $a,b \in I$ Mann nennt $f\cdot g \vert^b_a$ \emph{Randterm}. \\\\
Für unbestimmte Integrale schreibt man
\begin{equation*}
	\boxed{
	\int f' g\D x = f \cdot g \big\vert -  \int f \cdot g' \D x
	}
\end{equation*}
Man kann diese Gleichung lesen als eine Gleichheit von Funktionenmengen oder so, dass jeder Repräsentant der rechten Seite $f \cdot g \big\vert -  \int f \cdot g'$ ein Repräsentant der linken Seite $\int f' g$ ist.
\begin{proof}
Nach der Produktregel ist $f \cdot g$ Stammfunktion von $f'g+fg'$. Der Hauptsatz der Differential- und Integralrechnung liefert dann 
$$ \int (f' g + fg') = fg \big\vert$$
\end{proof}

\begin{example}
\begin{enumerate}
	\item Berechnung von $\int \ln x \D x$ auf $(0,\infty)$. Dort gilt wegen $\ln' x = \frac{1}{x}$
	\begin{equation*}
		\int \ln x \D x = \int (x)'\ln x \D x = x\ln x \big\vert - \int x \cdot \frac{1}{x} \D x = x (\ln x -1)\big\vert
	\end{equation*}
	\item Berechnung von $\int x e^x \D x$ auf $\R$.
	\begin{equation*}
		\int x e^x \D x = \int x (e^x)' \D x = x e^x\big\vert - \int \underbrace{(x)'}_{=1} e^x \D x =e^x (x-1)
	\end{equation*}
	\item[(ii)'] Berechnung von $\int x^n e^x \D x$ auf $\R$.
	\begin{equation*}
		I_n(x) := \int x^n e^x \D x  = \int x^n (e^x)' \D x = x^ne^x - n \underbrace{\int x^{n-1} e^x \D x}_{I_{n-1}(x)}
	\end{equation*}
	Wir erhalten die Rekursionsformel 
	$$I_n (x) = x^n e^x \big\vert - n I_{n-1}(x)$$
	\item Berechnung von $\int\sqrt{1-x^2}\D x $ auf $(-1,1)$.
	\begin{equation*}
	\begin{split}
		\int \sqrt{1-x^2} \D x= \int (x') \sqrt{1-x^2} \D x & = x \sqrt{1-x^2} \big\vert + \int x  \frac{-x}{\sqrt{1-x^2}} \D x \\
		& = x \sqrt{1-x^2}\big\vert + \int \underbrace{\frac{1-x^2}{\sqrt{1-x^2}}}_{\sqrt{1-x^2}}\D x + \underbrace{\int \frac{1}{\sqrt{1-x^2}} \D x}_{\arcsin x} \\
		& = \frac{1}{2}\cdot \Big( x \sqrt{1-x^2} + \arcsin (x) \Big) \Big\vert
	\end{split}
	\end{equation*}
	\begin{remark}
	Die Regel für die Berechnung der Ableitung von Umkehrfunktion ist
	$$ (f^{-1})' = \frac{1}{f'(f^{-1})} $$
	\end{remark}
	und somit haben wir die Ableitung von $\arcsin$:
	$$\arcsin' (x) = \frac{1}{\cos (\arcsin (x))} = \frac{1}{\sqrt{1-\sin^2\big(\arcsin(x)\big)}} = \frac{1}{1-x^2}$$
	Insbesondere erhalten wir durch Grenzübergang \begin{small} (Hier ist der Grenzübergang nötig, da $\sqrt{1-x^2}$ nicht stetig differenzierbar in Punkten $-1$ und $1$ sind) \end{small} für das abgeschlossene Intervall $[-1,1]$:
	\begin{equation*}
		\begin{split}
		\int_{-1}^1 \sqrt{1-x^2} \D x = \lim\limits_{\varepsilon \to 0} \int_{-1+\varepsilon}^{1-\varepsilon} \sqrt{1-x^2} \D x & = \lim\limits_{\varepsilon \to 0} \frac{1}{2} \big(x \sqrt{1-x^2} + \arcsin x\big) \Big\vert_{-1+\varepsilon}^{1-\varepsilon} \\
		& = \frac{1}{2} \left( \frac{\pi}{2} - \left( - \frac{\pi}{2} \right) \right) \\ 
		& = \frac{\pi }{2}
		\end{split}
	\end{equation*}
	Dies zeigt insbesondere, dass die Fläche der Einheitsscheibe $\pi$ ist.  \\ Außerdem können wir das Integral auch mit Substitution berechnen: 
	\begin{equation*}
	\begin{split}
	 \int_{-1}^1 \sqrt{1-x^2} \D x & \overset{x= \sin u}= \int^{\pi/2}_{-\pi/2} \sqrt{1-\sin^2(u)} \cos u \D u  = \int_{-\pi/2}^{\pi/2} \underbrace{\cos^2 u}_{ \frac{1+\cos 2u}{2}} \D u \\
		& = 	\Big(\frac{u}{2}+\frac{1}{4}\sin 2u\Big) \Big\vert_{-\pi/2}^{\pi/2} = \frac{\pi}{2}
	 \end{split}
	\end{equation*}
	\item Berechnung von $\int \arctan$: Wir bemerken, dass $\arctan ' (x)=\frac{1}{1+x^2}$, denn $\tan'(x)=1+\tan^2x $.
	\begin{equation*}
	\begin{split}
	  \int \arctan(x) \D x &= \int (x')\arctan(x) \D x = x \cdot \arctan x \big\vert - \int \frac{1}{2} \frac{2x}{1+x^2} \D x \\
	  &  \overset{t = x^2}= x \cdot \arctan x - \int \frac{1}{2} \cdot \frac{1}{1+t} \D t \Big\vert_{t=x^2} \\
	  & = x \cdot \arctan x - \frac{1}{2} \ln (1+x^2)
	  	\end{split}
	\end{equation*}
	\item Berechnung von $\int \arcsin(x)$ auf $(-1,1)$
	\begin{equation*}
	\begin{split}
		\int \arcsin(x) \D x = \int (x)' \arcsin(x) \D x & = x\arcsin(x) \big\vert - \int \frac{1}{2} \frac{2 x \D x}{\sqrt{1-x^2}} \\
		& \overset{t:= x^2}= x\arcsin (x) \big\vert - \int \frac{1}{2} \frac{\D t}{\sqrt{1-t}}  \Big\vert_{t=x^2}	 \\
		& = x\arcsin x + \sqrt{1-t}\big\vert_{t=x^2} \\
		& = x\arcsin x + \sqrt{1-x^2} \big\vert
	\end{split}
	\end{equation*}
	\item Berechnung von $\int \sin^2x \D x$. Da
	\begin{equation*}
		\int \sin^2 x \D x = \int (-\cos(x))' \sin x \D x = -\cos x \sin x + \int \underbrace{\cos^2 x}_{1-\sin^2 x} \D x
	\end{equation*}
	erhalten wir
	$$
	\int \sin^2 x \D x = \frac{1}{2} (-\cos x \sin x + x )	
	$$
	\item[(vi)'] Berechnung von $\int \sin^n(x) \D x$ für $n \in \N$.
	\begin{equation*}
		\begin{split}
		I_n (x) := \int \sin^n (x) \D x & = \int (-\cos(x))' \sin^{n-1}(x)\D x \\
				& = -\cos x \sin^{n-1} x + \int 	\underbrace{\cos^2 (x)}_{1-\sin^2(x)} (n-1) \sin^{n-2} (x) \D x \\
				&  = -\cos x \sin^{n-1} x +(n-1) \left( \underbrace{\int \sin^{n-2} (x) \D x}_{I_{n-2}(x)} - \underbrace{\int \sin^n(x)\D x}_{I_n(x)} \right)
		\end{split}
	\end{equation*}
	Wir erhalten:
	$$n \cdot I_n(x) = -\cos (x)\sin^{(n-1)}(x) + (n-1) I_{n-2}(x)$$
	Zum Beispiel gilt $I_0(x)=x$, $I_1(x)=-\cos x $, $I_2(x)=\frac{1}{2}(x-\sin x\cos x)$, $I_3(x)=\frac{1}{3} \cos^3 x - \cos x$.
\end{enumerate}
\end{example}
\newline
\rule{\textwidth}{0.4pt}

\paragraph{Rationale Funktionen:} $\int \frac{p(x)}{q(x)} \D x$ mit $p,q$ Polynome, $q \neq 0$.Zunächst $p,q \in \C[x]$:
\begin{satz}[\textbf{Fundamentalsatz der Algebra}] \begin{mdframed} Jedes Polynom $p \in \C[x]$ vom Grad $n= \deg (p)$ besitzt eine Zerlegung in Linearfaktoren:
$$ p(x) = c \cdot \prod_{n=1}^m (x-\alpha_k)^{n_k}$$
mit $m \leq n, c \in \C \setminus \{0\}, \alpha_1,...,\alpha_m \in \C, \sum_{k=1}^m n_k = n$. Hierbei sind $\alpha_1,...,\alpha_m$ die verschiedenen Nullstellen von $p$ und $n_k$ die Vielfachheit von $\alpha_k$. Die Linearfaktorzerlegung ist bis auf Vertauschung von Faktoren eindeutig.
\end{mdframed}
Polynomdivision führt zu:
$$\frac{p(x)}{q(x)} = s(x) + \frac{r(x)}{q(x)} \quad \text{mit } s\in \C [x] \text{ und } \deg r < \deg q$$
\end{satz}

\paragraph{Partialbruchzerlegung:} Sei $q(x) \in \C [x]$ mit $q(x) = c \cdot \prod_{n=1}^m (x-\alpha_k)^{n_k} $. \linebreak Für jedes komplexe Polynom $r(x) \in \C[x]$ mit $\deg r < \deg q =n $ existiert eine eindeutige Zerlegung:
$$
\frac{r(x)}{q(x)} = \sum_{k=1}^m \sum_{j=1}^{n_k} \frac{c_{kj}}{(x-x_k)^j}, \quad c_{jk} \in \C
$$
d.h.
$$r(x) = \sum_{k=1}^m \sum_{j=1}^{n_k} b_{kj}(x) c_{kj}$$
mit 
$$b_{kj} = (x-\alpha_k)^{n_k-j} \cdot \prod_{\substack{l=1 \\ l \neq k}}^m (x-\alpha_l)^{n_l}$$
Anderes gesagt, $\{b_{kj}\}$ für $k=1,...,m$ und $j=1,...,n_k$ bilden Basis von $\C_{\deg<n}[x]$. Insbesondere $\dim( \C_{\deg <n}[x])=n$.

\begin{example}
$$\int \frac{1}{1-x^2} \D x = \frac{1}{2} \int \frac{1}{1-x} \D x + \frac{1}{2} \int \frac{1}{1+x} \D x = \frac{1}{2} \ln \left| \frac{1+x}{1-x} \right|$$
\end{example}
\end{example}