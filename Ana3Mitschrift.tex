\documentclass[12pt,a4paper]{article}
\usepackage[T1]{fontenc}
\usepackage[utf8]{inputenc}
\usepackage{lmodern}
\usepackage[left=28mm,top=28mm,right=28mm,bottom=28mm] {geometry}
\usepackage{amsfonts}
\usepackage{mathrsfs}
\usepackage[intlimits]{amsmath}
\usepackage{stmaryrd}
\usepackage{relsize}
\usepackage{etoolbox}
\usepackage[ngerman]{babel}
\usepackage[utf8]{inputenc}
\usepackage[T1]{fontenc}
\usepackage{marvosym}
\usepackage[shortlabels]{enumitem}
\usepackage{mathtools}
\usepackage{amssymb}
\usepackage{cancel}
\usepackage{mdframed}
\usepackage{framed}
\usepackage{mathtools}
\usepackage{tablefootnote} 
\usepackage{listings}
\usepackage{amsthm}
\usepackage{xcolor}
\usepackage{etoolbox}
\usepackage[all]{xy}
\usepackage{tikz}
\usepackage{thmtools}
\usepackage[
   pdfpagelabels=true,
   pdftitle={Analysis III: Maßtheorie und Integralrechnung mehrerer Variablen},
   pdfauthor={MT},
 ]{hyperref}
\usepackage{bookmark}
\usetikzlibrary{cd}
\usetikzlibrary{calc}
\theoremstyle{definition}
\newtheorem*{definition}{Definition}
\newtheorem*{satz}{Satz}
\newtheorem*{lemma}{Lemma}
\newtheorem*{proposition}{Proposition}
\newtheorem*{korollar}{Korollar}
\newtheorem*{folgerung}{Folgerung}
\newtheorem*{example}{Beispiel}
\newtheorem*{hauptbsp}{Hauptbeispiel}
\theoremstyle{remark}
\newtheorem*{remark}{Bemerkung}
\newtheorem*{remark'}{Nebenbemerkung}
\newtheorem*{beobachtung}{Beobachtung}
\AfterEndEnvironment{lemma}{\noindent\ignorespaces}
\AfterEndEnvironment{definition}{\noindent\ignorespaces}
\AfterEndEnvironment{example}{\noindent\ignorespaces}
\AfterEndEnvironment{theorem}{\noindent\ignorespaces}
\AfterEndEnvironment{satz}{\noindent\ignorespaces}
\AfterEndEnvironment{korollar}{\noindent\ignorespaces}
\AfterEndEnvironment{remark}{\noindent\ignorespaces}
\AfterEndEnvironment{remark'}{\noindent\ignorespaces}
\AfterEndEnvironment{proposition}{\noindent\ignorespaces}
\AfterEndEnvironment{proof}{\noindent\ignorespaces}
\let\existstemp\exists
\let\foralltemp\forall
\newcommand{\tikzmark}[1]{\tikz[overlay,remember picture] \node (#1) {};}
\newcommand{\vsubset}{\rotatebox[origin=c]{90}{$\subset$}}
\newcommand{\vphi}{\phi}
\newcommand{\ol}{\overline}
%Differentiation
\newcommand{\D}{\, \mathrm{d}}
%Bold Symbols
\newcommand{\R}{\mathbb{R}}
\newcommand{\C}{\mathbb{C}}
\newcommand{\N}{\mathbb{N}}
\newcommand{\Q}{\mathbb{Q}}
%Calligraphic Symbols
\newcommand{\II}{\mathcal{I}}
\newcommand{\FF}{\mathcal{F}}
\newcommand{\QQ}{\mathcal{Q}}
\newcommand{\EE}{\mathcal{E}}
\newcommand{\PP}{\mathcal{P}}
\newcommand{\TT}{\mathcal{T}}
\newcommand{\HH}{\mathscr{H}}
\newcommand{\RR}{\mathscr{R}}
\newcommand{\BB}{\mathscr{B}}
\newcommand{\CC}{\mathscr{C}}
\newcommand{\DD}{\mathscr{D}}
\renewcommand{\AA}{\mathscr{A}}
% Operator names
\newcommand{\Hom}{\operatorname{Hom}}
\newcommand{\del}{\partial}
\newcommand{\vol}{\operatorname{vol}}
\newcommand{\Var}{\operatorname{Var}} 
\newcommand{\Cov}{\operatorname{Cov}}
\newcommand{\End}{\operatorname{End}}
\newcommand{\SL}{\operatorname{SL}}
\newcommand{\Bild}{\begin{tiny}(Bild hier)\end{tiny}}
\renewcommand*{\exists}{\existstemp\mkern2mu}
\renewcommand*{\forall}{\foralltemp\mkern2mu}
\renewcommand{\emptyset}{\varnothing}
\renewcommand{\Re}{\operatorname{Re}}
\renewcommand{\Im}{\operatorname{Im}}
\renewcommand{\qedsymbol}{$\blacksquare$}
\renewcommand{\phi}{\varphi}
\renewcommand{\thesection}{\Roman{section}}
\makeatletter 
\AfterEndEnvironment{mdframed}{%
 \tfn@tablefootnoteprintout% 
 \gdef\tfn@fnt{0}% 
}
\numberwithin{equation}{section}

\setlist[enumerate,1]{label={(\roman*)}}

\title{Mitschrift der Vorlesung \\ \huge Analysis III   \\ \vspace{0.2pc} \Large Maßtheorie und Integralrechnung mehrerer Variablen\footnote{im Wintersemester 2019/20 gelesen von Prof. Bernhard Leeb, Ph.D.}}
%\author{\normalsize bei Prof. Bernhard Leeb, Ph.D. \vspace{0.2pc}}
%\date{27. Oktober 2019}
\begin{document}
\maketitle
\tableofcontents 
\addtocontents{toc}{\protect\setcounter{tocdepth}{3}}
\newpage
\setcounter{page}{2}
\section{Maßtheorie}
\subsection{Maßproblem und Paradoxien}
\marginpar{\tiny{14.10.2019}}
Maßtheorie ist die Theorie des Volumens. Motivierende Beispiele sind:
\begin{enumerate}[label=\roman*),topsep=3pt, itemsep=0pt]
\item Volumina von Teilmengen des euklidischen Raums
\item Wahrscheinlichkeiten (= ``Volumina von Ereignissen'')
\end{enumerate}
Wir konzentrieren uns im Rest des Abschnitts auf $\R^d$. Wir wollen leistungsfähigen Volumenbegriff haben, sodass die Volumina von möglich vielen Teilmengen flexibel gemessen werden können. Unser erster ``naiver'' Ansatz wäre, dass wir Volumenmessung für \emph{alle} Teilmengen verlangen, also eine Funktion
\begin{equation*}
 \vol: \mathcal{P}(\R^d) \longrightarrow [0.\infty]
\end{equation*}
Unsere grundlegende Forderung ist die Additivität von Volumina bei Zerlegungen, also
\begin{enumerate}
\item[(i)] \textbf{(endliche) Additivität}: Sind $M_1,...,M_n\subset \R^d$ paarweise disjunkt, so gilt
\begin{equation*}
\vol (M_1\cup ... \cup M_n) = \vol(M_1)+...+\vol (M_n)
\end{equation*}
\end{enumerate}
Volumina als geometrische Größen sollten durch die metrische Struktur (Längenmessung) bestimmt sein, also invariant unter Symmetrien der metrischen Struktur:
\begin{enumerate}
\item[(ii)] \textbf{Bewegungsinvarianz}: Für jede Bewegung $\phi: \R^d \longrightarrow \R^d$ und jede Teilmenge $A\subset \R^d$ gilt
\begin{equation*}
\vol(\phi(A)) = \vol (A)
\end{equation*}
\item[(iii)] \textbf{Normierung}: $\vol([0,1]^d)=1$.
\end{enumerate}
Verstärke Forderung (i): (Borel, Lebesgue)
\begin{enumerate}
\item[(i')]\textbf{$\sigma$-Additivität\footnote{$\sigma$: abzählbar, unendlich oft.}}: Für Folgen $(M_n)_{n\in \N}$ paarweise disjunkter Teilmengen $M_n \subset \R^d$ gilt:
\begin{equation*}
\vol\bigg( \bigcup_{n\in \N} M_n \bigg) = \sum_{n\in \N} \underbrace{\vol (M_n)}_{\in [0,\infty]} 
\end{equation*}
\begin{remark}
Wegen des Umordnungssatzes spielt die Reihenfolge der Summanden keine Rolle, da sie alle positiv sind.
\end{remark}
$\leadsto$ flexibilisiert Volumenmessung entscheidend, wir können also komplizierte Figuren durch einfach Figuren approximieren.
\end{enumerate}
Cantons Mengenlehre $\leadsto$ Existenz von ``naiver'' Volumenfunktion wurde hinterfragt: %\vspace{-10pt}
\paragraph{Maßproblem} Existiert eine Volumenfunktion $\vol: \mathcal{P}(\R^d) \longrightarrow [0,\infty]$ mit (i') + (ii) + (iii)?
\begin{satz}[Vitali, 1905] Nein, das naive Maßproblem ist unlösbar.
\begin{proof}
Aus dem Auswahlaxiom folgt die Existenz ``verrückter'' (d.h. geometrisch unvorstellbarer) Teilmengen des $\R^d$. Hier existiert $M \subset \R^d$, ein \emph{Vertretersystem} für Nebenklassen von $\mathbb{Q}^d$ (Untergruppe von $\R^d$) in $\R^d$. Der Quotient abelscher Gruppen $\R^d/\mathbb{Q}^d$ ist also die Menge der Nebenklassen. Die Nebenklassen $a+\mathbb{Q}^d$ für $a\in \R^d$ partitionieren (d.h. zerlegen disjunkt) $\R^d$ (überabzählbar viele). Für alle $a,b \in \R^d$ besteht Dichotomie:
\begin{enumerate}[label=\roman*),topsep=3pt, itemsep=0pt]
\item entweder $a+\mathbb{Q}^d = b + \mathbb{Q}^d$ (nämlich wenn $a-b \in \mathbb{Q}^d$),
\item oder $(a+\mathbb{Q}^d)\cap (b+\mathbb{Q}^d) = \emptyset$ (nämlich wenn $a-b \notin \mathbb{Q}^d$).
\end{enumerate}
D.h. für alle $a\in \R^d$ besteht $M \cap (a+\mathbb{Q}^d)$ aus genau einem Element. Daraus folgt, die Translate $q+M$ (abzählbar viele) für $q \in \mathbb{Q}^d$ partitionieren $\R^d$. Aus der $\sigma$-Additivität von Volumen folgt
\begin{equation*}
\underbrace{\vol (\R^d)}_{>0}= \sum_{q \in \mathbb{Q}^d} \underbrace{\vol (q+M)}_{\overset{\text{Bew Inv}}{=} \vol(M)}
\end{equation*} 
und somit also $\vol (M)>0$.\\
Jetzt wähle $M$ spezieller, nämlich beschränkt, z.B. für $O\subset \R^d$ offen können wir $M$ so wählen, dass $M\subset O$, weil $a+\mathbb{Q}^d$ dicht in $\R^d$, also $(a+\mathbb{Q}^d)\cap O \neq \emptyset$. Z.B. wähle $M \subset (0,\frac{1}{2})^d$, so enthält $[0,1]^d$ abzählbar unendlich viele paarweise disjunkte Translate $q+M$, nämlich für alle $q\in \mathbb{Q}^d \cap (0,\frac{1}{2})^d$ gilt
\begin{equation*}
V:= \bigcup_{q\in(0,\frac{1}{2})^d \cap \mathbb{Q}^d} (q+M) \subset [0,1]^d
\end{equation*}
weil $\vol (V) + \underbrace{\vol ([0,1]^d-V)}_{\geq 0} = \underbrace{\vol ([0,1]^d)}_{=1}$. Daraus folgt $\vol (V) \leq 1 < \infty$ und 
\begin{equation*}
\vol(V) = \sum_{q\in(0,\frac{1}{2})^d \cap \mathbb{Q}^d} \underbrace{\vol (q+M)}_{=\vol(M)}
\end{equation*}
Somit muss gelten $\vol(M)=0$. \Lightning
\end{proof}
\end{satz}
Noch dramatischer: In $\dim \geq 3$ kann man je zwei Teilmengen (unter sehr allgemeinen Annahmen) aus demselben (abzählbaren, oft sogar endlichen) ``Bausatz'' zusammensetzen.
\begin{satz}[Banach-Tarski, 1924] Seien $A,B\subset \R^d$ Teilmengen mit nichtleerem Inneren.
\begin{enumerate}[(\roman*)]
\item Sei $d \geq 3$ und seien $A,B$ beschränkt. Dann existieren endlich viele Teilmengen $M_k \subset \R^d$ und Bewegungen $\phi_k$ des $\R^d$, so dass \emph{disjunkte Zerlegungen} $A=\bigsqcup_k M_k$ und $B = \bigsqcup_k \phi(M_k)$ bestehen.
\item Jetzt $d\geq 1$ beliebig und $A,B$ nicht notwendig beschränkt. Dann existieren abzählbar viele Teilmengen $M_k \subset \R^d$ und Bewegungen $\phi_k$, sodass  \emph{disjunkte Zerlegungen} $A=\bigsqcup_k M_k$ und $B = \bigsqcup_k \phi(M_k)$ bestehen.
\end{enumerate}
Der Beweis verwendet Gruppentheorie, Struktur von orthogonalen Gruppen $\mathrm{O}(d)$. (nicht mehr auflösbar für $d\geq 3$.)
\end{satz}
Das naive \emph{Inhaltsproblem}, also eine Volumenfunktion mit Eigenschaften (i), (ii) und (iii), ist lösbar in $d\leq 2$, aber nicht eindeutig, nicht lösbar in $d\geq 3$. (Banach 1923, Hausdorff 1914) Dies führt zu:
\vspace{-10pt}
\paragraph{Maßproblem (post-paradox)}: Man definiere eine Volumenfunktion $\vol: \mathcal{F} \longrightarrow [0,\infty]$ mit Eigenschaften (i'), (ii) und (iii) auf einer möglich großen und flexiblen Familie $\mathcal{F} \subset \mathcal{P}(\R^d)$, die die geometrisch wichtigen Teilmengen umfasst und abgeschlossen ist unter grundlegenden mengentheoretischen Operationen (Vereinigung, Schnitt, Differenz und Komplement). 
\subsection{Ringe und Algebren}
\marginpar{\tiny{17.10.2019}}
Wir untersuchen Familien von Teilmengen (einer festen Menge), die unter grundlegenden (endlichen) Mengenoperationen abgeschlossen/ stabil sind. ($\cup,\cap,\setminus,\complement$) \\
Sie werden Definitionsbereiche der allgemeinsten von uns betrachteten Volumenfunktion sein. (``Inhalte'')

\subsubsection{Die Ringstruktur auf Potenzmengen}
Sei $X$ eine Menge. Die Potenzmenge ist definiert als die Familie aller Teilmengen $\mathcal{P}(X)$. Wir können die Potenzmenge ebenfalls auffassen als
\begin{equation*}
\mathcal{P}(X) \xleftrightarrow[\text{bij}]{\cong} \{0,1\}^X = \{f:X \longrightarrow \{0,1\}\}
\end{equation*}
da
\begin{align*}
A &\longmapsto \chi_A(x) = \begin{cases}
1, \quad & \text{falls } x\in A\\
0, \quad & \text{sonst}
\end{cases}\\
f^{-1}(1) & \longmapsfrom f
\end{align*}
wobei $\chi_A$ de charakteristische Funktion von $A$ ist. \newline \newline
Wir fassen nun $\{0,1\}$ auf als den Körper mit 2 Elementen (Restklassen modulo 2). So ist $\{0,1\}^X$ ein kommutativer Ring mit Eins (multiplikatives Einselement) (im Sinne der Algebra), sogar eine $\mathbb{F}_2$-Algebra.
\begin{remark} Die Addition und Multiplikation von Funktionen erfolgen punktweise:
\begin{enumerate} [-, itemsep=0pt,topsep=3pt]
\item  $(f+g)(x) := f(x)+g(x)$ 
\item $(fg)(x)= f(x) \cdot g(x)$
\end{enumerate}
und $\{0,1\} = \mathbb{F}_2$ ist ein Körper mit zwei Elementen.
\end{remark}
Die Nullelement ist $f \equiv 0$, also $\chi_\emptyset$ und das Einselement ist $\chi_X (\equiv 1)$. Die Addition von charakteristischen Funktionen entspricht der symmetrischen Differenz $A \triangle B$ und die Multiplikation entspricht dem Duchrschnitt von Mengen. Also
\begin{align*}
\chi_A + \chi_B = A \triangle B \\
\chi_A \cdot \chi_B = A \cap B
\end{align*}
Somit ist $(\mathcal{P}(X), \triangle, \cap) \cong (\mathbb{F}_2^X, +, \cdot)$ ein kommutativer Ring mit dem Nullelement $\emptyset$ bzw. $\chi_\emptyset$ und dem Einselement $X$ bzw. $\chi_X$. 
\subsubsection{Ringe und Algebren}
\begin{definition}
\begin{mdframed}
Eine Familie $\mathcal{R}\subset \mathcal{P}(X)$ heißt
\begin{enumerate}[itemsep=0pt,topsep=3pt]
\item[($\rho$)] ein \textbf{Ring} auf $X$, falls sie ein Unterring von $(\mathcal{P}(X), \triangle, \cap)$ ist.
\item[($\alpha$)] eine \textbf{Algebra} auf $X$, falls sie außerdem das Einselement enthält, d.h. $X \in \mathcal{R}$.
\end{enumerate}
\end{mdframed}
\end{definition}
\begin{remark}
\begin{small}
``Algebra'' wird in verschiedenen Bedingungen verwendet, nämlich die Algebra als ein mathematisches Gebiet, eine Algebra als algebraische Struktur im Sinne der Algebra und eine Algebra im Sinne der obigen Definition.
\end{small}
\end{remark}
$(\rho)$ bedeutet $\emptyset \in \mathcal{R}$, abgeschlossen unter Addition ($\triangle$) (dasselbe wie Subtraktion, da $\mod 2$) und Multiplikation ($\cap$), d.h.
\begin{equation*}
A,B \in \mathcal{R} \implies A\triangle B, A\cap B \in \mathcal{R}
\end{equation*} 
d.h. $\triangle$- stabil und $\cap$- stabil. Wir können $\triangle, \cap$ ausdrücken durch $\setminus$ und $\cup$:
\begin{align*}
A \triangle B & = (A \setminus B) \cup (B \setminus A) \\
A \cap B & = A \setminus (A \setminus B)
\end{align*}
und umgekehrt
\begin{align*}
A \setminus B &= (A \triangle B) \cap A \\
A \cup B &= (A \triangle B) \triangle (A \cap B)
\end{align*}
\textit{Bemerkung.} Die letzte Gleichung gilt, da $(A\triangle B)$ und $(A\cap B)$ disjunkt sind. \newline \newline
Daraus folgt die Charakterisierung von Ringen:
\begin{lemma}
\begin{mdframed}
Eine Familie $\mathcal{R}\subset \mathcal{P}(X)$ ist genau dann ein Ring auf $X$, wenn
\begin{enumerate}[(\roman*), topsep=3pt, itemsep=0pt]
\item $\emptyset \in \mathcal{R}$,
\item $\setminus$- stabil, d.h. $A,B \in \mathcal{R} \implies A \setminus B \in \mathcal{R}$,
\item $\cup$- stabil, d.h. $A, B \in \mathcal{R} \implies A \cup B \in \mathcal{R}$.
\end{enumerate}
\end{mdframed}
\end{lemma}
entspricht für Algebren:
\begin{lemma}
\begin{mdframed}
Eine Familie $\mathcal{A} \subset \mathcal{P}(X)$ ist genau dann eine Algebra auf $X$, wenn
\begin{enumerate}[topsep=3pt, itemsep=0pt]
\item[(i)] $\emptyset \in \mathcal{A}$,
\item[(iii)] $\cup$- stabil,
\item[(iv)] $\complement$- stabil, d.h. $A \in \mathcal{A} \implies \complement A := X\setminus A \in \mathcal{A}$.
\end{enumerate}
\end{mdframed}
\begin{proof}
Sind diese Eigenschaften erfüllt, so implizieren (i + iv), dass
\begin{equation*}
X = \complement \emptyset \in \mathcal{A}
\end{equation*}
``$\setminus$'' kann ausgedrückt werden durch ``$\cup$'' und ``$\complement$'': Aus
\begin{equation*}
\complement (A \setminus B) = (\complement A) \cup B
\end{equation*}
folgt
\begin{equation*}
A \setminus B = \complement \big( (\complement A) \cup B \big)
\end{equation*}
Also ist $\mathcal{A}$ ein Ring, und damit $\mathcal{A}$ eine Algebra. \\
Ist umgekehrt $\mathcal{A}$ eine Algebra, so gelten (i + iii). Da auch $X \in \mathcal{A}$, können wir ``$\complement$'' durch ``$\setminus$'' ausdrücken
\begin{equation*}
\complement A = X \setminus A
\end{equation*}
Also gilt auch (iv).
\end{proof}
\end{lemma}

\begin{folgerung}
Ist $\mathcal{R}$ ein Ring auf $X$ und $A,B \in \mathcal{R}$, so auch $A\setminus B, A\cap B, B\setminus A$ und $A\cup B \in \mathcal{R}$. (\textit{Bem.} Alle in $A\cup B$ enthalten.) Ist $\mathcal{A}$ eine Algebra auf $X$ und $A,B \in \mathcal{A}$, so ist außerdem auch $\complement (A \cup B) \in \mathcal{A}$.
\end{folgerung}

\begin{example} \
\begin{enumerate}
\item[(o)] $\{ \emptyset\} \subset \mathcal{P}(X)$ ist ein Ring auf $X$, \newline
$\{\emptyset, X\} \subset \mathcal{P}(X)$ ist die kleinste Algebra auf $X$, $\mathcal{P}(X) \subset \mathcal{P}(X)$ die größte.
\item [(i)] $\{ \emptyset, A \} \subset \mathcal{P}(X)$ ist ein Ring auf $X$ für ein $A \in \mathcal{P}(X)$, \newline
$\{\emptyset, A, \complement A, X\} \subset \mathcal{P}(X) $ ist eine Algebra auf $X$.
\item[(ii)] Die Familie der endlichen (bzw. abzählbaren) Teilmengen von $X$ ist ein Ring. (eine Algebra, nur falls $X$ selbst endlich bzw. abzählbar) \newline
Die Familie der Teilmengen, die endlich (bzw. abzählbar) sind oder endliches (bzw. abzählbares) Komplement haben, ist eine Algebra.
\end{enumerate}
Weitere Beispiele folgen nach der Diskussion vom Erzeugendensystem.
\end{example}
\textit{Beobachtung.} Der Durchschnitt beliebig vieler Ringe (bzw. Algebren) auf einer festen Menge ist wieder ein Ring (bzw. eine Algebra). Zu jeder Menge $\mathcal{E} \subset \mathcal{P}(X)$ gibt es eine(n) bezüglich mengentheoretischer Inklusion kleinste(n) Ring (bzw. Algebra), der (die) $\mathcal{E}$ umfasst, nämlich den Durchschnitt aller Ringe (bzw. Algebren), die $\mathcal{E}$ umfassen.
\begin{definition}[\textbf{Erzeugendensystem}]
\begin{mdframed}
Der von einer Familie $\mathcal{E} \subset \mathcal{P}(X)$ erzeugte Ring auf $X$ ist der kleinste Ring, der sie enthält. Man nennt $\mathcal{E}$ ein \emph{Erzeugendensystem} dieses Rings, oder \emph{Erzeuger}. (Analog für Algebren)
\end{mdframed}
Die Algebra eines Erzeugendensystems ist oft die einfachste Art, eine(n) Ring bzw. Algebra zu beschreiben. 
\end{definition}
Ein Ring geht aus einem Erzeugendensystem $\mathcal{E} \subset \mathcal{P}(X)$ \emph{konstruktiv} durch einen \emph{abzählbaren} (induktiv!) Prozess hervor, ebenso eine Algebra. \vspace{0.5pc}
\newline
\textbf{Ring.} Definiere induktiv eine Folge von Familien $\mathcal{F}_0 \subset \mathcal{F}_1 \subset ... \subset \mathcal{F}_n \subset ... \subset \mathcal{P}(X)$ mit
\begin{align*}
\mathcal{F}_0 : = &\mathcal{E}\cup \{\emptyset\}  \\
\mathcal{F}_n :=  &\{ A\setminus B, A \cup B \mid A,B \in \mathcal{F}_{n-1} \},\  n \geq 1
\end{align*}
So ist $\bigcup_{n\in \N} \mathcal{F}_n \subset \mathcal{P}(x)$ $\setminus$- und $\cup$-stabil, also ein Ring.  \vspace{0.5pc} \newline 
\textbf{Algebra.} analog.
\subsubsection{Halbringe}
Hat ein Erzeugendensystem strukturelle Eigenschaft, so ist die Beschreibung des erzeugenden Rings einfach. Eine natürliche auftretende Bedingung ist:
\begin{definition}[\textbf{Halbringe}]
\begin{mdframed}
Eine Familie $\mathcal{H}\subset \mathcal{P}(X)$ heißt ein \emph{Halbring} auf $X$, falls
\begin{enumerate}[(\roman*), topsep=3pt, itemsep=0pt]
	\item $\emptyset \in \mathcal{H}$,
	\item $\mathcal{H}$ ist $\cap$- stabil,
	\item Für $A,B \in \mathcal{H}$ existieren \emph{disjunkte} Teilmengen $C_1,...,C_n \in \mathcal{H}$ mit $A\setminus B = C_1 \sqcup ...\sqcup C_n$.
\end{enumerate}
\end{mdframed}
\end{definition}
\begin{remark}
Halbring ist eine Verallgemeinerung des Begriffs Ring, Ringe sind also Halbringe.
\end{remark}
\begin{example} \
\begin{enumerate}
	\item[(o)] $\{ \emptyset\} \subset \mathcal{P}(X)$ ist ein Halbring auf $X$.
	\item[(i)] Die Familie bestehend aus $\emptyset$ und allen (einelementigen) Teilmengen $\{\emptyset\} \cup \{ \{a\} \mid a \in X \}$  ist ein Halbring auf $X$, sie erzeugt den Ring der endlichen Teilmengen von $X$.
\end{enumerate}
Der Grundbaustein für später:
\begin{enumerate}
	\item[(ii)] Die Familie der \emph{halboffenen} Intervalle $[a,b) \subset \R$, falls $a<b$, also $\{[a,b) \mid a,b \in \R, a<b\}$, ist ein Halbring auf $\R$.
\end{enumerate}
\end{example}
Beschreibe den von einem Halbring erzeugenden Ring, die folgende Beobachtung wird darüber hinaus nützlich sein:

\begin{lemma}[Simultane Zerlegung]
\begin{mdframed}
Zu beliebigen Teilmengen $H_1,...,H_m \in \mathcal{H}$ existieren paarweise disjunkte Teilmengen $H_1',...,H_n' \in \mathcal{H}$, sodass jedes $H_i$ sich als die Vereinigung einiger $H_j'$'s darstellen lässt.
 \label{lemmaA}
\end{mdframed}
\begin{proof}
Betrachte die $2^m-1$ Durchschnitte der Form $G_1 \cap ... \cap G_m$, wobei $G_i = H_i$ oder $\complement H_i$ und nicht alle gleich $\complement H_i$. Sie sind paarweise disjunkt und zerlegen $H_1 \cup ... \cup H_m$. Jedes $H_i$ ist die Vereinigung von $2^{m-1}$ von ihnen. Es genügt zu zeigen, dass diese Durchschnitte disjunkte Vereinigungen von Teilmengen aus $\mathcal{H}$ sind. Da Halbringe $\cap$-stabil sind, reicht es zu zeigen, dass die Teilmengen der Form
\begin{equation*}
H \cap \complement \widetilde{H}_l \cap ... \cap \complement \widetilde{H}_1	\quad \text{mit }H,\widetilde{H}_1,...,\widetilde{H}_l \in \mathcal{H}
\end{equation*}
disjunkte Vereinigungen von Teilmengen in $\mathcal{H}$ sind. \\
Da für $H \cap \complement \widetilde{H}_l = H \setminus \widetilde{H}_l$ (Axiom (iii)) gilt, reduziert die Behauptung für $l$ auf Behauptung für $l-1$, mit Induktion liefert dann die Behauptung.
\end{proof}
\end{lemma}

\marginpar{\tiny{21.10.2019}}

\begin{proposition}
\begin{mdframed}
Jede Teilmenge im von einem Halbring $\mathcal{H}$ erzeugten Ring $\mathcal{R}$ ist eine endliche disjunkte Vereinigung von Teilmengen in $\mathcal{H}$, d.h. \begin{scriptsize}(ein einfacher Erzeugungsprozess!)\end{scriptsize}
\begin{equation}
\mathcal{R} := \left\{ \bigsqcup_{k=1}^n A_k \  \Big\rvert \ n\in \N, A_1,...,A_n \in \mathcal{H} \right\}
\label{eqI14}
\end{equation}
\end{mdframed}
\begin{proof}
Sei $\mathcal{R}$ die Familie der endlichen \emph{disjunkten} Vereinigungen von Teilmengen in $\mathcal{H}$. Mit dem letzten \hyperref[lemmaA]{Lemma ``Simultane Zerlegung''} ist $\mathcal{R}$ gleich der Familie aller endlichen Vereinigungen von Teilmengen in $\mathcal{H}$. Sie ist offensichtlich $\cup$-stabil. Zu verifizieren bleibt die $\setminus$- Stabilität. Seien hierzu $A = A_1 \cup ... \cup A_m$ und $B = B_1 \cup ... \cup B_n$, $A_i, B_j \in \mathcal{H}$. Aus dem \hyperref[lemmaA]{Lemma ``Simultane Zerlegung''} folgt, dass es endlich viele nichtleere, \emph{paarweise disjunkte} $H_k' \in \mathcal{H}$ existieren, sodass jedes $A_i$ und $B_j$ eine Vereinigung einiger $H_k'$'s ist. Daraus folgt, dass auch $A$ und $B$ Vereinigungen einiger $H_k'$'s sind. So ist auch $A\setminus B$ die Vereinigung einiger $H_k'$'s, nämlich derer, die in $A$, aber nicht in $B$ enthalten sind. Also ist $\mathcal{R}$ ein Ring, enthalten in von $\mathcal{H}$ erzeugendem Ring\begin{scriptsize}(denn Ringe sind $\cup$-stabil)\end{scriptsize}, also gleich.
\end{proof}
\end{proposition}

\begin{remark}
Man kann den von einer Familie $\mathcal{E}$ erzeugten Halbring nicht (analog zu Ringen und Algebren) definieren, denn das Halbring-Axiom (iii) vererbt nicht auf Durchschnitte von Familien. Es gibt Durchschnitte von Halbringen, die keine Halbringe sind. M.a.W. existieren Familien, die nicht in einem eindeutigen kleinsten Halbring enthalten sind.
\end{remark}

\subsubsection{Produkte von Halbringen und Ringen}
Sind $\mathcal{F}_i \subset \mathcal{P}(X_i), i=1,...,n$ Familien von Teilmengen, so entstehen das Produkt von ``Quadern''
\begin{equation*}
\begin{split}
\mathcal{F}_1 \ast ... \ast \mathcal{F}_n := \{ \underbrace{M_1 \times ... \times M_n}_{\subset X_1 \times ... \times X_n} \mid M_i \in \F_i \text{ für } i=1,...,n\}  \\ \subset  \mathcal{P}(X_1 \times ... \times X_n)
\end{split}
\end{equation*}
und die $\cup$-stabile Hülle, die Familie $\F_1 \boxtimes ... \boxtimes \F_n$ der endlichen Vereinigungen von ``Quadern'' in $\F_1 \ast ... \ast \F_n$, die Figuren, 
\begin{equation*}
\F_1 \boxtimes ... \boxtimes \F_n = \left\{\text{endlcihe Vereinigungen von Teilmengen in } \F_1 \ast ... \ast \F_n	\right\}
\end{equation*}
Beide Produkte $\ast$ und $\boxtimes$ sind \emph{assoziativ}, d.h.\begin{small}
\begin{equation*}
(\F_1 \ast \F_2) \ast \F_3 = \F_1 \ast \F_2 \ast \F_3 = \F_1 \ast (\F_2 \ast \F_3)
\end{equation*}
und
\begin{equation*}
(\F_1 \boxtimes \F_2) \boxtimes \F_3 = \F_1 \boxtimes \F_2 \boxtimes \F_3 = \F_1 \boxtimes (\F_2 \boxtimes \F_3)
\end{equation*}
\end{small}
Wir definieren weiter
\begin{equation*}
\mathcal{Z} = \mathcal{Z} (\F_1,...,\F_n) \subset \F_1 \ast ... \ast \F_n
\end{equation*}
die Familie der Zylindermengen bestehend aus
\begin{equation*}
\begin{split}
\pi_k^{-1}(M_k) = X_1 \times ... \times X_{k-1} \times M_k \times X_{k+1} \times ... \times X_n \quad\\ \text{mit } 1 \leq k \leq n, M_k \in \F_k
\end{split}
\end{equation*}
wobei $\pi_k: X_1 \times ... \times X_n \to X_k, (x_1,...,x_n) \mapsto x_k$ die natürliche Projektion ist. %\newpage
\begin{proposition}
\begin{mdframed}
\begin{enumerate}[(\roman*),topsep=5pt]
	\item Seien $\mathcal{H}_i \subset \mathcal{P}(X_i)$ Halbringe $(i=1,...,n)$ und $\mathcal{R}_i \subset \mathcal{P}(X_i)$ die von ihnen erzeugten Ringe. Dann ist $\mathcal{H}_1 \ast ... \ast \mathcal{H}_n$ ein Halbring auf $X_1 \times ... \times X_n$ und $\mathcal{H}_1 \boxtimes ... \boxtimes \mathcal{H}_n =\mathcal{R}_1 \boxtimes ... \boxtimes \mathcal{R}_n$ der von ihm erzeugte Ring.
	\item Sind $\mathcal{R}_i \subset \mathcal{P}(X_i)$ Ringe und $\mathcal{E}_i \subset \mathcal{R}_i$ Erzeugendensysteme für $i=1,...,n$, so erzeugt die Familie von Zylindermengen $\mathcal{Z}(\mathcal{E}_1,...,\mathcal{E}_n)$ den Produktring.
	%\item Sind $\mathcal{E}_i \subset \mathcal{H}_i$ Erzeugendensysteme der Halbringe, so ist $\mathcal{Z}(\mathcal{E}_1,...,\mathcal{E}_n)$ ein Erzeugendensystem von $\mathcal{H}_1 \ast ... \ast \mathcal{H}_n$ als Halbring sowie (folglich) ein Erzeugendensystem von $\mathcal{H}_1 \boxtimes ... \boxtimes \mathcal{H}_n$ als Ring.
\end{enumerate}
\end{mdframed}
\label{propC}
\begin{proof} \
\begin{enumerate}[(\roman*),topsep=5pt]
	\item Zunächst im Fall $n=2$. Klar enthält $\mathcal{H}_1 \ast \mathcal{H}_2$ auch $\emptyset$ und ist $\cap$-stabil. Wir betrachten die disjunkte Zerlegung
	\begin{equation*}
	\begin{split}
		&(A_1 \times A_2) \setminus (B_1 \times B_2) =\\
	 \big( \underbrace{(A_1 \cap B_1)}_{\in \mathcal{H}_1} \times \underbrace{(A_2 \setminus B_2)}_{\substack{\text{zerlegbar in} \\ \text{Teilmengen}\\ \text{aus }\mathcal{H}_2}} \big) & \sqcup \underbrace{\big( \underbrace{(A_1 \setminus B_1)}_{\substack{\text{zerlegbar in} \\ \text{Teilmengen}\\ \text{aus }\mathcal{H}_1}} \times \underbrace{(A_2 \setminus B_2)}_{\substack{\text{zerlegbar in} \\ \text{Teilmengen}\\ \text{aus }\mathcal{H}_2}} \big)}_{\text{zerlegbar in Teilmengen aus } \mathcal{H}_1 \times \mathcal{H}_2}
		\sqcup \underbrace{\big( (A_1 \setminus B_1) \times (A_2 \cap B_2)\big)}_{\text{analog zerlegbar}}
	\end{split}
	\end{equation*}
	Also ist $(A_1 \times A_2) \setminus (B_1 \times B_2) $ disjunkt zerlegbar in Teilmengen aus $\mathcal{H}_1 \ast \mathcal{H}_2$, also erfüllt Axiom (iii) für Halbringe, d.h.  $\mathcal{H}_1 \ast \mathcal{H}_2$ ist ein Halbring. \\
	Mit Induktion liefert dann die Behauptung auch für $\mathcal{H}_1 \ast ... \ast \mathcal{H}_n, n\geq 1$. \newline
	Aus \eqref{eqI14} folgt, dass $\mathcal{H}_1 \boxtimes ... \boxtimes \mathcal{H}_n$ der von $\mathcal{H}_1 \ast ... \ast \mathcal{H}_n$ erzeugte Ring ist.
	\item Für jedes $i$ wird der Ring auf $X_1 \times ... \times X_n$ bestehend aus den Zylindermengen $\pi_i^{-1}(M_i)$ für $M_i \in \mathcal{R}_i$ von der Familie der Zylindermengen $\pi_i^{-1}(E_i)$ für $E_i \in \mathcal{E}_i$ erzeugt, denn jede Teilmenge $M_i \in \mathcal{R}_i$ kann durch endlich viele Mengenoperationen aus Teilmengen $E_{ij} \in \mathcal{E}_i$ hergestellt werden und $\pi^{-1}_i(M_i)$ entsprechend aus den $\pi_i^{-1}(E_{ij})$. \\
	Daraus folgt, dass $\mathcal{Z}(\mathcal{E}_1,...,\mathcal{E}_n)$ erzeugt denselben Ring wie $\mathcal{Z}( \mathcal{R}_1,...,\mathcal{R}_n)$ und denselben wie $\mathcal{R} \ast ... \ast \mathcal{R}_n$, also den Produktring $\mathcal{R}_1 \boxtimes ... \boxtimes \mathcal{R}_n$.
	%\item Für jedes $k$ gilt: Der Halbring auf $X_1 \times ... \times X_n$ bestehend aus den Zylindermengen $\pi_k^{-1}(H_k)$ für $H_k \in \mathcal{H}_k$ wird erzeugt von den Zylindermengen $\pi_k^{-1}(E_k)$ für $E_k \in \mathcal{E}_k$. Da
	%\begin{equation*}
	%\begin{split}
	%& \mathcal{E}_k \subset \\ 
	%& \underbrace{\{ H_k \in \mathcal{H}_k \mid \pi^{-1}_k (H_k) \text{ gehört zum von den } \pi^{-1}_k(E_k)   \text{ für } E_k \in \mathcal{E}_k \text{ erzeugten Halbring} \}}_{\text{ist Halbring, deshalb Gleicheit}}
	% \\ & \subset \mathcal{H}_k
	%\end{split}
	%\end{equation*}
	%folgt, dass $\mathcal{Z}(\mathcal{E}_1,...,\mathcal{E}_k)$ erzeugen denselben Halbring $\mathcal{Z}(\mathcal{H}_1,...\mathcal{H}_n)$ und damit denselben wie $\mathcal{H}_1 \ast ... \ast \mathcal{H}_n$.
\end{enumerate}
\end{proof}
\end{proposition}
\begin{definition}
\begin{mdframed}
Wir nennen den Halbring $\mathcal{H}_1 \ast ... \ast \mathcal{H}_n$ das Produkt der Halbringe $\mathcal{H}_i$, den Ring $\mathcal{R}_1 \boxtimes ... \boxtimes \mathcal{R}_n$ das Produkt der Ringe $\mathcal{R}_i$.
\end{mdframed}
\end{definition}

\begin{hauptbsp}[Quader und Figuren in $\R^d$]
\begin{mdframed}
Ist $a,b \in \R^d$ mit $a_i < b_i \forall i$, so entsteht achsenparalleler halboffener Quader
\begin{equation*}
[a,b) := [a_1, b_1) \times ... \times [a_d,b_d)
\end{equation*}
Wir bezeichnen
\begin{equation*}
\mathcal{Q}^d := \text{Familie dieser Quader}
\end{equation*}
und
\begin{equation*}
\mathcal{I} := \mathcal{Q}^1, \text{Familie der halboffenen Intervalle}
\end{equation*}
Also gilt
\begin{equation*}
\mathcal{Q}^d = \underbrace{\mathcal{I} \ast ... \ast \mathcal{I}}_{d\text{-Mal}}
\end{equation*}
Aus der \hyperref[propC]{letzten Proposition} folgt, dass $\mathcal{Q}^d$ ein Halbring auf $\mathcal{R}^d$ ist. Es folgt ebenfalls, dass
\begin{equation*}
\mathcal{F}^d := \underbrace{\mathcal{I} \boxtimes ... \boxtimes \mathcal{I}}_{d\text{-Mal}}
\end{equation*}
der Ring erzeugt von $\mathcal{Q}^d$ ist, also der Ring der $d$-dimensionale ``Figuren''.
Wir haben gesehen: Figuren sind disjunkte Vereinigungen von Quadern.
\end{mdframed}
\end{hauptbsp}

Wir arbeiten aus technischen Gründen mit halboffenen Intervallen und Quadern. Besonders übersichtliche sind Halbringe, die abgeschlossen unter Produktbildung sind,  jedoch nicht die Teilmengen enthalten, die uns geometrisch primär interessieren: die offenen und abgeschlossenen Quader, die nicht achsenparallele sind, Polygone und Polytope, gekrümmte ``elementare Geometrie'' sowie Gebilde: Scheiben, Bälle, Zylinder und Kegel. Deshalb müssen wir unsere Ringe weiter anreichern und flexibilisieren, damit sie stabil unter abzählbaren Vereinigungen sind. $\leadsto$ $\sigma$-Algebra.

\subsection{Inhalte und Prämaße}
Wir beginnen mit der Untersuchung von Volumenfunktionen. Die grundlegende Forderung ist die \emph{Additivität}. Volumina dürfen nicht negativ sein, d.h. $\in [0,\infty] := [0, \infty) \cup \{ \infty\}$, also die erweiterte positive Halbgerade. 
\begin{remark}
Die erweiterten reellen Zahlen ist definiert als $\ol{\R} := \{- \infty\} \cup \R \cup \{ \infty \}$ mit natürlichen Konventionen
\begin{align*}
	x + \infty = \infty & \text{ für } x > - \infty	\\
	x \cdot \infty = \infty &\text{ für } x> 0
\end{align*}
Später werden wir außerdem sehen 
\begin{equation*}
	0 \cdot \infty = 0
\end{equation*}
(da $ 0 \cdot n \longrightarrow 0$ für $n \longrightarrow \infty$).
\end{remark}
\subsubsection{Inhalte auf Halbringen und Ringen}
Die allgemeinste Sorte von uns betrachteter Volumenfunktion ist endlich additiv und definiert auf Halbringen.

\begin{definition}[\textbf{Inhalt}]
\begin{mdframed}
Ein Inhalt auf einer Halbring $\mathcal{H}$ ist eine Funktion $\mu: \mathcal{H} \to [0,\infty]$ mit den Eigenschaften:
\begin{enumerate}[(\roman*),topsep=5pt, itemsep = 0 pt]
	\item $\mu (\emptyset) = 0$
	\item \emph{Additivität}: Sind $A_1,...,A_n \in \mathcal{H}$ paarweise disjunkt mit $A_1 \sqcup ... \sqcup A_n \in \mathcal{H}$ \tablefootnote{Die Voraussetzung ist redundant, falls $\mathcal{H}$ ein Ring ist.}, so gilt $\mu(A_1 \sqcup ... \sqcup A_n) = \mu(A_1) + ... +\mu(A_n)$
\end{enumerate}
\end{mdframed}
\end{definition}

\begin{example} \
\begin{enumerate}
	\item[(o)] $\mu \equiv 0$ ``der Nullinhalt'' und\\
				$\nu(A) = \begin{cases}
					0,       & A = \emptyset \\
					\infty, & \text{sonst}
				\end{cases}$
				sind stets Inhalte auf beliebigen Halbringen.
	\item[(i)] Sei $X$ nichtleer. Betrachte die Algebra $\{ \emptyset, X\} \subset \mathcal{P}(X)$. So wird ein Inhalt \\
			$\begin{cases}
				\emptyset \mapsto 0 \\
				x \mapsto v \in [0,\infty] \text{ beliebig}
			\end{cases}$
			definiert.
\end{enumerate}
\end{example}

\begin{example}[Halboffene Intervalle in $\R$, der Grundbaustein für Lebesgue-Maß]
Wir betrachten den Halbring $\I = \mathcal{Q}^1 \subset \mathcal{P}(\R)$. So wird ein Inhalt gegeben durch die euklidische Länge
\begin{align} \label{I23}
\lambda^1_\I : \I \to [0, \infty), \quad \lambda^1_\I \big([a,b)\big) := b-a \ (a<b)
\end{align}
Wir überprüfen die Additivität: Sei $a = x_0 < x_1 < ... < x_{n} = b$ eine Unterteilung von $[a,b)$. So entsteht die disjunkte Zerlegung $[a,b) = [a,x_1) \sqcup ... \sqcup [x_{n-1},b)$. Es folgt 
\begin{equation*}
\underbrace{\lambda^1_\I \big([a,b)]}_{b-a} = \underbrace{\lambda^1_\I \big( [a,x_1)\big)}_{x_1-a} + \underbrace{\lambda^1_\I \big([x_1,x_2)\big)}_{x_2 -x_1} + ... + \underbrace{\lambda^1_\I \big( [x_{n-1},b ) \big)}_{b-x_{n-1}}
\end{equation*}
\end{example}
\marginpar{\tiny{24.10.2019}}
\begin{lemma}[Einfache Eigenschaften von Inhalten]
\begin{mdframed}
Seien $\mathcal{H}$ ein Halbring und $\mu:\mathcal{H} \longrightarrow [0,\infty]$ ein Inhalt. Dann gilt:
\begin{enumerate}[(\roman*),topsep=5pt, itemsep = 0 pt]
	\item \emph{Monotonie}: Ist $A, B \in \mathcal{H}$ mit $A \subset B$, so ist $\mu (A) \leq \mu (B)$.
	\item \emph{Subadditivität:} Seien $A_1, ..., A_n \in \mathcal{H}$ \emph{(nicht notwendigerweise disjunkt!)} mit $A_1 \cup ... \cup A_n \in \mathcal{H}$, dann gilt
	\begin{equation*}
	\mu(A_1 \cup .... \cup A_n) \leq \mu (A_1) + ... + \mu (A_n)
	\end{equation*}
\end{enumerate}
\end{mdframed}
\begin{proof} \
\begin{enumerate}[(\roman*),topsep=5pt, itemsep = 0 pt]
\item Setze $B \setminus A = C_1 \sqcup ... \sqcup C_n$ mit $C_i \in \mathcal{H}$ paarweise disjunkt, bzw. $B= A \sqcup C_1 \sqcup ... \sqcup C_n$. Aus der Additivität folgt dann $\mu(B) = \mu (A) + \underbrace{\mu(C_1) + ... + \mu (C_n)}_{\geq 0} \geq \mu (A)$.
\item Aus dem \hyperref[lemmaA]{Lemma ``simultane Zerlegung''} folgt, dass es paarweise disjunkte $H_i \in \mathcal{H}$ existieren, sodass jedes $A_j$ die Vereinigung einiger von $H_i$ ist. Entsprechend summieren sich die Volumina auf. Die Ungleichung folgt, denn jedes $\mu(H_j)$ \emph{genau einmal} auf der linken Seite und je \emph{mindestens einmal} auf der rechten Seite.
\end{enumerate}
\end{proof}
\end{lemma}

\subsubsection{Fortsetzung von Inhalten von Halbringen auf Ringe}
\begin{satz}
\begin{mdframed}
Jeder Inhalt $\mu$ auf einem Halbring $\mathcal{H}$ besitzt eine eindeutige Fortsetzung zu einem Inhalt $\ol{\mu}$ auf dem von $\mathcal{H}$ erzeugten Ring $\mathcal{R}$.
\end{mdframed}
\begin{proof} \
\begin{enumerate}[-,topsep=5pt, itemsep = 0 pt]
	\item \emph{Eindeutigkeit} folgt aus der Additivität von Inhalten und Beschreibung des erzeugten Rings $\mathcal{R}$. Jede Teilmenge in $\mathcal{R}$ ist eine disjunkte Vereinigung (wegen \eqref{eqI14}) $A_1 \sqcup ... \sqcup A_n$ mit $A_i \in \mathcal{H}$. Daher notwendig
	\begin{equation} \label{eqI25}
	\ol{\mu} (A_1 \sqcup ... \sqcup A_n) = \underbrace{\mu (A_1)}_{= \ol\mu(A_1)} + ... + \ol\mu (A_n)
	\end{equation}
	\item \emph{Existenz} bzw. \emph{Wohldefiniertheit} von $\ol\mu$ durch \eqref{eqI25}: Wir betrachten eine weitere disjunkte Zerlegung derselben Teilmengen
	\begin{equation*}
	A_1 \sqcup ... \sqcup A_n = B_1 \sqcup ... \sqcup B_m \in \mathcal{R}, A_i, B_j \in \mathcal{H}
	\end{equation*}
	so entstehen Zerlegungen 
	\begin{align*}
		A_i = \bigsqcup_{j=1}^m (A_i \cap B_j)	\\
		B_j = \bigsqcup_{i=1}^n (A_i \cap B_j)
	\end{align*}
	Daraus folgt \emph{(Wir bemerken, dass $A_i \cap B_j \in \mathcal{H}$, denn $\mathcal{H}$ $\cap$-stabil ist.)}
	\begin{equation*}
	\sum_i \mu (A_i ) = \sum_i \underbrace{\sum_j \mu (A_i \cap B_j)}_{\mu (A_i)}
	= \sum_j \underbrace{\sum_i \mu (A_i \cap B_j)}_{\mu (B_j)} = \sum_j \mu (B_j)
	\end{equation*}
	Also ist $\ol\mu$ wohldefiniert.
	\item Es bleibt zu zeigen, dass $\ol\mu$ tatsächlich ein Inhalt ist. $\ol\mu$ ist laut der Definition \eqref{eqI25} offensichtlich additiv und somit ein Inhalt.
\end{enumerate}
\end{proof}
\begin{remark}
Hat $\mu$ endliche Werte, so hat $\ol\mu$ auch endliche Werte.
\end{remark}
\end{satz}
\begin{example}
Wir setzen den in dem letzten Beispiel definierten Inhalt \eqref{I23}
\begin{equation*}
\lambda^1_{\mathcal{Q}^1}: \underset{\text{\tiny{Halbring}}}{\mathcal{I} = \mathcal{Q}^1} \longrightarrow [0,\infty)
\end{equation*} auf den Halbring $\mathcal{I}$ fort zu
\begin{equation*}
\lambda^1_{\F^1}:\mathcal{F}^1 \longrightarrow [0,\infty)
\end{equation*}
wobei $\mathcal{F}^1$ den von $\mathcal{Q}^1$ erzeugten Ring der 1-dimensionale Figuren bezeichnen, also ist $\lambda^1_{\mathcal{F}}$ definiert als die Summe der Längen der Teilintervalle.
\end{example}
\begin{remark}
Die Fortsetzung von Inhalten von Ringen auf Algebren ist nicht eindeutig. Z.B. betrachten wir die vom Ring $\mathcal{R}=\{ \emptyset\}$ erzeugte Algebra $\mathcal{A} = \{ \emptyset, X\}$, so können wir den Inhalt von $X$ beliebig $\in [0,\infty)$ wählen.
\end{remark}
\subsubsection{Prämaße}
Wir betrachten Verhalten von Volumina bei gewissen Grenzprozessen. (Approximation von innen und außen) Wir arbeiten mit Teilmengen einer festen Menge $X$.\newline \newline
Falls $(A_n)_{n \in \N}$ eine aufsteigende Folge von Teilmengen von $X$ mit $\bigcup_{n \in \N} A_n =:A$,
\begin{equation*}
A_1 \subset A_2 \subset ... \subset A_n \subset ... \subset X
\end{equation*}  so schreiben wir $A_n \nearrow A$. \newline \newline
Falls $(A'_n)_{n \in \N}$ absteigend mit $\bigcap_{n \in \N} A'_n = A$, 
\begin{equation*}
X \supset A_1' \supset A_2' \supset ... \supset A_n' \supset ...
\end{equation*}
so schreiben wir $A_n' \searrow A$.
\begin{beobachtung}
Es gelten
\begin{align*}
	A_n \nearrow A \iff A \setminus A_n \searrow \emptyset	\\
	A_n' \searrow A \iff A_n'	\setminus A \searrow \emptyset
\end{align*}
Sei $\mu: \mathcal{R} \longrightarrow [0,\infty]$ ein Inhalt. Dann gilt
\begin{equation*}
A_n \nearrow A \searrow A_n' \xRightarrow[]{\text{Monotonie}} \underset{\text{wächst}}{\mu(A_n)} \leq \mu(A) \leq \underset{\text{fällt}}{\mu(A_n')}
\end{equation*}
Da $\mu(A_n)$ und $\mu(A'_n)$ nur \emph{schwach} monoton sind, folgt nur die Ungleichung
\begin{equation}	\label{I33}
\lim\limits_{n\to\infty} \mu(A_n) \leq \mu(A) \leq \lim\limits_{n \to \infty} \mu (A_n')
\end{equation}
\end{beobachtung}
Wir formulieren nun einen disjunkte Zerlegung für $A_n$ durch einen Induktiven Prozess:
\begin{equation*}\begin{split}
	A_0 & := \emptyset	\\
	\widetilde{A}_n &:= A_n \setminus A_{n-1}	 \end{split}
\end{equation*}
so entsteht die disjunkte Zerlegung
\begin{equation*}
	A = \bigsqcup_{n\in \N} \widetilde{A}_n
\end{equation*}
Dann ist \eqref{I33} äquivalent zu: Für Folgen $(\widetilde{A}_n)_{n \in \N}$ paarweise disjunkter Teilmengen mit $A:=\bigsqcup_{n\in\N}\widetilde{A}_n \in \mathcal{R}$ gilt
\begin{equation*}
\mu\bigg( \underbrace{\bigsqcup_{n\in \N} \widetilde{A}_n}_{=A} \bigg) \geq \sum_{n=1}^\infty \mu(\widetilde{A}_n)
\tag{$\sigma$-Supadditivität}
\end{equation*}
Gilt in einer der Gleichung \eqref{I33} die Gleichheit, so fassen wir das als \textbf{Stetigkeitseigenschaften} auf. Wir vergleichen nun die Stetigkeitseigenschaften:

\begin{proposition}
\begin{mdframed}
Für einen Inhalt $\mu : \mathcal{R} \longrightarrow [0,\infty]$ auf einem Ring $\mathcal{R}$ sind die beiden folgenden Eigenschaften äquivalent:
\begin{enumerate}[(\roman*),topsep=5pt, itemsep = 0 pt]
	\item \emph{$\sigma$-Additivität:} Ist $(A_n)_{n \in \N}$ eine Folge paarweise disjunkter Teilmengen, $A_n \in \mathcal{R}$ mit $\bigsqcup_nA_n \in \mathcal{R}$, so gilt
	$$
	\mu \bigg(\bigsqcup_{n=1}^\infty A_n \bigg) = \sum_{n=1}^\infty \mu (A_n)	
	$$
	\item \textit{Stetigkeit von unten:} Ist $(B_n)_{n\in\N}$ aufsteigend, $B_n \in \mathcal{R}$ mit $B_n \nearrow B \in \mathcal{R}$, so gilt
	$$
	\mu (B_n) \nearrow \mu (B)	
	$$
\end{enumerate}
	Sie implizieren die beiden folgenden, ebenfalls zueinander äquivalenten, Eigenschaften:
	\begin{enumerate}[(\roman*),topsep=5pt, itemsep = 0 pt]
	\item[(iii)] \emph{Stetigkeit von oben:} Ist $(C_n)_{n \in \N}$ absteigend, $C_n \in \mathcal{R}$ \textbf{mit} $\mu(C_n) < \infty$ \textbf{und} $C_n \searrow C \in \mathcal{R}$, so gilt
	$$
	\mu(C_n) \searrow \mu (C)	
	$$
	\item[(iv)] \emph{Stetigkeit von $\emptyset$:} Ist $(D_n)_{n \in \N}$ absteigend, $D_n \in \mathcal{R}$ mit $\mu(D_n) < \infty$ und $D_n \searrow \emptyset$, so gilt
	$$ \mu (D_n) \searrow 0$$
	\end{enumerate}
	Falls $\mu$ endliche Werte hat, gilt umgekehrt (iii), (iv) $\implies$ (i), (ii).
\end{mdframed}
\begin{proof}\
\begin{enumerate}[(\roman*),topsep=5pt, itemsep = 0 pt]
	\item[-] (i) $\iff$ (ii). Übergang durch $B_n = A_1 \sqcup ... \sqcup A_n$, bzw. $A_n = B_n \setminus B_{n-1}$, $B = \bigsqcup_n A_n$. Aus der endlichen Additivität folgt $\mu (B_n) = \sum_{i=1}^n \mu(A_i)$. Daraus folgt $\lim\limits_{n \to \infty} \mu (B_n) = \sum_{n \in \N} \mu (A_n)$. Außerdem gilt $\mu(B) = \mu (\bigsqcup_n A_n)$. \checkmark
	\item[-] (ii) $\implies$ (iii). Sei $C_n \searrow C$ mit endlichen Inhalten. Setze $B_n := C_1 \setminus C_n \in \mathcal{R}$ und $B := C_1 \setminus C$, d.h. $C_1 = B_n \sqcup C_n$ und $C_1 = B \sqcup C$. Daraus folgt wegen der Additivität des Inhalts $\mu(C_1) = \mu (B_n) + \mu(C_n)=\mu(B) + \mu (C)$. Es gilt n.V. $\mu(B_n) \nearrow \mu (B)$, da $B_n \nearrow B$. Da alle Inhalte endlich sind, gilt
		$\mu (B_n) = \mu(C_1) - \mu (C_n)$ und
		$\mu (B) = \mu(C_1) - \mu (C)$. Also gilt $\mu(C_1) - \mu (C_n) \nearrow \mu(C_1) - \mu (C)$.
	Daraus folgt, dass $\mu(C_n) \searrow \mu(C)$, also (iii).
	\item[-] (iii) $\Longleftarrow$ (iv).  (Die andere Richtung ist klar, da (iv) ist Spezialfall von (iii)!) \newline
	Sei $C_n \searrow C$ mit endlichen Inhalten. Setze $D_n := \underbrace{C_n \setminus C}_{\in \mathcal{R}} \searrow \emptyset$. Dann ist $C_n = D_n \sqcup C$ und somit $\mu(C_n) = \mu(D_n) + \mu(C)$. Da alle Inhalte endlich sind, gilt $\underbrace{\mu(D_n)}_{\searrow 0 \text{ wegen (iv)}} = \mu(C_n) -\mu (C)$. Daraus folgt $\mu(C_n) \searrow \mu (C)$, also (iii).
	\item[-] $\mu$ habe endliche Werte, es gelte (iv). Zeige (ii). Sei $B_n \nearrow B$. Setze $D_n:=B \setminus B_n$. Da $\underbrace{\mu(D_n)}_{\searrow 0 \text{ wg (iv)}} = \mu(B) - \mu (B_n)$ und alle Inhalte endlich sind, gilt $\mu(B_n) \nearrow \mu(B)$, d.h. (ii).
\end{enumerate}
\end{proof}
\end{proposition}

\marginpar{\tiny{28.10.2019}}

Es ist natürlich und geboten, von Inhalten $\sigma$-Additivität zu verlangen. $\leadsto$ leistungsfähige Volumentheorie. (E. Borel, Lebesgue $\sim$1900)

\begin{definition}
\begin{mdframed}
Ein \textbf{Prämaß} auf einem \emph{Ring} ist $\sigma$-additiver Inhalt.
\end{mdframed}
\end{definition}

\begin{example}
Sei $X$ unendlich. Betrachte $\mathcal{A} \subset \mathcal{P}(X)$ die Algebra erzeugt von endlichen Teilmengen, besteht aus endlichen Teilmengen und deren Komplementen. Definieren Inhalt $\mu$ auf $\mathcal{A}$ durch
$$ \mu (A) := \begin{cases}
	0, & \text{falls } A \text{ endlich} \\
	1, & \text{falls } \complement A \text{ endlich}
	\end{cases}$$
Sind $A_1,...,A_n$ paarweise disjunkt und nicht alle $A_i$ endlich, so gibt es genau ein $A_i$ unendlich. Daraus folgt
$$\underset{\text{genau ein Beitrag }1}{\mu(A_1)+...\mu(A_n)} = 1 = \mu(A_1 \sqcup ... \sqcup A_n).$$
Also war die endliche Additivität. Die $\sigma$-Additivität wird genau dann verletzt, falls $X$ abzählbare Zerlegung in endlichen Teilmengen ($\iff$ $X$ abzählbar) zulässt. \begin{small}Zum Beispiel betrachten wir die natürlichen Zahlen $\N$. Es gilt $\sum_{n\in \N} \mu (\{n\}) =0 \neq 1 = \mu \left( \bigcup_{n \in \N} \{n\} \right)$.
\end{small} Daraus folgt: $\mu $ Prämaß $\iff$ $X$ überabzählbar (z.B. $X = \R$).
\end{example}

\begin{satz}
\begin{mdframed}
Der Inhalt $\lambda^1_{\mathcal{F}^1}$ ist ein Prämaß.
\end{mdframed}
(Später, nach Produkten, Beweis in beliebiger Dimension.)
\begin{proof}
Zu zeigen ist die $\sigma$-Additivität. Dies ist wegen endlicher Werte von $\lambda^1_{\mathcal{F}^1}$ äquivalent zu Stetigkeit in $\emptyset$. Diese weisen wir jetzt nach. Das Argument beruht auf \emph{topologischen} Eigenschaften von $\R$, nämlich Lokalkompaktheit. Wir betrachten eine Folge absteigender Figuren (d.h. \emph{endliche} Vereinigung halboffener Intervalle $[a,b)$), nämlich
$$F_1 \supset F_2 \supset ... \supset ... \supset F_n \supset ...  \quad  \text{mit } F_n \in \mathcal{F}^1.$$
Stetigkeit in $\emptyset$ bedeutet: $\underbrace{F_n \searrow \emptyset}_{\iff \bigcap_n F_n = \emptyset} \implies \lambda^1_{\mathcal{F}^1}(F_n) \searrow 0$. Wir notieren $\lambda = \lambda^1_{\mathcal{F}^1}$.
Annahme: $\lim\limits_{n \to \infty} \lambda(F_n) \geq v_0 >0$. Zu zeigen: $\bigcap_n F_n \neq \emptyset$.\\
Dazu approximieren wir von innen durch eine geschachtelte Folge von Kompakta $K_n$.
Sei $\varepsilon_n \searrow 0$. Dann existieren $F_n' \in \mathcal{F}^1$ und Kompakta $K_n \subset \R$ sodass $F_n' \subset K_n \subset F_n$ \begin{scriptsize} (\textit{Bem.} $K_n$ sind nicht notwendigweise geschachtelt. Wir brauchen $F_n'$, denn $\mu$ ist nicht auf Kompakta definiert.) \end{scriptsize}und $\lambda(F_n') > \lambda(F_n) -\varepsilon_n$. Wir vergleichen absteigende Folgen
$$ \bigcap_{i\leq n} F_i' \subset \bigcap_{i \leq n} K_i \subset \underbrace{\bigcap_{i \leq n} F_i}_{= F_n}.$$
Es genügt zu zeigen, dass
\begin{equation}\label{I5}
\bigcap_{i \leq n} K_i \neq \emptyset \  \ \forall_n,
\end{equation}
denn \eqref{I5} $ \xRightarrow{\text{Topologie}} \bigcap_{i=1}^\infty K_i \neq \emptyset \implies \bigcap_{i=1}^n F_i \neq \emptyset$. \newline
Es gilt
$$
\underbrace{\left( \bigcap_{i \leq n} F_i \right)}_{F_n} \setminus \left( \bigcap_{i \leq n} F_i' \right) = \bigcup_{i \leq n} (\underbrace{F_n}_{\subset F_i} \setminus F_i') \subset \bigcup_{i \leq n}(F_i \setminus F_i'),
$$
also
$$F_n = {\bigcap_{i\leq n} F_i} \subset \left( \bigcap_{i \leq n} F_i' \right) \cup \bigcup_{i \leq n} (F_i \setminus F_i'),$$
d.h. Überdeckung von $F_n$ durch $n+1$ Teilmengen aus $\mathcal{F}^1$. Aus der $\sigma$-Subaddtivität folgt dann
$$
\lambda (F_n) \leq \lambda \left( \bigcap_{i\leq n} F_i' \right) + \sum_{i \leq n} \underbrace{\lambda (F_i \setminus F_i')}_{< \varepsilon_i}
$$
Da $\lambda$ endlich-wertig ist, können wir die Gleichung umstellen zu
$$
\lambda\left( \bigcap_{i \leq n} F_i' \right) \geq \underbrace{\lambda(F_n)}_{\geq v_0} - \sum_{i \leq n} \varepsilon_i > v_0 - \sum_{n \in \N} \varepsilon_n
$$
Wähle $(\varepsilon_n)$ so, dass $\sum_{n\in \N} \varepsilon_n < v_0$. Dann folgt
$$\lambda \left( \bigcap_{i \leq n } F_i' \right) > 0 \implies \bigcap_{i \leq n} F_i' \neq \emptyset \implies \bigcap_{i \leq n }K_i \neq \emptyset $$
Also haben wir \eqref{I5} gezeigt und somit war die Behauptung.
\end{proof}

\begin{definition}
\begin{mdframed}
$\lambda^1_{\mathcal{F}^1}$ heißt das 1-dimensionale \textbf{Lebesgue-Prämaß}.
\end{mdframed}
\end{definition}
\end{satz}

\subsubsection{Produkte von Inhalten und Prämaßen}
Man kann in natürlicher Weise Produkte von Inhalten auf Halbringen und Ringen bilden. Zunächst Halbringe:
\begin{satz}
\begin{mdframed}
Seien $\mu_i$ Inhalte auf Halbringen $\mathcal{H}_i \subset \mathcal{P}(X_i), i=1,...,n$. Dann wird durch
$$
(\mu_1 \times ... \times \mu_n) \big(H_1 \times ... \times H_n \big) := \mu(H_1) \cdot ... \cdot \mu(H_n), H_i \in \mathcal{H}_i
$$
ein Inhalt $\mu_1 \times ... \times \mu_n$ auf dem Produkthalbring $\mathcal{H}_1 \ast ... \ast \mathcal{H}_n \subset \mathcal{P}(X_1 \times ... \times X_n)$ definiert. \\
Zusätzliche \emph{Konvention} für Multiplikation auf $\R$: $0 \cdot \infty = 0$, weil $ 0 \cdot n \xrightarrow{n \to \infty} 0$. 
\end{mdframed}
\begin{proof} (Für $n=2$, der allgemeiner Fall folgt mit Induktion.) \\
Wir müssen die \emph{Additivität} nachweisen. Betrachte die endliche Zerlegung eines ``Rechtecks'' $H_1 \times H_2 \in \mathcal{H}_1 \ast \mathcal{H}_2$ in endlich viele paarweise disjunkte ``Rechtecken'' $H_1^{(j)} \times H_2^{(j)} \in \mathcal{H}_1 \ast \mathcal{H}_2$:
$$
H_1 \cdot H_2 = \bigsqcup_{j} (H_1^{(j)} \times H_2^{(j)}).
$$
Aus dem \hyperref[lemmaA]{Lemma ``simultane Zerlegung''} folgt, dass wir $H_i$ in $H_i'^{(k)} \in \mathcal{H}_i, k \in \underset{\text{endlich!}}{K_i}$ zerlegen können, sodass jedes $H_i^{(j)}$  die disjunkte Vereinigung einiger $H_i'^{(k)}$ ist, also
$$
H_i^{(j)} = \bigsqcup_{k \in K_i^{(j)}} H_i'^{(k)}, K_i^{(j)}\subset K_i.
$$
So entsteht die ``karierte'' Zerlegung
$$
H_1 \times H_2 = \bigsqcup_{(k_1,k_2)\in K_1\times K_2} \Big (H_1'^{(k_1)} \times H_2'^{(k_2)} \Big)	
$$
bzw.
$$
H_1^{j} \times H_2^{j} = \bigsqcup_{(k_1,k_2)\in K_1^{(j)}\times K_2^{(j)}} \Big (H_1'^{(k_1)} \times H_2'^{(k_2)} \Big).
$$
Es gilt die Zerlegung der Indexmenge
$$
K_1 \times K_2 = \bigsqcup_j \Big( K_1^{(j)} \times K_2^{(j)} \Big).
$$
Da $\mu_i$ additiv ist, folgt
$$
\mu_i (H_i) = \sum_{k \in K_i} \mu_i (H_i'^{(k)})
$$
bzw. 
$$
\mu_i (H_i^{(j)}) = \sum_{k \in K_i^{(j)}} \mu_i (H_i'^{(k)}).
$$
Es folgt \begin{scriptsize}
(Die Additivität von $\mu_1 \times \mu_2$ ist klar für ``karierte'' Zerlegung)
\end{scriptsize}
\begin{equation*}
	\begin{split}
	(\mu_1 \times \mu_2) \big( H_1 \times H_2 \big) & = \sum_{(k_1,k_2) \in K_1 \times K_2} (\mu_1 \times \mu_2) \big(H_1'^{(k_1)} \times H_2'^{(k_2)} \big) \\
	& = \sum_j \underbrace{ \sum_{(k_1,k_2) \in K_1^{(j)} \times K_2^{(j)}} (\mu_1 \times \mu_2) \big(H_1'^{(k_1)} \times H_2'^{(k_2)} \big)}_{= \mu_1 \times \mu_2 (H_1^{(j)} \times H_2 ^{(j)})}.
	\end{split}
\end{equation*}
Also war die Additivität von $\mu_1 \times \mu_2$.
\end{proof}
\end{satz}

\begin{definition}
\begin{mdframed}
Wir nennen den Inhalt $\mu_1 \times ... \times \mu_n$ auf dem Halbring $\mathcal{H}_1 \ast ... \ast \mathcal{H}_n$ das Produkte der Inhalte $\mu_i$.
\end{mdframed}
\end{definition}

\begin{hauptbsp}[\textbf{Elementarinhalt}]
Der Elementarinhalt achsenparalleler Quader $[a,b) \subset \R^d$ für $a,b \in \R^d$ mit $a_i < b_i$ ist definiert als ihr $d$-dimensionales euklidisches Volumen
$$
\lambda^d_{\mathcal{Q}^d}\big([a,b)\big) := (b_1 -a_1) \cdot ... \cdot (b_d-a_d)
$$
Der Elementarinhalt $\lambda^d_{\mathcal{Q}^d}$ ist ein Inhalt, denn
$$
\lambda^d_{\mathcal{Q}^d} = \underbrace{\lambda^1_\mathcal{I} \times ... \times \lambda^1_\mathcal{I}}_{d\text{-mal}}
$$
bezüglich der Zerlegung $\mathcal{Q}^d = \mathcal{I} \times ... \times \mathcal{I}$.
\end{hauptbsp}

Für Ringe folgt mit dem Fortsetzungsresultat für Inhalte von Halbringen auf (die von den erzeugte) Ringe:

\begin{korollar}
Seien $\mu_i$ Inhalte auf Ringen $\mathcal{R}_i \subset \mathcal{P}(X_i)$, $(i=1,...,n)$. Dann existiert einen eindeutigen Inhalt $\mu_1 \boxtimes .... \boxtimes \mu_n$ auf dem Produktring $\mathcal{R}_1 \boxtimes ... \boxtimes \mathcal{R}_n \subset \mathcal{P}(X_1 \times ... \times X_n)$ mit
$$(\mu_1 \boxtimes ... \boxtimes \mu_n) \big( R_1 \times ... R_n\big) = \mu_1 (R_1) \cdot ... \mu_n(R_n), R_i \in \mathcal{R}_i$$
\end{korollar}

\begin{hauptbsp}
Der Elementarinhalt $\lambda^d_{\mathcal{Q}^d}: \mathcal{Q}^d \longrightarrow [0,\infty)$ setzt sich eindeutig fort zu einem Inhalt
$$
\lambda^d_{\mathcal{F}^d}: \mathcal{F}^d \longrightarrow [0,\infty)
$$
auf $d$-dimensionalen Figuren, den wir als \emph{$d$-dimensionales euklidisches Volumen} auffassen.
\end{hauptbsp}

Auch Prämaße verhalten sich gut unter Produkten:

\begin{satz}
\begin{mdframed}
Endliche Produkte von Prämaßen sind wieder Prämaße.
\end{mdframed}
\begin{proof}
($n=2$, der allgemeiner Fall folgt mit Induktion.)   \newline
Seien $\mu_i$ Prämaße auf Ringen $R_i \subset \mathcal{P}(X_i)$, $i=1,2$. Zu zeigen: Der Produktinhalt $\mu_1 \boxtimes \mu_2$ auf $\mathcal{R}_1 \boxtimes \mathcal{R}_2$ ist $\sigma$-additiv. Weil Figuren endliche disjunkte Vereinigungen von Quadern sind und der Inhalt $\mu_1 \boxtimes \mu_2$ (endlich) additiv ist, genügt es, abzählbare disjunkte Zerlegungen
$$
A_1 \times A_2 = \bigsqcup_m (A_{1,m} \times A_{2,m})
$$
mit $A_i, A_{i,m} \in \mathcal{R}_i$ zu betrachten. Zu zeigen ist also
$$
\sum_m \mu_1(A_{1,m}) \cdot \mu_2(A_{2,m}) = \mu_1 (A_1) \cdot \mu_2(A_2)
$$
Wegen der Monotonie gilt die Richtung ``$\leq$'', zu zeigen ist ``$\geq$''.
\marginpar{\tiny{31.10.2019}}
\end{proof}
\end{satz}


%\appendix
\newpage
\subsubsection*{Wiederholung: Integrale stetiger Funktionen (einer reellen Variable)} %\small (einer reellen Variable)}
\marginpar{\tiny{23.10.2019}}
Wir unterscheiden zwischen
\begin{enumerate}[- ,topsep =-3pt]
	\item dem \emph{bestimmten Integral}
				$$ \int\limits_a^b f(x)\D x$$
	\item und dem \emph{unbestimmten Integral}, d.h. die Menge der Funktionen dieser Art
	$$x \mapsto \int\limits_a^x f(\xi) \D \xi + \underset{\text{\tiny const}}C$$
	Notation: $\int f \D x$.
\end{enumerate}

\begin{satz}[\textbf{Hauptsatz der Differential- und Integralrechnung}]  \begin{mdframed} \
\begin{enumerate}[(\roman*), topsep = -1 pt]
	\item Ist $f$ von der Klasse $\mathcal{C}^0$ (d.h. stetig), so ist 
		$$\left(\int f \D x\right)' = f$$
		d.h. die Repräsentanten des unbestimmten Integrals sind Stammfunktionen.
	\item Ist $f$ von der Klasse $\mathcal{C}^1$ (d.h. stetig differenzierbar), so ist
	$$ \int F' \D x =F$$
	(zu lesen: $F$ repräsentiert $\int F' \D x)$ bzw.
	$$ \int\limits_a^x F'(\xi) \D x = F(x) -F(a)$$
\end{enumerate}
\end{mdframed}
\end{satz}

\paragraph{Rechenregeln für Differentialrechnung $\leadsto$ Rechenregeln für Integralrechnung}z.B.
\begin{equation*}
	\begin{split}
		(\ln |x|)' = \frac{1}{x} & \implies \int \frac{\D x}{x} = \ln |x| \text{ auf } \R\setminus\{0\}\\
		\arcsin' x {=\frac{1}{\sqrt{1-x^2}}} & \implies \int \frac{\D x}{\sqrt{1-x^2}} = \arcsin x \text{ auf } (-1,1) \\ 		
		\arctan' x = \frac{1}{1+x^2} & \implies \int \frac{\D x}{1+x^2} = \arctan x \text{ auf }\R
	\end{split}
\end{equation*}
\paragraph{Kettenregel $\leadsto$ Substitutionsregel} Aus der Kettenregel
$$(F \circ \varphi)' (u) = F'(\varphi (u)) \cdot \varphi'(u)$$
folgt mit dem Hauptsatz der Differential- und Integralrechnung: Sei $f := F'$ ($\mathcal{C}^0$),
$$ \int\limits_a^b f(\varphi(u)) \cdot \varphi'(u) \D u = \int\limits_a^b (F \circ \varphi) (u) \D u= F \circ \varphi \big\vert_a^b = F\big\vert^{\varphi(b)}_{\varphi(a)} = \int\limits_{\varphi(a)}^{\varphi(b)} f \D x
$$
Also die \textbf{Substitutionsregel} (Bezeichne $I:= (a,b), J = (\phi(a),\phi(b))$)
$$
\boxed{
\int\limits_a^b f(\varphi(u)) \varphi'(u) \D u = \int_{\phi(a)}^{\phi(b)} f(x) \D x 
}
$$
und die \textbf{Version für unbestimmtes Integral}
$$ \int f(\phi(u)) \phi'(u) \D u = \underbrace{\int f(x) \D x \Big \vert_{x =\phi(u)}}_{\substack{\text{die Komposition }\phi \\ \text{ mit } \int f(x)\D x}} $$
\begin{example} \
\begin{enumerate}
 \item \textbf{Lineare Substitution} mit $x = u+\alpha, \alpha \in \R$ 
 $$
	\int_a^b f(u+\alpha) \D x = \int_{a + \alpha}^{b+ \alpha} f(x) \D x
 $$
 bzw. 
 $$
 	\int f( u +\alpha) \D u = \int f(x) \D x \Big\vert_{x = u + \alpha}
 $$
 z.B. $f(x) = \frac{1}{x}$ auf $\R \setminus \{0\}$
 $$ \int_a^b \frac{\D u}{u+\alpha} = \int_{a+\alpha}^{b+\alpha} \frac{\D x}{x} = \ln |x| \big\vert^{b+\alpha}_{a+\alpha} = \ln \left| \frac{b+\alpha}{a+\alpha} \right|$$
 bzw.
 $$ \int \frac{\D u}{u+\alpha} = \int \frac{\D x}{x} \Big\vert_{x= u+\alpha}= \ln |x| \big\vert_{x = u + \alpha} = \ln |u + \alpha|$$
 \item[(i')] \textbf{(Multiplikative) lineare Substitution} mit $x= \lambda u (\lambda \in \R \setminus \{0\})$
 	$$ \int_a^b f(\lambda u) \D u = \frac{1}{\lambda} \int_{\lambda a}^{\lambda b} f(x) \D x$$ bzw. 
 	$$ \int f(\lambda u) \D u = \frac{1}{\lambda} \int f(x) \D x \Big\vert_{x = \lambda u}$$
 	z.B.
 $$\int \cos \lambda u \D u = \frac{1}{\lambda} \int \underbrace{\cos x}_{\sin' x} \D x \Big\vert_{x= \lambda u} = \frac{1}{\lambda} \sin \lambda u$$
 \item \textbf{Quadratische Substitution} mit $x = u^2$
 $$ \int_a^b f(u^2) u \D u =\frac{1}{2} \int_{a^2}^{b^2} f(x) \D x$$
 bzw.
 $$ f(u^2) u \D u = \frac{1}{2} f(x) \D x \Big\vert_{x=u^2}$$
 z.B. $f(x) = e^x$:
 $$ \int u  e^{u^2} \D u = \frac{1}{2} \int e^x \D x \Big\vert_{x=u^2} = \frac{1}{2} e^{u^2}$$
 \item Mit $f(x) = \frac{1}{x}$, (falls $\phi|_J$ keine Nullstelle hat)
 $$ \int_a^b \frac{\phi'(u)}{\phi(u)} \D u = \int_{\phi(a)}^{\phi(b)}\frac{\D x}{x} = \ln |x| \big\vert^{\phi(b)}_{\phi(a)} = \ln |\phi(u)| \big\vert_{a}^b$$
 bzw.
 $$ \int \frac{\phi'(u)}{ \phi(u)} \D u = \int \frac{\D x}{x} \Big\vert_{x=\phi(u)} = \ln | \phi(u) |$$
 z.B. $\phi(u)= \cos u$ auf $(-\frac{\pi}{2},\frac{\pi}{2})$
 $$\int \tan u \D u  = \int -\frac{\cos' u}{\cos u} \D u= -\ln |\cos u|$$
 Berechne $\int \frac{\D x}{\sqrt{1+x^2}}$ auf $\R (=I =J)$. Substituiere $x=\sinh u$ mit der Umkehrfunktion $u = \operatorname{arsinh} x$.
 $$\int \frac{\D x}{\sqrt{1+x^2}} = \int \frac{\sinh' u}{\sqrt{1+\sinh^2 u}} \D u \Bigg\vert_{u = \operatorname{arsinh}x}= \int \frac{\cosh' u}{\cosh' u} \D u\Big\vert_{u = \operatorname{arsinh}x} = \operatorname{arsinh}x$$
\end{enumerate}
\rule{\textwidth}{0.4pt}
\marginpar{\tiny{30.10.2019}}
Die \emph{Produktregel} für die Ableitung führt zur Methode der \emph{partiellen} Integration.
\paragraph{Partielle Integration:} Für $\mathcal{C}^1$ Funktionen $f,g:I\to \C$ auf einem offenen Intervall $I \subset \R$ gilt:
\begin{equation*}
	\boxed{
	\int_a^b f' \cdot g \D x = f\cdot g \big\vert^b_a - \int^b_a f \cdot g' \D x	
	}
\end{equation*}
für $a,b \in I$ Mann nennt $f\cdot g \vert^b_a$ \emph{Randterm}. \\\\
Für unbestimmte Integrale schreibt man
\begin{equation*}
	\boxed{
	\int f' g\D x = f \cdot g \big\vert -  \int f \cdot g' \D x
	}
\end{equation*}
Man kann diese Gleichung lesen als eine Gleichheit von Funktionenmengen oder so, dass jeder Repräsentant der rechten Seite $f \cdot g \big\vert -  \int f \cdot g'$ ein Repräsentant der linken Seite $\int f' g$ ist.
\begin{proof}
Nach der Produktregel ist $f \cdot g$ Stammfunktion von $f'g+fg'$. Der Hauptsatz der Differential- und Integralrechnung liefert dann 
$$ \int (f' g + fg') = fg \big\vert$$
\end{proof}

\begin{example}
\begin{enumerate}
	\item Berechnung von $\int \ln x \D x$ auf $(0,\infty)$. Dort gilt wegen $\ln' x = \frac{1}{x}$
	\begin{equation*}
		\int \ln x \D x = \int (x)'\ln x \D x = x\ln x \big\vert - \int x \cdot \frac{1}{x} \D x = x (\ln x -1)\big\vert
	\end{equation*}
	\item Berechnung von $\int x e^x \D x$ auf $\R$.
	\begin{equation*}
		\int x e^x \D x = \int x (e^x)' \D x = x e^x\big\vert - \int \underbrace{(x)'}_{=1} e^x \D x =e^x (x-1)
	\end{equation*}
	\item[(ii)'] Berechnung von $\int x^n e^x \D x$ auf $\R$.
	\begin{equation*}
		I_n(x) := \int x^n e^x \D x  = \int x^n (e^x)' \D x = x^ne^x - n \underbrace{\int x^{n-1} e^x \D x}_{I_{n-1}(x)}
	\end{equation*}
	Wir erhalten die Rekursionsformel 
	$$I_n (x) = x^n e^x \big\vert - n I_{n-1}(x)$$
	\item Berechnung von $\int\sqrt{1-x^2}\D x $ auf $(-1,1)$.
	\begin{equation*}
	\begin{split}
		\int \sqrt{1-x^2} \D x= \int (x') \sqrt{1-x^2} \D x & = x \sqrt{1-x^2} \big\vert + \int x  \frac{-x}{\sqrt{1-x^2}} \D x \\
		& = x \sqrt{1-x^2}\big\vert + \int \underbrace{\frac{1-x^2}{\sqrt{1-x^2}}}_{\sqrt{1-x^2}}\D x + \underbrace{\int \frac{1}{\sqrt{1-x^2}} \D x}_{\arcsin x} \\
		& = \frac{1}{2}\cdot \Big( x \sqrt{1-x^2} + \arcsin (x) \Big) \Big\vert
	\end{split}
	\end{equation*}
	\begin{remark}
	Die Regel für die Berechnung der Ableitung von Umkehrfunktion ist
	$$ (f^{-1})' = \frac{1}{f'(f^{-1})} $$
	\end{remark}
	und somit haben wir die Ableitung von $\arcsin$:
	$$\arcsin' (x) = \frac{1}{\cos (\arcsin (x))} = \frac{1}{\sqrt{1-\sin^2\big(\arcsin(x)\big)}} = \frac{1}{1-x^2}$$
	Insbesondere erhalten wir durch Grenzübergang \begin{small} (Hier ist der Grenzübergang nötig, da $\sqrt{1-x^2}$ nicht stetig differenzierbar in Punkten $-1$ und $1$ sind) \end{small} für das abgeschlossene Intervall $[-1,1]$:
	\begin{equation*}
		\begin{split}
		\int_{-1}^1 \sqrt{1-x^2} \D x = \lim\limits_{\varepsilon \to 0} \int_{-1+\varepsilon}^{1-\varepsilon} \sqrt{1-x^2} \D x & = \lim\limits_{\varepsilon \to 0} \frac{1}{2} \big(x \sqrt{1-x^2} + \arcsin x\big) \Big\vert_{-1+\varepsilon}^{1-\varepsilon} \\
		& = \frac{1}{2} \left( \frac{\pi}{2} - \left( - \frac{\pi}{2} \right) \right) \\ 
		& = \frac{\pi }{2}
		\end{split}
	\end{equation*}
	Dies zeigt insbesondere, dass die Fläche der Einheitsscheibe $\pi$ ist.  \\ Außerdem können wir das Integral auch mit Substitution berechnen: 
	\begin{equation*}
	\begin{split}
	 \int_{-1}^1 \sqrt{1-x^2} \D x & \overset{x= \sin u}= \int^{\pi/2}_{-\pi/2} \sqrt{1-\sin^2(u)} \cos u \D u  = \int_{-\pi/2}^{\pi/2} \underbrace{\cos^2 u}_{ \frac{1+\cos 2u}{2}} \D u \\
		& = 	\Big(\frac{u}{2}+\frac{1}{4}\sin 2u\Big) \Big\vert_{-\pi/2}^{\pi/2} = \frac{\pi}{2}
	 \end{split}
	\end{equation*}
	\item Berechnung von $\int \arctan$: Wir bemerken, dass $\arctan ' (x)=\frac{1}{1+x^2}$, denn $\tan'(x)=1+\tan^2x $.
	\begin{equation*}
	\begin{split}
	  \int \arctan(x) \D x &= \int (x')\arctan(x) \D x = x \cdot \arctan x \big\vert - \int \frac{1}{2} \frac{2x}{1+x^2} \D x \\
	  &  \overset{t = x^2}= x \cdot \arctan x - \int \frac{1}{2} \cdot \frac{1}{1+t} \D t \Big\vert_{t=x^2} \\
	  & = x \cdot \arctan x - \frac{1}{2} \ln (1+x^2)
	  	\end{split}
	\end{equation*}
	\item Berechnung von $\int \arcsin(x)$ auf $(-1,1)$
	\begin{equation*}
	\begin{split}
		\int \arcsin(x) \D x = \int (x)' \arcsin(x) \D x & = x\arcsin(x) \big\vert - \int \frac{1}{2} \frac{2 x \D x}{\sqrt{1-x^2}} \\
		& \overset{t:= x^2}= x\arcsin (x) \big\vert - \int \frac{1}{2} \frac{\D t}{\sqrt{1-t}}  \Big\vert_{t=x^2}	 \\
		& = x\arcsin x + \sqrt{1-t}\big\vert_{t=x^2} \\
		& = x\arcsin x + \sqrt{1-x^2} \big\vert
	\end{split}
	\end{equation*}
	\item Berechnung von $\int \sin^2x \D x$. Da
	\begin{equation*}
		\int \sin^2 x \D x = \int (-\cos(x))' \sin x \D x = -\cos x \sin x + \int \underbrace{\cos^2 x}_{1-\sin^2 x} \D x
	\end{equation*}
	erhalten wir
	$$
	\int \sin^2 x \D x = \frac{1}{2} (-\cos x \sin x + x )	
	$$
	\item[(vi)'] Berechnung von $\int \sin^n(x) \D x$ für $n \in \N$.
	\begin{equation*}
		\begin{split}
		I_n (x) := \int \sin^n (x) \D x & = \int (-\cos(x))' \sin^{n-1}(x)\D x \\
				& = -\cos x \sin^{n-1} x + \int 	\underbrace{\cos^2 (x)}_{1-\sin^2(x)} (n-1) \sin^{n-2} (x) \D x \\
				&  = -\cos x \sin^{n-1} x +(n-1) \left( \underbrace{\int \sin^{n-2} (x) \D x}_{I_{n-2}(x)} - \underbrace{\int \sin^n(x)\D x}_{I_n(x)} \right)
		\end{split}
	\end{equation*}
	Wir erhalten:
	$$n \cdot I_n(x) = -\cos (x)\sin^{(n-1)}(x) + (n-1) I_{n-2}(x)$$
	Zum Beispiel gilt $I_0(x)=x$, $I_1(x)=-\cos x $, $I_2(x)=\frac{1}{2}(x-\sin x\cos x)$, $I_3(x)=\frac{1}{3} \cos^3 x - \cos x$.
\end{enumerate}
\end{example}
\newline
\rule{\textwidth}{0.4pt}

\paragraph{Rationale Funktionen:} $\int \frac{p(x)}{q(x)} \D x$ mit $p,q$ Polynome, $q \neq 0$.Zunächst $p,q \in \C[x]$:
\begin{satz}[\textbf{Fundamentalsatz der Algebra}] \begin{mdframed} Jedes Polynom $p \in \C[x]$ vom Grad $n= \deg (p)$ besitzt eine Zerlegung in Linearfaktoren:
$$ p(x) = c \cdot \prod_{n=1}^m (x-\alpha_k)^{n_k}$$
mit $m \leq n, c \in \C \setminus \{0\}, \alpha_1,...,\alpha_m \in \C, \sum_{k=1}^m n_k = n$. Hierbei sind $\alpha_1,...,\alpha_m$ die verschiedenen Nullstellen von $p$ und $n_k$ die Vielfachheit von $\alpha_k$. Die Linearfaktorzerlegung ist bis auf Vertauschung von Faktoren eindeutig.
\end{mdframed}
Polynomdivision führt zu:
$$\frac{p(x)}{q(x)} = s(x) + \frac{r(x)}{q(x)} \quad \text{mit } s\in \C [x] \text{ und } \deg r < \deg q$$
\end{satz}

\paragraph{Partialbruchzerlegung:} Sei $q(x) \in \C [x]$ mit $q(x) = c \cdot \prod_{n=1}^m (x-\alpha_k)^{n_k} $. \linebreak Für jedes komplexe Polynom $r(x) \in \C[x]$ mit $\deg r < \deg q =n $ existiert eine eindeutige Zerlegung:
$$
\frac{r(x)}{q(x)} = \sum_{k=1}^m \sum_{j=1}^{n_k} \frac{c_{kj}}{(x-x_k)^j}, \quad c_{jk} \in \C
$$
d.h.
$$r(x) = \sum_{k=1}^m \sum_{j=1}^{n_k} b_{kj}(x) c_{kj}$$
mit 
$$b_{kj} = (x-\alpha_k)^{n_k-j} \cdot \prod_{\substack{l=1 \\ l \neq k}}^m (x-\alpha_l)^{n_l}$$
Anderes gesagt, $\{b_{kj}\}$ für $k=1,...,m$ und $j=1,...,n_k$ bilden Basis von $\C_{\deg<n}[x]$. Insbesondere $\dim( \C_{\deg <n}[x])=n$.

\begin{example}
$$\int \frac{1}{1-x^2} \D x = \frac{1}{2} \int \frac{1}{1-x} \D x + \frac{1}{2} \int \frac{1}{1+x} \D x = \frac{1}{2} \ln \left| \frac{1+x}{1-x} \right|$$
\end{example}
\end{example}
\end{document}
