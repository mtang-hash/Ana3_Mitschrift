\documentclass[12pt,a4paper]{article}
\usepackage[T1]{fontenc}
\usepackage[utf8]{inputenc}
\usepackage{lmodern}
\usepackage[left=28mm,top=28mm,right=28mm,bottom=28mm] {geometry}
\usepackage{amsfonts}
\usepackage{mathrsfs}
\usepackage[intlimits]{amsmath}
\usepackage{stmaryrd}
\usepackage{relsize}
\usepackage{etoolbox}
\usepackage[ngerman]{babel}
\usepackage[utf8]{inputenc}
\usepackage[T1]{fontenc}
\usepackage{marvosym}
\usepackage[shortlabels]{enumitem}
\usepackage{mathtools}
\usepackage{amssymb}
\usepackage{cancel}
\usepackage{mdframed}
\usepackage{framed}
\usepackage{mathtools}
\usepackage{tablefootnote} 
\usepackage{listings}
\usepackage{amsthm}
\usepackage{xcolor}
\usepackage{etoolbox}
\usepackage[all]{xy}
\usepackage{tikz}
\usepackage{thmtools}
\usepackage[
   pdfpagelabels=true,
   pdftitle={Analysis III: Maßtheorie und Integralrechnung mehrerer Variablen},
   pdfauthor={MT},
 ]{hyperref}
\usepackage{bookmark}
\usetikzlibrary{cd}
\usetikzlibrary{calc}
\theoremstyle{definition}
\newtheorem*{definition}{Definition}
\newtheorem*{satz}{Satz}
\newtheorem*{lemma}{Lemma}
\newtheorem*{proposition}{Proposition}
\newtheorem*{korollar}{Korollar}
\newtheorem*{folgerung}{Folgerung}
\newtheorem*{example}{Beispiel}
\newtheorem*{hauptbsp}{Hauptbeispiel}
\theoremstyle{remark}
\newtheorem*{remark}{Bemerkung}
\newtheorem*{remark'}{Nebenbemerkung}
\newtheorem*{beobachtung}{Beobachtung}
\AfterEndEnvironment{lemma}{\noindent\ignorespaces}
\AfterEndEnvironment{definition}{\noindent\ignorespaces}
\AfterEndEnvironment{example}{\noindent\ignorespaces}
\AfterEndEnvironment{theorem}{\noindent\ignorespaces}
\AfterEndEnvironment{satz}{\noindent\ignorespaces}
\AfterEndEnvironment{korollar}{\noindent\ignorespaces}
\AfterEndEnvironment{remark}{\noindent\ignorespaces}
\AfterEndEnvironment{remark'}{\noindent\ignorespaces}
\AfterEndEnvironment{proposition}{\noindent\ignorespaces}
\AfterEndEnvironment{proof}{\noindent\ignorespaces}
\let\existstemp\exists
\let\foralltemp\forall
\newcommand{\tikzmark}[1]{\tikz[overlay,remember picture] \node (#1) {};}
\newcommand{\vsubset}{\rotatebox[origin=c]{90}{$\subset$}}
\newcommand{\vphi}{\phi}
\newcommand{\ol}{\overline}
%Differentiation
\newcommand{\D}{\, \mathrm{d}}
%Bold Symbols
\newcommand{\R}{\mathbb{R}}
\newcommand{\C}{\mathbb{C}}
\newcommand{\N}{\mathbb{N}}
\newcommand{\Q}{\mathbb{Q}}
%Calligraphic Symbols
\newcommand{\II}{\mathcal{I}}
\newcommand{\FF}{\mathcal{F}}
\newcommand{\QQ}{\mathcal{Q}}
\newcommand{\EE}{\mathcal{E}}
\newcommand{\PP}{\mathcal{P}}
\newcommand{\TT}{\mathcal{T}}
\newcommand{\HH}{\mathscr{H}}
\newcommand{\RR}{\mathscr{R}}
\newcommand{\BB}{\mathscr{B}}
\newcommand{\CC}{\mathscr{C}}
\newcommand{\DD}{\mathscr{D}}
\renewcommand{\AA}{\mathscr{A}}
% Operator names
\newcommand{\Hom}{\operatorname{Hom}}
\newcommand{\del}{\partial}
\newcommand{\vol}{\operatorname{vol}}
\newcommand{\Var}{\operatorname{Var}} 
\newcommand{\Cov}{\operatorname{Cov}}
\newcommand{\End}{\operatorname{End}}
\newcommand{\SL}{\operatorname{SL}}
\newcommand{\Bild}{\begin{tiny}(Bild hier)\end{tiny}}
\renewcommand*{\exists}{\existstemp\mkern2mu}
\renewcommand*{\forall}{\foralltemp\mkern2mu}
\renewcommand{\emptyset}{\varnothing}
\renewcommand{\Re}{\operatorname{Re}}
\renewcommand{\Im}{\operatorname{Im}}
\renewcommand{\qedsymbol}{$\blacksquare$}
\renewcommand{\phi}{\varphi}
\renewcommand{\thesection}{\Roman{section}}
\makeatletter 
\AfterEndEnvironment{mdframed}{%
 \tfn@tablefootnoteprintout% 
 \gdef\tfn@fnt{0}% 
}
\numberwithin{equation}{section}

\setlist[enumerate,1]{label={(\roman*)}}

\title{Analysis III: Maßtheorie und Integralrechnung mehrerer Variablen}
%\date{}
\begin{document}
%\maketitle
%\tableofcontents
\addtocontents{toc}{\protect\setcounter{tocdepth}{3}}
\newpage
\setcounter{page}{2}
\section{Maßtheorie}
\subsection{Maßproblem und Paradoxien}
\marginpar{\tiny{14.10.2019}}
Maßtheorie ist die Theorie des Volumens. Motivierende Beispiele sind:
\begin{enumerate}[label=\roman*),topsep=3pt, itemsep=0pt]
\item Volumina von Teilmengen des euklidischen Raums
\item Wahrscheinlichkeiten (= ``Volumina von Ereignissen'')
\end{enumerate}
Wir konzentrieren uns im Rest des Abschnitts auf $\R^d$. Wir wollen leistungsfähigen Volumenbegriff haben, sodass die Volumina von möglich vielen Teilmengen flexibel gemessen werden können. Unser erster ``naiver'' Ansatz wäre, dass wir Volumenmessung für \emph{alle} Teilmengen verlangen, also eine Funktion
\begin{equation}
 \vol: \mathcal{P}(\R^d) \longrightarrow [0.\infty]
\end{equation}
Unsere grundlegende Forderung ist die Additivität von Volumina bei Zerlegungen, also
\begin{subequations}
\begin{enumerate}
\item[(i)] \textbf{(endliche) Additivität}: Sind $M_1,...,M_n\subset \R^d$ paarweise disjunkt, so gilt
\begin{equation}
\vol (M_1\cup ... \cup M_n) = \vol(M_1)+...+\vol (M_n)
\end{equation}
\end{enumerate}
Volumina als geometrische Größen sollten durch die metrische Struktur (Längenmessung) bestimmt sein, also invariant unter Symmetrien der metrischen Struktur:
\begin{enumerate}
\item[(ii)] \textbf{Bewegungsinvarianz}: Für jede Bewegung $\phi: \R^d \longrightarrow \R^d$ und jede Teilmenge $A\subset \R^d$ gilt
\begin{equation}
\vol(\phi(A)) = \vol (A)
\end{equation}
\item[(iii)] \textbf{Normierung}: $\vol([0,1]^d)=1$.
\end{enumerate}
Verstärke Forderung (i): (Borel, Lebesgue)
\begin{enumerate}
\item[(i')]\textbf{$\sigma$-Additivität\footnote{$\sigma$: abzählbar, unendlich oft.}}: Für Folgen $(M_n)_{n\in \N}$ paarweise disjunkter Teilmengen $M_n \subset \R^d$ gilt:
\begin{equation}
\vol\bigg( \bigcup_{n\in \N} M_n \bigg) = \sum_{n\in \N} \underbrace{\vol (M_n)}_{\in [0,\infty]} 
\end{equation}
\begin{remark}
Wegen des Umordnungssatzes spielt die Reihenfolge der Summanden keine Rolle, da sie alle positiv sind.
\end{remark}
$\leadsto$ flexibilisiert Volumenmessung entscheidend, wir können also komplizierte Figuren durch einfach Figuren approximieren.
\end{enumerate}
\end{subequations}
Cantons Mengenlehre $\leadsto$ Existenz von ``naiver'' Volumenfunktion wurde hinterfragt: %\vspace{-10pt}
\paragraph{Maßproblem} Existiert eine Volumenfunktion $\vol: \mathcal{P}(\R^d) \longrightarrow [0,\infty]$ mit (i') + (ii) + (iii)?
\begin{satz}[Vitali, 1905] Nein, das naive Maßproblem ist unlösbar.
\begin{proof}
Aus dem Auswahlaxiom folgt die Existenz ``verrückter'' (d.h. geometrisch unvorstellbarer) Teilmengen des $\R^d$. Hier existiert $M \subset \R^d$, ein \emph{Vertretersystem} für Nebenklassen von $\mathbb{Q}^d$ (Untergruppe von $\R^d$) in $\R^d$. Der Quotient abelscher Gruppen $\R^d/\mathbb{Q}^d$ ist also die Menge der Nebenklassen. Die Nebenklassen $a+\mathbb{Q}^d$ für $a\in \R^d$ partitionieren (d.h. zerlegen disjunkt) $\R^d$ (überabzählbar viele). Für alle $a,b \in \R^d$ besteht Dichotomie:
\begin{enumerate}[label=\roman*),topsep=3pt, itemsep=0pt]
\item entweder $a+\mathbb{Q}^d = b + \mathbb{Q}^d$ (nämlich wenn $a-b \in \mathbb{Q}^d$),
\item oder $(a+\mathbb{Q}^d)\cap (b+\mathbb{Q}^d) = \emptyset$ (nämlich wenn $a-b \notin \mathbb{Q}^d$).
\end{enumerate}
D.h. für alle $a\in \R^d$ besteht $M \cap (a+\mathbb{Q}^d)$ aus genau einem Element. Daraus folgt, die Translate $q+M$ (abzählbar viele) für $q \in \mathbb{Q}^d$ partitionieren $\R^d$. Aus der $\sigma$-Additivität von Volumen folgt
\begin{equation}
\underbrace{\vol (\R^d)}_{>0}= \sum_{q \in \mathbb{Q}^d} \underbrace{\vol (q+M)}_{\overset{\text{Bew Inv}}{=} \vol(M)}
\end{equation} 
und somit also $\vol (M)>0$.\\
Jetzt wähle $M$ spezieller, nämlich beschränkt, z.B. für $O\subset \R^d$ offen können wir $M$ so wählen, dass $M\subset O$, weil $a+\mathbb{Q}^d$ dicht in $\R^d$, also $(a+\mathbb{Q}^d)\cap O \neq \emptyset$. Z.B. wähle $M \subset (0,\frac{1}{2})^d$, so enthält $[0,1]^d$ abzählbar unendlich viele paarweise disjunkte Translate $q+M$, nämlich für alle $q\in \mathbb{Q}^d \cap (0,\frac{1}{2})^d$ gilt
\begin{equation}
V:= \bigcup_{q\in(0,\frac{1}{2})^d \cap \mathbb{Q}^d} (q+M) \subset [0,1]^d
\end{equation}
weil $\vol (V) + \underbrace{\vol ([0,1]^d-V)}_{\geq 0} = \underbrace{\vol ([0,1]^d)}_{=1}$. Daraus folgt $\vol (V) \leq 1 < \infty$ und 
\begin{equation}
\vol(V) = \sum_{q\in(0,\frac{1}{2})^d \cap \mathbb{Q}^d} \underbrace{\vol (q+M)}_{=\vol(M)}
\end{equation}
Somit muss gelten $\vol(M)=0$. \Lightning
\end{proof}
\end{satz}
Noch dramatischer: In $\dim \geq 3$ kann man je zwei Teilmengen (unter sehr allgemeinen Annahmen) aus demselben (abzählbaren, oft sogar endlichen) ``Bausatz'' zusammensetzen.
\begin{satz}[Banach-Tarski, 1924] Seien $A,B\subset \R^d$ Teilmengen mit nichtleerem Inneren.
\begin{enumerate}[(\roman*)]
\item Sei $d \geq 3$ und seien $A,B$ beschränkt. Dann existieren endlich viele Teilmengen $M_k \subset \R^d$ und Bewegungen $\phi_k$ des $\R^d$, so dass \emph{disjunkte Zerlegungen} $A=\bigsqcup_k M_k$ und $B = \bigsqcup_k \phi(M_k)$ bestehen.
\item Jetzt $d\geq 1$ beliebig und $A,B$ nicht notwendig beschränkt. Dann existieren abzählbar viele Teilmengen $M_k \subset \R^d$ und Bewegungen $\phi_k$, sodass  \emph{disjunkte Zerlegungen} $A=\bigsqcup_k M_k$ und $B = \bigsqcup_k \phi(M_k)$ bestehen.
\end{enumerate}
Der Beweis verwendet Gruppentheorie, Struktur von orthogonalen Gruppen $\mathrm{O}(d)$. (nicht mehr auflösbar für $d\geq 3$.)
\end{satz}
Das naive \emph{Inhaltsproblem}, also eine Volumenfunktion mit Eigenschaften (i), (ii) und (iii), ist lösbar in $d\leq 2$, aber nicht eindeutig, nicht lösbar in $d\geq 3$. (Banach 1923, Hausdorff 1914) Dies führt zu:
\vspace{-10pt}
\paragraph{Maßproblem (post-paradox)}: Man definiere eine Volumenfunktion $\vol: \mathcal{F} \longrightarrow [0,\infty]$ mit Eigenschaften (i'), (ii) und (iii) auf einer möglich großen und flexiblen Familie $\mathcal{F} \subset \mathcal{P}(\R^d)$, die die geometrisch wichtigen Teilmengen umfasst und abgeschlossen ist unter grundlegenden mengentheoretischen Operationen (Vereinigung, Schnitt, Differenz und Komplement). 
\subsection{Ringe und Algebren}
\marginpar{\tiny{17.10.2019}}
Wir untersuchen Familien von Teilmengen (einer festen Menge), die unter grundlegenden (endlichen) Mengenoperationen abgeschlossen/ stabil sind. ($\cup,\cap,\setminus,\complement$) \\
Sie werden Definitionsbereiche der allgemeinsten von uns betrachteten Volumenfunktion sein. (``Inhalte'')

\subsubsection{Die Ringstruktur auf Potenzmengen}
Sei $X$ eine Menge. Die Potenzmenge ist definiert als die Familie aller Teilmengen $\mathcal{P}(X)$. Wir können die Potenzmenge ebenfalls auffassen als
\begin{equation}
\mathcal{P}(X) \xleftrightarrow[\text{bij}]{\cong} \{0,1\}^X = \{f:X \longrightarrow \{0,1\}\}
\end{equation}
da
\begin{subequations} 
\begin{align}
A &\longmapsto \chi_A(x) = \begin{cases}
1, \quad & \text{falls } x\in A\\
0, \quad & \text{sonst}
\end{cases}\\
f^{-1}(1) & \longmapsfrom f
\end{align}
\end{subequations}
wobei $\chi_A$ de charakteristische Funktion von $A$ ist. \newline \newline
Wir fassen nun $\{0,1\}$ auf als den Körper mit 2 Elementen (Restklassen modulo 2). So ist $\{0,1\}^X$ ein kommutativer Ring mit Eins (multiplikatives Einselement) (im Sinne der Algebra), sogar eine $\mathbb{F}_2$-Algebra.
\begin{remark} Die Addition und Multiplikation von Funktionen erfolgt punktweise:
\begin{enumerate} [-, itemsep=0pt,topsep=3pt]
\item  $(f+g)(x) := f(x)+g(x)$ 
\item $(fg)(x)= f(x) \cdot g(x)$
\end{enumerate}
und $\{0,1\} = \mathbb{F}_2$ ist ein Körper mit zwei Elementen.
\end{remark}
Die Nullelement ist $f \equiv 0$, also $\chi_\emptyset$ und das Einselement ist $\chi_X (\equiv 1)$. Die Addition von charakteristischen Funktionen entspricht der symmetrischen Differenz $A \triangle B$ und die Multiplikation entspricht dem Duchrschnitt von Mengen. Also
\begin{subequations}
\begin{align}
\chi_A + \chi_B = A \triangle B \\
\chi_A \cdot \chi_B = A \cap B
\end{align}
\end{subequations}
Somit ist $(\mathcal{P}(X), \triangle, \cap) \cong (\mathbb{F}_2^X, +, \cdot)$ ein kommutativer Ring mit dem Nullelement $\emptyset$ bzw. $\chi_\emptyset$ und dem Einselement $X$ bzw. $\chi_X$. 
\subsubsection{Ringe und Algebren}
\begin{definition}
\begin{mdframed}
Eine Familie $\mathcal{R}\subset \mathcal{P}(X)$ heißt
\begin{enumerate}[itemsep=0pt,topsep=3pt]
\item[($\rho$)] ein \textbf{Ring} auf $X$, falls sie ein Unterring von $(\mathcal{P}(X), \triangle, \cap)$ ist.
\item[($\alpha$)] eine \textbf{Algebra} auf $X$, falls sie außerdem das Einselement enthält, d.h. $X \in \mathcal{R}$.
\end{enumerate}
\end{mdframed}
\end{definition}
\begin{remark}
\begin{small}
``Algebra'' wird in verschiedenen Bedingungen verwendet, nämlich die Algebra als ein mathematisches Gebiet, eine Algebra als algebraische Struktur im Sinne der Algebra und eine Algebra im Sinne der obigen Definition.
\end{small}
\end{remark}
$(\rho)$ bedeutet $\emptyset \in \mathcal{R}$, abgeschlossen unter Addition ($\triangle$) (dasselbe wie Subtraktion, da $\mod 2$) und Multiplikation ($\cap$), d.h.
\begin{equation}
A,B \in \mathcal{R} \implies A\triangle B, A\cap B \in \mathcal{R}
\end{equation} 
d.h. $\triangle$- stabil und $\cap$- stabil. Wir können $\triangle, \cap$ ausdrücken durch $\setminus$ und $\cup$:
\begin{subequations}
\begin{align}
A \triangle B & = (A \setminus B) \cup (B \setminus A) \\
A \cap B & = A \setminus (A \setminus B)
\end{align}
und umgekehrt
\begin{align}
A \setminus B &= (A \triangle B) \cap A \\
A \cup B &= (A \triangle B) \triangle (A \cap B)
\end{align}
\end{subequations}
\textit{Bemerkung.} Die letzte Gleichung gilt, da $(A\triangle B)$ und $(A\cap B)$ disjunkt sind. \newline \newline
Daraus folgt die Charakterisierung von Ringen:
\begin{lemma}
\begin{mdframed}
Eine Familie $\mathcal{R}\subset \mathcal{P}(X)$ ist genau dann ein Ring auf $X$, wenn
\begin{enumerate}[(\roman*), topsep=3pt, itemsep=0pt]
\item $\emptyset \in \mathcal{R}$,
\item $\setminus$- stabil, d.h. $A,B \in \mathcal{R} \implies A \setminus B \in \mathcal{R}$,
\item $\cup$- stabil, d.h. $A, B \in \mathcal{R} \implies A \cup B \in \mathcal{R}$.
\end{enumerate}
\end{mdframed}
\end{lemma}
entspricht für Algebren:
\begin{lemma}
\begin{mdframed}
Eine Familie $\mathcal{A} \subset \mathcal{P}(X)$ ist genau dann eine Algebra auf $X$, wenn
\begin{enumerate}[topsep=3pt, itemsep=0pt]
\item[(i)] $\emptyset \in \mathcal{A}$,
\item[(iii)] $\cup$- stabil,
\item[(iv)] $\complement$- stabil, d.h. $A \in \mathcal{A} \implies \complement A := X\setminus A \in \mathcal{A}$
\end{enumerate}
\end{mdframed}
\begin{proof}
\begin{subequations}
Sind diese Eigenschaften erfüllt, so implizieren (i + iv), dass
\begin{equation}
X = \complement \emptyset \in \mathcal{A}
\end{equation}
``$\setminus$'' kann ausgedrückt werden durch ``$\cup$'' und ``$\complement$'': Aus
\begin{equation}
\complement (A \setminus B) = (\complement A) \cup B
\end{equation}
folgt
\begin{equation}
A \setminus B = \complement \big( (\complement A) \cup B \big)
\end{equation}
Also ist $\mathcal{A}$ ein Ring, und damit $\mathcal{A}$ eine Algebra. \\
Ist umgekehrt $\mathcal{A}$ eine Algebra, so gelten (i + iii). Da auch $X \in \mathcal{A}$, können wir ``$\complement$'' durch ``$\setminus$'' ausdrücken
\begin{equation}
\complement A = X \setminus A
\end{equation}
Also gilt auch (iv).
\end{subequations}
\end{proof}
\end{lemma}
\textbf{Folgerung.} Ist $\mathcal{R}$ ein Ring auf $X$ und $A,B \in \mathcal{R}$, so auch $A\setminus B, A\cap B, B\setminus A$ und $A\cup B \in \mathcal{R}$. (\textit{Bem.} Alle in $A\cup B$ enthalten.) Ist $\mathcal{A}$ eine Algebra auf $X$ und $A,B \in \mathcal{A}$, so ist außerdem auch $\complement (A \cup B) \in \mathcal{A}$.

\begin{example} \
\begin{enumerate}
\item[(o)] $\{ \emptyset\} \subset \mathcal{P}(X)$ ist ein Ring auf $X$, \newline
$\{\emptyset, X\} \subset \mathcal{P}(X)$ ist die kleinste Algebra auf $X$, $\mathcal{P}(X) \subset \mathcal{P}(X)$ die größte.
\item [(i)] $\{ \emptyset, A \} \subset \mathcal{P}(X)$ ist ein Ring auf $X$ für ein $A \in \mathcal{P}(X)$, \newline
$\{\emptyset, A, \complement A, X\} \subset \mathcal{P}(X) $ ist eine Algebra auf $X$.
\item[(ii)] Die Familie der endlichen (bzw. abzählbaren) Teilmengen von $X$ ist ein Ring. (eine Algebra, nur falls $X$ selbst endlich bzw. abzählbar) \newline
Die Familie der Teilmengen, die endlich (bzw. abzählbar) sind oder endliches (bzw. abzählbares) Komplement haben, ist eine Algebra.
\end{enumerate}
Weitere Beispiele folgen nach der Diskussion vom Erzeugendensystem.
\end{example}
\textit{Beobachtung.} Der Durchschnitt beliebig vieler Ringe (bzw. Algebren) auf einer festen Menge ist wieder ein Ring (bzw. eine Algebra). Zu jeder Menge $\mathcal{E} \subset \mathcal{P}(X)$ gibt es eine(n) bezüglich mengentheoretischer Inklusion kleinste(n) Ring (bzw. Algebra), der (die) $\mathcal{E}$ umfasst, nämlich den Durchschnitt aller Ringe (bzw. Algebren), die $\mathcal{E}$ umfassen.
\begin{definition}[\textbf{Erzeugendensystem}]
\begin{mdframed}
Der von einer Familie $\mathcal{E} \subset \mathcal{P}(X)$ erzeugte Ring auf $X$ ist der kleinste Ring, der sie enthält. Man nennt $\mathcal{E}$ ein \emph{Erzeugendensystem} dieses Rings, oder \emph{Erzeuger}. (Analog für Algebren)
\end{mdframed}
Die Algebra eines Erzeugendensystems ist oft die einfachste Art, eine(n) Ring bzw. Algebra zu beschreiben. 
\end{definition}
Ein Ring geht aus einem Erzeugendensystem $\mathcal{E} \subset \mathcal{P}(X)$ \emph{konstruktiv} durch einen \emph{abzählbaren} (induktiv!) Prozess hervor, ebenso eine Algebra. \vspace{0.5pc}
\newline
\textbf{Ring.} Definiere induktiv eine Folge von Familien $\mathcal{F}_0 \subset \mathcal{F}_1 \subset ... \subset \mathcal{F}_n \subset ... \subset \mathcal{P}(X)$ mit
\begin{subequations}
\begin{align}
\mathcal{F}_0 : = &\mathcal{E}\cup \{\emptyset\}  \\
\mathcal{F}_n :=  &\{ A\setminus B, A \cup B \mid A,B \in \mathcal{F}_{n-1} \},\  n \geq 1
\end{align}
So ist $\bigcup_{n\in \N} \mathcal{F}_n \subset \mathcal{P}(x)$ $\setminus$- und $\cup$-stabil, also ein Ring.  \vspace{0.5pc} \newline 
\textbf{Algebra.} analog.
\end{subequations}
\subsubsection{Halbringe}
Hat ein Erzeugendensystem strukturelle Eigenschaft, so ist die Beschreibung des erzeugenden Rings einfach. Eine natürliche auftretende Bedingung ist:
\begin{definition}[\textbf{Halbringe}]
\begin{mdframed}
Eine Familie $\mathcal{H}\subset \mathcal{P}(X)$ heißt ein \emph{Halbring} auf $X$, falls
\begin{enumerate}[(\roman*), topsep=3pt, itemsep=0pt]
	\item $\emptyset \in \mathcal{H}$,
	\item $\mathcal{H}$ ist $\cap$- stabil,
	\item Für $A,B \in \mathcal{H}$ existieren \emph{disjunkte} Teilmengen $C_1,...,C_n \in \mathcal{H}$ mit $A\setminus B = C_1 \cup ...\cup C_n$.
\end{enumerate}
\end{mdframed}
\end{definition}
\begin{remark}
Halbring ist eine Verallgemeinerung des Begriffs Ring, Ringe sind also Halbringe.
\end{remark}
\begin{example} \
\begin{enumerate}
	\item[(o)] $\{ \emptyset\} \subset \mathcal{P}(X)$ ist ein Halbring auf $X$.
	\item[(i)] Die Familie bestehend aus $\emptyset$ und allen (einelementigen) Teilmengen $\{\emptyset\} \cup \{ \{a\} \mid a \in X \}$  ist ein Halbring auf $X$, sie erzeugt den Ring der endlichen Teilmengen von $X$.
\end{enumerate}
Der Grundbaustein für später:
\begin{enumerate}
	\item[(ii)] Die Familie der \emph{halboffenen} Intervalle $[a,b) \subset \R$, falls $a<b$, also $\{[a,b) \mid a,b \in \R, a<b\}$, ist ein Halbring auf $\R$.
\end{enumerate}
\end{example}
Beschreibe den von einem Halbring erzeugenden Ring, die folgende Beobachtung wird darüber hinaus nützlich sein:

\begin{lemma}[Simultane Zerlegung]
%\begin{mdframed}
Zu beliebigen Teilmengen $H_1,...,H_m \in \mathcal{H}$ existieren paarweise disjunkte Teilmengen $H_1',...,H_n' \in \mathcal{H}$, sodass jedes $H_i$ sich als die Vereinigung einiger $H_j'$'s darstellen lässt.
 \label{lemmaA}
%\end{mdframed}
\begin{proof}
Betrachte die $2^m-1$ Durchschnitte der Form $G_1 \cap ... \cap G_m$, wobei $G_i = H_i$ oder $\complement H_i$ und nicht alle gleich $\complement H_i$. Sie sind paarweise disjunkt und zerlegen $H_1 \cup ... \cup H_m$. Jedes $H_i$ ist die Vereinigung von $2^{m-1}$ von ihnen. Es genügt zu zeigen, dass diese Durchschnitte disjunkte Vereinigungen von Teilmengen aus $\mathcal{H}$ sind. Da Halbringe $\cap$-stabil sind, reicht es zu zeigen, dass die Teilmengen der Form
\begin{equation}
H \cap \complement \widetilde{H}_l \cap ... \cap \complement \widetilde{H}_1	\quad \text{mit }H,\widetilde{H}_1,...,\widetilde{H}_l \in \mathcal{H}
\end{equation}
disjunkte Vereinigungen von Teilmengen in $\mathcal{H}$ sind. \\
Da für $H \cap \complement \widetilde{H}_l = H \setminus \widetilde{H}_l$ (Axiom (iii)) gilt, reduziert die Behauptung für $l$ auf Behauptung für $l-1$, mit Induktion liefert dann die Behauptung.
\end{proof}
\end{lemma}

\marginpar{\tiny{21.10.2019}}

\begin{proposition}
Jede Teilmenge im von einem Halbring $\mathcal{H}$ erzeugten Ring $\mathcal{R}$ ist eine endliche disjunkte Vereinigung von Teilmengen in $\mathcal{H}$, d.h. \begin{scriptsize}(ein einfacher Erzeugungsprozess!)\end{scriptsize}
\begin{equation}
\mathcal{R} := \left\{ \bigcup_{k=1}^n A_k \  \Big\rvert \ n\in \N, A_1,...,A_n \in \mathcal{H} \text{ paarweise disjunkt} \right\}
\label{eqI14}
\end{equation}
\begin{proof}
Sei $\mathcal{R}$ die Familie der endlichen \emph{disjunkten} Vereinigungen von Teilmengen in $\mathcal{H}$. Mit dem letzten \hyperref[lemmaA]{Lemma} ist $\mathcal{R}$ gleich der Familie aller endlichen Vereinigungen von Teilmengen in $\mathcal{H}$. Sie ist offensichtlich $\cup$-stabil. Zu verifizieren bleibt die $\setminus$- Stabilität. Seien hierzu $A = A_1 \cup ... \cup A_m$ und $B = B_1 \cup ... \cup B_n$, $A_i, B_j \in \mathcal{H}$. Aus dem \hyperref[lemmaA]{Lemma}  folgt, dass es endlich viele nichtleere, \emph{paarweise disjunkte} $H_k' \in \mathcal{H}$ existieren, sodass jedes $A_i$ und $B_j$ eine Vereinigung einiger $H_k'$'s ist. Daraus folgt, dass auch $A$ und $B$ Vereinigungen einiger $H_k'$'s sind. So ist auch $A\setminus B$ die Vereinigung einiger $H_k'$'s, nämlich derer, die in $A$, aber nicht in $B$ enthalten sind. Also ist $\mathcal{R}$ ein Ring \begin{scriptsize}($\cup$-stabil)\end{scriptsize}, enthalten in von $\mathcal{H}$ erzeugendem Ring, also gleich.
\end{proof}
\end{proposition}

\subsubsection{Produkte von Halbringen und Ringen}
Sind $\mathcal{F}_i \subset \mathcal{P}(X_i), i=1,...,n$ Familien von Teilmengen, so entstehen das Produkt von ``Quadern''
\begin{equation}
\begin{split}
\mathcal{F}_1 \ast ... \ast \mathcal{F}_n := \{ M_1 \times ... \times M_n \mid M_i \in \F_i \text{ für } i=1,...,n\}  \\ \subset  \mathcal{P}(X_1 \times ... \times X_n)
\end{split}
\end{equation}
und die $\cup$-stabile Hülle, die Familie $\F_1 \boxtimes ... \boxtimes \F_n$ der endlichen Vereinigungen von ``Quadern'' in $\F_1 \ast ... \ast \F_n$, die Figuren, d.h.
\begin{equation}
\F_1 \boxtimes ... \boxtimes \F_n = \left\{ \bigcup_{i=1}^m  M_{1_i} \times ... \times M_{n_i} \ \Big\vert \  M_{j_k} \in \F_j,  1 \leq j \leq n, 1 \leq k \leq m, m \in \N			\right\}
\end{equation}
Beide Produkte $\ast$ und $\boxtimes$ sind \emph{assoziativ}, z.B.
\begin{subequations}
\begin{equation}
(\F_1 \ast \F_2) \ast \F_3 = \F_1 \ast \F_2 \ast \F_3 = \F_1 \ast (\F_2 \ast \F_3)
\end{equation}
und
\begin{equation}
(\F_1 \boxtimes \F_2) \boxtimes \F_3 = \F_1 \boxtimes \F_2 \boxtimes \F_3 = \F_1 \boxtimes (\F_2 \boxtimes \F_3)
\end{equation}
\end{subequations}
Wir definieren weiter
\begin{equation}
\mathcal{Z} = \mathcal{Z} (\F_1,...,\F_n) \subset \F_1 \ast ... \ast \F_n
\end{equation}
die Zylindermenge
\begin{equation}
\begin{split}
\pi_k^{-1}(M_k) = X_1 \times ... \times X_{k-1} \times M_k \times X_{k+1} \times ... \times X_n \quad\\ \text{mit } 1 \leq k \leq n, M_k \in \F_k
\end{split}
\end{equation}
wobei $\pi_k: X_1 \times ... \times X_n \to X_k, (x_1,...,x_n) \mapsto x_k$ die natürliche Projektion ist.
\begin{proposition} \
%\begin{mdframed}
\begin{enumerate}[(\roman*),topsep=5pt]
	\item Seien $\mathcal{H}_i \subset \mathcal{P}(X_i)$ Halbringe $(i=1,...,n)$ und $\mathcal{R}_i \subset \mathcal{P}(X_i)$ die von ihnen erzeugten Ringe. Dann ist $\mathcal{H}_1 \ast ... \ast \mathcal{H}_n$ ein Halbring auf $X_1 \times ... \times X_n$ und $\mathcal{H}_1 \boxtimes ... \boxtimes \mathcal{H}_n =\mathcal{R}_1 \boxtimes ... \boxtimes \mathcal{R}_n$ der von ihm erzeugte Ring.
	\item Sind $\mathcal{E}_i \subset \mathcal{H}_i$ Erzeugendensysteme der Halbringe, so ist $\mathcal{Z}(\mathcal{E}_1,...,\mathcal{E}_n)$ ein Erzeugendensystem von $\mathcal{H}_1 \ast ... \ast \mathcal{H}_n$ als Halbring sowie (folglich) ein Erzeugendensystem von $\mathcal{H}_1 \boxtimes ... \boxtimes \mathcal{H}_n$ als Ring.
\end{enumerate}
%\end{mdframed}
\label{propC}
\begin{proof} \
\begin{enumerate}[(\roman*),topsep=5pt]
	\item Zunächst im Fall $n=2$. Klar enthält $\mathcal{H}_1 \ast \mathcal{H}_2$ auch $\emptyset$ und ist $\cap$-stabil. Wir betrachten die disjunkte Zerlegung
	\begin{equation}
	\begin{split}
		&(A_1 \times A_2) \setminus (B_1 \times B_2) =\\
	 \big( \underbrace{(A_1 \cap B_1)}_{\in \mathcal{H}_1} \times \underbrace{(A_2 \setminus B_2)}_{\substack{\text{zerlegbar in} \\ \text{Teilmengen}\\ \text{aus }\mathcal{H}_2}} \big) & \sqcup \underbrace{\big( \underbrace{(A_1 \setminus B_1)}_{\substack{\text{zerlegbar in} \\ \text{Teilmengen}\\ \text{aus }\mathcal{H}_1}} \times \underbrace{(A_2 \setminus B_2)}_{\substack{\text{zerlegbar in} \\ \text{Teilmengen}\\ \text{aus }\mathcal{H}_2}} \big)}_{\text{zerlegbar in Teilmengen aus } \mathcal{H}_1 \times \mathcal{H}_2}
		\sqcup \underbrace{\big( (A_1 \setminus B_1) \times (A_2 \cap B_2)\big)}_{\text{analog zerlegbar}}
	\end{split}
	\end{equation}
	Also ist $(A_1 \times A_2) \setminus (B_1 \times B_2) $ disjunkt zerlegbar in Teilmengen aus $\mathcal{H}_1 \ast \mathcal{H}_2$, also erfüllt Axiom (iii) für Halbringe, d.h.  $\mathcal{H}_1 \ast \mathcal{H}_2$ ist ein Halbring. \\
	Mit Induktion liefert dann die Behauptung auch für $\mathcal{H}_1 \ast ... \ast \mathcal{H}_n, n\geq 1$. \newline
	Aus \eqref{eqI14} folgt, dass $\mathcal{H}_1 \boxtimes ... \boxtimes \mathcal{H}_n$ der von $\mathcal{H}_1 \ast ... \ast \mathcal{H}_n$ erzeugte Ring ist.
	\item Für jedes $k$ gilt: Der Halbring auf $X_1 \times ... \times X_n$ bestehend aus den Zylindermengen $\pi_k^{-1}(H_k)$ für $H_k \in \mathcal{H}_k$ wird erzeugt von den Zylindermengen $\pi_k^{-1}(E_k)$ für $E_k \in \mathcal{E}_k$. Da
	\begin{equation}
	\begin{split}
	& \mathcal{E}_k \subset \\ 
	& \underbrace{\{ H_k \in \mathcal{H}_k \mid \pi^{-1}_k (H_k) \text{ gehört zum von den } \pi^{-1}_k(E_k)   \text{ für } E_k \in \mathcal{E}_k \text{ erzeugten Halbring} \}}_{\text{ist Halbring, deshalb Gleicheit}}
	 \\ & \subset \mathcal{H}_k
	\end{split}
	\end{equation}
	folgt, dass $\mathcal{Z}(\mathcal{E}_1,...,\mathcal{E}_k)$ erzeugen denselben Halbring $\mathcal{Z}(\mathcal{H}_1,...\mathcal{H}_n)$ und damit denselben wie $\mathcal{H}_1 \ast ... \ast \mathcal{H}_n$.
\end{enumerate}
\end{proof}
\end{proposition}
\begin{definition}
Wir nennen den Halbring $\mathcal{H}_1 \ast ... \ast \mathcal{H}_n$ das Produkt der Halbringe $\mathcal{H}_i$, den Ring $\mathcal{R}_1 \boxtimes ... \boxtimes \mathcal{R}_n$ das Produkt der Ringe $\mathcal{R}_i$.
\end{definition}

\begin{example}[\textit{Hauptbeispiel}, Quader und Figuren in $\R^d$]
Ist $a,b \in \R^d$ mit $a_i < b_i \forall i$, so entsteht achsenparalleler halboffener Quader
\begin{subequations}
\begin{equation}
[a,b) := [a_1, b_1) \times ... \times [a_d,b_d)
\end{equation}
Wir bezeichnen
\begin{equation}
\mathcal{Q}^d := \text{Familie dieser Quader}
\end{equation}
und
\begin{equation}
\mathcal{I} := \mathcal{Q}^1, \text{Familie der halboffenen Intervalle}
\end{equation}
Also gilt
\begin{equation}
\mathcal{Q}^d = \underbrace{\mathcal{I} \ast ... \ast \mathcal{I}}_{d\text{-Mal}}
\end{equation}
Aus der \hyperref[propC]{letzten Proposition} folgt, dass $\mathcal{Q}^d$ ein Halbring auf $\mathcal{R}^d$ ist. Es folgt ebenfalls, dass
\begin{equation}
\mathcal{F}^d := \underbrace{\mathcal{I} \boxtimes ... \boxtimes \mathcal{I}}_{d\text{-Mal}}
\end{equation}
der Ring erzeugt von $\mathcal{Q}^d$ ist, also der Ring der $d$-dimensionale ``Figuren''.
Wir haben gesehen: Figuren sind disjunkte Vereinigungen von Quadern.
\end{subequations}
\end{example}

Wir arbeiten aus technischen Gründen mit halboffenen Intervallen und Quadern. Besonders übersichtliche sind Halbringe, die abgeschlossen unter Produktbildung sind,  jedoch nicht die Teilmengen enthalten, die uns geometrisch primär interessieren: die offenen und abgeschlossenen Quader, die nicht achsenparallele sind, Polygone und Polytope, gekrümmte ``elementare Geometrie'' sowie Gebilde: Scheiben, Bälle, Zylinder und Kegel. Deshalb müssen wir unsere Ringe weiter anreichern und flexibilisieren, damit sie stabil unter abzählbaren Vereinigungen sind. $\leadsto$ $\sigma$-Algebra.

\subsection{Inhalte und Prämaße}
Wir beginnen mit der Untersuchung von Volumenfunktionen. Die grundlegende Forderung ist die \emph{Additivität}. Volumina dürfen nicht negativ sein, d.h. $\in [0,\infty] := [0, \infty) \cup \{ \infty\}$, also die erweiterte positive Halbgerade. 
\begin{remark}
Die erweiterten reellen Zahlen ist definiert als $\ol{\R} := \{- \infty\} \cup \R \{ \infty \}$ mit natürlichen Konventionen
\begin{subequations}
\begin{align}
	x + \infty = \infty & \text{ für } x > - \infty	\\
	x \cdot \infty = \infty &\text{ für } x> 0
\end{align}
Später werden wir außerdem sehen 
\begin{equation}
	0 \cdot \infty = 0
\end{equation}
\end{subequations}
(da $ 0 \cdot n \longrightarrow 0$ für $n \longrightarrow \infty$).
\end{remark}
\subsubsection{Inhalte auf Halbringen und Ringen}
Die allgemeinste von uns betrachteter Volumenfunktion Sorte ist endlich additiv und definiert auf Halbringen.

\begin{definition}[\textbf{Inhalt}]
\begin{mdframed}
Ein Inhalt auf einer Halbring $\mathcal{H}$ ist eine Funktion $\mu: \mathcal{H} \to [0,\infty]$ mit den Eigenschaften:
\begin{enumerate}[(\roman*),topsep=5pt, itemsep = 0 pt]
	\item $\mu (\emptyset) = 0$
	\item \emph{Additivität}: Sind $A_1,...,A_n \in \mathcal{H}$ paarweise disjunkt mit $A_1 \cup ... \cup A_n \in \mathcal{H}$ \footnote{Die Voraussetzung ist redundant, falls $\mathcal{H}$ ein Ring ist.}, so gilt $\mu(A_1 \cup ... \cup A_n) = \mu(A_1) + ... +\mu(A_n)$
\end{enumerate}
\end{mdframed}
\end{definition}

\begin{example} \
\begin{enumerate}
	\item[(o)] $\mu \equiv 0$ ``der Nullinhalt'' und\\
				$\nu(A) = \begin{cases}
					0,       & A = \emptyset \\
					\infty, & \text{sonst}
				\end{cases}$
				sind stets Inhalte auf beliebigen Halbringen.
	\item[(i)] Sei $X$ nichtleer. Betrachte die Algebra $\{ \emptyset, X\} \subset \mathcal{P}(X)$. So wird ein Inhalt \\
			$\begin{cases}
				\emptyset \mapsto 0 \\
				x \mapsto v \in [0,\infty] \text{ beliebig}
			\end{cases}$
			definiert.
\end{enumerate}
\end{example}

Der Grundbaustein (für Lebesgue-Maß):
\begin{example}[halboffene Intervalle in $\R$]
Wir betrachten den Halbring $\I = \mathcal{Q}^1 \subset \mathcal{P}(\R)$. So wird ein Inhalt gegeben durch die euklidische Länge
\begin{align}
\lambda^1_\I : \I \to [0, \infty), \quad \lambda^1_\I \big([a,b)\big) := b-a \ (a<b)
\end{align}
Wir überprüfen die Additivität: Sei $a = x_0 < x_1 < ... < x_{n} = b$ eine Unterteilung von $[a,b)$. So entsteht die disjunkte Zerlegung $[a,b) = [a,x_1) \sqcup ... \sqcup [x_{n-1},b)$. Es folgt 
\begin{equation}
\underbrace{\lambda^1_\I \big([a,b)]}_{b-a} = \underbrace{\lambda^1_\I \big( [a,x_1)\big)}_{x_1-a} + \underbrace{\lambda^1_\I \big([x_1,x_2)\big)}_{x_2 -x_1} + ... + \underbrace{\lambda^1_\I \big( [x_{n-1},b ) \big)}_{b-x_{n-1}}
\end{equation}
\end{example}
\marginpar{\tiny{24.10.2019}}

\end{document}
