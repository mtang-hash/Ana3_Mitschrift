\documentclass[12pt,a4paper]{article}
\usepackage[T1]{fontenc}
\usepackage[utf8]{inputenc}
\usepackage{lmodern}
\usepackage[left=28mm,top=28mm,right=28mm,bottom=28mm] {geometry}
\usepackage{amsfonts}
\usepackage{mathrsfs}
\usepackage[intlimits]{amsmath}
\usepackage{stmaryrd}
\usepackage{relsize}
\usepackage{etoolbox}
\usepackage[ngerman]{babel}
\usepackage[utf8]{inputenc}
\usepackage[T1]{fontenc}
\usepackage{marvosym}
\usepackage[shortlabels]{enumitem}
\usepackage{mathtools}
\usepackage{amssymb}
\usepackage{cancel}
\usepackage{mdframed}
\usepackage{framed}
\usepackage{mathtools}
\usepackage{tablefootnote} 
\usepackage{listings}
\usepackage{amsthm}
\usepackage{xcolor}
\usepackage{etoolbox}
\usepackage[all]{xy}
\usepackage{tikz}
\usepackage{thmtools}
\usepackage[
   pdfpagelabels=true,
   pdftitle={Analysis III: Maßtheorie und Integralrechnung mehrerer Variablen},
   pdfauthor={MT},
 ]{hyperref}
\usepackage{bookmark}
\usetikzlibrary{cd}
\usetikzlibrary{calc}
\theoremstyle{definition}
\newtheorem*{definition}{Definition}
\newtheorem*{satz}{Satz}
\newtheorem*{lemma}{Lemma}
\newtheorem*{proposition}{Proposition}
\newtheorem*{korollar}{Korollar}
\newtheorem*{folgerung}{Folgerung}
\newtheorem*{example}{Beispiel}
\newtheorem*{hauptbsp}{Hauptbeispiel}
\theoremstyle{remark}
\newtheorem*{remark}{Bemerkung}
\newtheorem*{remark'}{Nebenbemerkung}
\newtheorem*{beobachtung}{Beobachtung}
\AfterEndEnvironment{lemma}{\noindent\ignorespaces}
\AfterEndEnvironment{definition}{\noindent\ignorespaces}
\AfterEndEnvironment{example}{\noindent\ignorespaces}
\AfterEndEnvironment{theorem}{\noindent\ignorespaces}
\AfterEndEnvironment{satz}{\noindent\ignorespaces}
\AfterEndEnvironment{korollar}{\noindent\ignorespaces}
\AfterEndEnvironment{remark}{\noindent\ignorespaces}
\AfterEndEnvironment{remark'}{\noindent\ignorespaces}
\AfterEndEnvironment{proposition}{\noindent\ignorespaces}
\AfterEndEnvironment{proof}{\noindent\ignorespaces}
\let\existstemp\exists
\let\foralltemp\forall
\newcommand{\tikzmark}[1]{\tikz[overlay,remember picture] \node (#1) {};}
\newcommand{\vsubset}{\rotatebox[origin=c]{90}{$\subset$}}
\newcommand{\vphi}{\phi}
\newcommand{\ol}{\overline}
%Differentiation
\newcommand{\D}{\, \mathrm{d}}
%Bold Symbols
\newcommand{\R}{\mathbb{R}}
\newcommand{\C}{\mathbb{C}}
\newcommand{\N}{\mathbb{N}}
\newcommand{\Q}{\mathbb{Q}}
%Calligraphic Symbols
\newcommand{\II}{\mathcal{I}}
\newcommand{\FF}{\mathcal{F}}
\newcommand{\QQ}{\mathcal{Q}}
\newcommand{\EE}{\mathcal{E}}
\newcommand{\PP}{\mathcal{P}}
\newcommand{\TT}{\mathcal{T}}
\newcommand{\HH}{\mathscr{H}}
\newcommand{\RR}{\mathscr{R}}
\newcommand{\BB}{\mathscr{B}}
\newcommand{\CC}{\mathscr{C}}
\newcommand{\DD}{\mathscr{D}}
\renewcommand{\AA}{\mathscr{A}}
% Operator names
\newcommand{\Hom}{\operatorname{Hom}}
\newcommand{\del}{\partial}
\newcommand{\vol}{\operatorname{vol}}
\newcommand{\Var}{\operatorname{Var}} 
\newcommand{\Cov}{\operatorname{Cov}}
\newcommand{\End}{\operatorname{End}}
\newcommand{\SL}{\operatorname{SL}}
\newcommand{\Bild}{\begin{tiny}(Bild hier)\end{tiny}}
\renewcommand*{\exists}{\existstemp\mkern2mu}
\renewcommand*{\forall}{\foralltemp\mkern2mu}
\renewcommand{\emptyset}{\varnothing}
\renewcommand{\Re}{\operatorname{Re}}
\renewcommand{\Im}{\operatorname{Im}}
\renewcommand{\qedsymbol}{$\blacksquare$}
\renewcommand{\phi}{\varphi}
\renewcommand{\thesection}{\Roman{section}}
\makeatletter 
\AfterEndEnvironment{mdframed}{%
 \tfn@tablefootnoteprintout% 
 \gdef\tfn@fnt{0}% 
}
\numberwithin{equation}{section}

\setlist[enumerate,1]{label={(\roman*)}}

\title{\LARGE Mitschrift zur
 Vorlesung \\ \huge \scshape Analysis III   \\ \vspace{0.2pc} \normalfont \Large Maßtheorie und Integralrechnung mehrerer Variablen\footnote{im Wintersemester 2019/20 gelesen von Prof. Bernhard Leeb, Ph.D.}}
%\date{27. Oktober 2019}
\begin{document}
\maketitle
\tableofcontents 
\addtocontents{toc}{\protect\setcounter{tocdepth}{3}}
\newpage
\section{Maßtheorie}
\subsection{Maßproblem und Paradoxien}
\marginpar{\tiny{14.10.2019}}
Maßtheorie ist die Theorie des Volumens. Motivierende Beispiele sind:
\begin{enumerate}[label=\roman*),topsep=3pt, itemsep=0pt]
\item Volumina von Teilmengen des euklidischen Raums
\item Wahrscheinlichkeiten (= ``Volumina von Ereignissen'')
\end{enumerate}
Wir konzentrieren uns im Rest des Abschnitts auf $\R^d$. Wir wollen einen leistungsfähigen Volumenbegriff haben, sodass die Volumina von möglich vielen Teilmengen flexibel gemessen werden können. Unser erster ``naiver'' Ansatz wäre, dass wir Volumenmessung für \emph{alle} Teilmengen verlangen, also eine Funktion
\begin{equation*}
 \vol: \mathcal{P}(\R^d) \longrightarrow [0.\infty]
\end{equation*}
Unsere grundlegende Forderung ist die Additivität von Volumina bei Zerlegungen, also
\begin{enumerate}
\item[(i)] \textbf{(endliche) Additivität}: Sind $M_1,...,M_n\subset \R^d$ paarweise disjunkt, so gilt
\begin{equation*}
\vol (M_1\cup ... \cup M_n) = \vol(M_1)+...+\vol (M_n)
\end{equation*}
\end{enumerate}
Volumina als geometrische Größen sollten durch die metrische Struktur (Längenmessung) bestimmt sein, also invariant unter Symmetrien der metrischen Struktur:
\begin{enumerate}
\item[(ii)] \textbf{Bewegungsinvarianz}: Für jede Bewegung $\phi: \R^d \longrightarrow \R^d$ und jede Teilmenge $A\subset \R^d$ gilt
\begin{equation*}
\vol(\Phi(A)) = \vol (A)
\end{equation*}
\item[(iii)] \textbf{Normierung}: $\vol([0,1]^d)=1$.
\end{enumerate}
Verstärke Forderung (i): (Borel, Lebesgue)
\begin{enumerate}
\item[(i')]\textbf{$\sigma$-Additivität\footnote{$\sigma$: abzählbar, unendlich oft.}}: Für Folgen $(M_n)_{n\in \N}$ paarweise disjunkter Teilmengen $M_n \subset \R^d$ gilt:
\begin{equation*}
\vol\bigg( \bigcup_{n\in \N} M_n \bigg) = \sum_{n\in \N} \underbrace{\vol (M_n)}_{\in [0,\infty]} 
\end{equation*}
\begin{remark}
Wegen des Umordnungssatzes spielt die Reihenfolge der Summanden keine Rolle, da sie alle positiv sind.
\end{remark}
$\leadsto$ flexibilisiert Volumenmessung entscheidend, wir können also komplizierte Figuren durch einfach Figuren approximieren.
\end{enumerate}
Cantons Mengenlehre $\leadsto$ Existenz von ``naiver'' Volumenfunktion wurde hinterfragt: %\vspace{-10pt}
\paragraph{Maßproblem} Existiert eine Volumenfunktion $\vol: \mathcal{P}(\R^d) \longrightarrow [0,\infty]$ mit (i') + (ii) + (iii)?
\begin{satz}[Vitali, 1905] Nein, das naive Maßproblem ist unlösbar.
\begin{proof}
Aus dem Auswahlaxiom folgt die Existenz ``verrückter'' (d.h. geometrisch unvorstellbarer) Teilmengen des $\R^d$. Hier existiert $M \subset \R^d$, ein \emph{Vertretersystem} für Nebenklassen von $\mathbb{Q}^d$ (Untergruppe von $\R^d$) in $\R^d$. Der Quotient abelscher Gruppen $\R^d/\mathbb{Q}^d$ ist also die Menge der Nebenklassen. Die Nebenklassen $a+\mathbb{Q}^d$ für $a\in \R^d$ partitionieren (d.h. zerlegen disjunkt) $\R^d$ (überabzählbar viele). Für alle $a,b \in \R^d$ besteht Dichotomie:
\begin{enumerate}[label=\roman*),topsep=3pt, itemsep=0pt]
\item entweder $a+\mathbb{Q}^d = b + \mathbb{Q}^d$ (nämlich wenn $a-b \in \mathbb{Q}^d$),
\item oder $(a+\mathbb{Q}^d)\cap (b+\mathbb{Q}^d) = \emptyset$ (nämlich wenn $a-b \notin \mathbb{Q}^d$).
\end{enumerate}
D.h. für alle $a\in \R^d$ besteht $M \cap (a+\mathbb{Q}^d)$ aus genau einem Element. Daraus folgt, die Translate $q+M$ (abzählbar viele) für $q \in \mathbb{Q}^d$ partitionieren $\R^d$. Aus der $\sigma$-Additivität von Volumen folgt
\begin{equation*}
\underbrace{\vol (\R^d)}_{>0}= \sum_{q \in \mathbb{Q}^d} \underbrace{\vol (q+M)}_{\overset{\text{Bew Inv}}{=} \vol(M)}
\end{equation*} 
und somit also $\vol (M)>0$.\\
Jetzt wähle $M$ spezieller, nämlich beschränkt, z.B. für $O\subset \R^d$ offen können wir $M$ so wählen, dass $M\subset O$, weil $a+\mathbb{Q}^d$ dicht in $\R^d$, also $(a+\mathbb{Q}^d)\cap O \neq \emptyset$. Z.B. wähle $M \subset (0,\frac{1}{2})^d$, so enthält $[0,1]^d$ abzählbar unendlich viele paarweise disjunkte Translate $q+M$, nämlich für alle $q\in \mathbb{Q}^d \cap (0,\frac{1}{2})^d$ gilt
\begin{equation*}
V:= \bigcup_{q\in(0,\frac{1}{2})^d \cap \mathbb{Q}^d} (q+M) \subset [0,1]^d
\end{equation*}
weil $\vol (V) + \underbrace{\vol ([0,1]^d-V)}_{\geq 0} = \underbrace{\vol ([0,1]^d)}_{=1}$. Daraus folgt $\vol (V) \leq 1 < \infty$ und 
\begin{equation*}
\vol(V) = \sum_{q\in(0,\frac{1}{2})^d \cap \mathbb{Q}^d} \underbrace{\vol (q+M)}_{=\vol(M)}
\end{equation*}
Somit muss gelten $\vol(M)=0$. \Lightning
\end{proof}
\end{satz}
Noch dramatischer: In $\dim \geq 3$ kann man je zwei Teilmengen (unter sehr allgemeinen Annahmen) aus demselben (abzählbaren, oft sogar endlichen) ``Bausatz'' zusammensetzen.
\begin{satz}[Banach-Tarski, 1924] Seien $A,B\subset \R^d$ Teilmengen mit nichtleerem Inneren.
\begin{enumerate}[(\roman*)]
\item Sei $d \geq 3$ und seien $A,B$ beschränkt. Dann existieren endlich viele Teilmengen $M_k \subset \R^d$ und Bewegungen $\Phi_k$ des $\R^d$, so dass \emph{disjunkte Zerlegungen} $A=\bigsqcup_k M_k$ und $B = \bigsqcup_k \Phi(M_k)$ bestehen.
\item Jetzt $d\geq 1$ beliebig und $A,B$ nicht notwendig beschränkt. Dann existieren abzählbar viele Teilmengen $M_k \subset \R^d$ und Bewegungen $\Phi_k$, sodass  \emph{disjunkte Zerlegungen} $A=\bigsqcup_k M_k$ und $B = \bigsqcup_k \Phi(M_k)$ bestehen.
\end{enumerate}
Der Beweis verwendet Gruppentheorie, Struktur von orthogonalen Gruppen $\mathrm{O}(d)$. (nicht mehr auflösbar für $d\geq 3$.)
\end{satz}
Das naive \emph{Inhaltsproblem}, also eine Volumenfunktion mit Eigenschaften (i), (ii) und (iii), ist lösbar in $d\leq 2$, aber nicht eindeutig, nicht lösbar in $d\geq 3$. (Banach 1923, Hausdorff 1914) Dies führt zu:
\vspace{-10pt}
\paragraph{Maßproblem (post-paradox)}: Man definiere eine Volumenfunktion $\vol: \FF \longrightarrow [0,\infty]$ mit Eigenschaften (i'), (ii) und (iii) auf einer möglich großen und flexiblen Familie $\FF \subset \mathcal{P}(\R^d)$, die die geometrisch wichtigen Teilmengen umfasst und abgeschlossen ist unter grundlegenden mengentheoretischen Operationen (Vereinigung, Schnitt, Differenz und Komplement). 
\subsection{Ringe und Algebren}
\marginpar{\tiny{17.10.2019}}
Wir untersuchen Familien von Teilmengen (einer festen Menge), die unter grundlegenden (endlichen) Mengenoperationen abgeschlossen/ stabil sind. ($\cup,\cap,\setminus,\complement$) \\
Sie werden Definitionsbereiche der allgemeinsten von uns betrachteten Volumenfunktion sein. (``Inhalte'')

\subsubsection{Die Ringstruktur auf Potenzmengen}
Sei $X$ eine Menge. Die Potenzmenge ist definiert als die Familie aller Teilmengen $\mathcal{P}(X)$. Wir können die Potenzmenge ebenfalls auffassen als
\begin{equation*}
\mathcal{P}(X) \xleftrightarrow[\text{bij}]{\cong} \{0,1\}^X = \{f:X \longrightarrow \{0,1\}\}
\end{equation*}
da
\begin{align*}
A &\longmapsto \chi_A(x) = \begin{cases}
1, \quad & \text{falls } x\in A\\
0, \quad & \text{sonst}
\end{cases}\\
f^{-1}(1) & \longmapsfrom f
\end{align*}
wobei $\chi_A$ de charakteristische Funktion von $A$ ist. \newline \newline
Wir fassen nun $\{0,1\}$ auf als den Körper mit 2 Elementen (Restklassen modulo 2). So ist $\{0,1\}^X$ ein kommutativer Ring mit Eins (multiplikatives Einselement) (im Sinne der Algebra), sogar eine $\mathbb{F}_2$-Algebra.
\begin{remark} Die Addition und Multiplikation von Funktionen erfolgen punktweise:
\begin{enumerate} [-, itemsep=0pt,topsep=3pt]
\item  $(f+g)(x) := f(x)+g(x)$ 
\item $(fg)(x)= f(x) \cdot g(x)$
\end{enumerate}
und $\{0,1\} = \mathbb{F}_2$ ist ein Körper mit zwei Elementen.
\end{remark}
Die Nullelement ist $f \equiv 0$, also $\chi_\emptyset$ und das Einselement ist $\chi_X (\equiv 1)$. Die Addition von charakteristischen Funktionen entspricht der symmetrischen Differenz $A \triangle B$ und die Multiplikation entspricht dem Duchrschnitt von Mengen. Also
\begin{align*}
\chi_A + \chi_B = A \triangle B \\
\chi_A \cdot \chi_B = A \cap B
\end{align*}
Somit ist $(\mathcal{P}(X), \triangle, \cap) \cong (\mathbb{F}_2^X, +, \cdot)$ ein kommutativer Ring mit dem Nullelement $\emptyset$ bzw. $\chi_\emptyset$ und dem Einselement $X$ bzw. $\chi_X$. 
\subsubsection{Ringe und Algebren}
\begin{definition}
\begin{mdframed}
Eine Familie $\RR\subset \mathcal{P}(X)$ heißt
\begin{enumerate}[itemsep=0pt,topsep=3pt]
\item[($\rho$)] ein \textbf{Ring} auf $X$, falls sie ein Unterring von $(\mathcal{P}(X), \triangle, \cap)$ ist.
\item[($\alpha$)] eine \textbf{Algebra} auf $X$, falls sie außerdem das Einselement enthält, d.h. $X \in \RR$.
\end{enumerate}
\end{mdframed}
\end{definition}
\begin{remark}
\begin{small}
``Algebra'' wird in verschiedenen Bedingungen verwendet, nämlich die Algebra als ein mathematisches Gebiet, eine Algebra als algebraische Struktur im Sinne der Algebra und eine Algebra im Sinne der obigen Definition.
\end{small}
\end{remark}
$(\rho)$ bedeutet $\emptyset \in \RR$, abgeschlossen unter Addition ($\triangle$) (dasselbe wie Subtraktion, da $\mod 2$) und Multiplikation ($\cap$), d.h.
\begin{equation*}
A,B \in \RR \implies A\triangle B, A\cap B \in \RR
\end{equation*} 
d.h. $\triangle$- stabil und $\cap$- stabil. Wir können $\triangle, \cap$ ausdrücken durch $\setminus$ und $\cup$:
\begin{align*}
A \triangle B & = (A \setminus B) \cup (B \setminus A) \\
A \cap B & = A \setminus (A \setminus B)
\end{align*}
und umgekehrt
\begin{align*}
A \setminus B &= (A \triangle B) \cap A \\
A \cup B &= (A \triangle B) \triangle (A \cap B)
\end{align*}
\textit{Bemerkung.} Die letzte Gleichung gilt, da $(A\triangle B)$ und $(A\cap B)$ disjunkt sind. \newline \newline
Daraus folgt die Charakterisierung von Ringen:
\begin{lemma}
\begin{mdframed}
Eine Familie $\RR\subset \mathcal{P}(X)$ ist genau dann ein Ring auf $X$, wenn
\begin{enumerate}[(\roman*), topsep=3pt, itemsep=0pt]
\item $\emptyset \in \RR$,
\item $\setminus$- stabil, d.h. $A,B \in \RR \implies A \setminus B \in \RR$,
\item $\cup$- stabil, d.h. $A, B \in \RR \implies A \cup B \in \RR$.
\end{enumerate}
\end{mdframed}
\end{lemma}
entspricht für Algebren:
\begin{lemma}
\begin{mdframed}
Eine Familie $\AA \subset \mathcal{P}(X)$ ist genau dann eine Algebra auf $X$, wenn
\begin{enumerate}[topsep=3pt, itemsep=0pt]
\item[(i)] $\emptyset \in \AA$,
\item[(iii)] $\cup$- stabil,
\item[(iv)] $\complement$- stabil, d.h. $A \in \AA \implies \complement A := X\setminus A \in \AA$.
\end{enumerate}
\end{mdframed}
\begin{proof}
Sind diese Eigenschaften erfüllt, so implizieren (i + iv), dass
\begin{equation*}
X = \complement \emptyset \in \AA
\end{equation*}
``$\setminus$'' kann ausgedrückt werden durch ``$\cup$'' und ``$\complement$'': Aus
\begin{equation*}
\complement (A \setminus B) = (\complement A) \cup B
\end{equation*}
folgt
\begin{equation*}
A \setminus B = \complement \big( (\complement A) \cup B \big)
\end{equation*}
Also ist $\AA$ ein Ring, und damit $\AA$ eine Algebra. \\
Ist umgekehrt $\AA$ eine Algebra, so gelten (i + iii). Da auch $X \in \AA$, können wir ``$\complement$'' durch ``$\setminus$'' ausdrücken
\begin{equation*}
\complement A = X \setminus A
\end{equation*}
Also gilt auch (iv).
\end{proof}
\end{lemma}

\begin{folgerung}
Ist $\RR$ ein Ring auf $X$ und $A,B \in \RR$, so auch $A\setminus B, A\cap B, B\setminus A$ und $A\cup B \in \RR$. (\textit{Bem.} Alle in $A\cup B$ enthalten.) Ist $\AA$ eine Algebra auf $X$ und $A,B \in \AA$, so ist außerdem auch $\complement (A \cup B) \in \AA$.
\end{folgerung}

\begin{example} \
\begin{enumerate}
\item[(o)] $\{ \emptyset\} \subset \mathcal{P}(X)$ ist ein Ring auf $X$, \newline
$\{\emptyset, X\} \subset \mathcal{P}(X)$ ist die kleinste Algebra auf $X$, $\mathcal{P}(X) \subset \mathcal{P}(X)$ die größte.
\item [(i)] $\{ \emptyset, A \} \subset \mathcal{P}(X)$ ist ein Ring auf $X$ für ein $A \in \mathcal{P}(X)$, \newline
$\{\emptyset, A, \complement A, X\} \subset \mathcal{P}(X) $ ist eine Algebra auf $X$.
\item[(ii)] Die Familie der endlichen (bzw. abzählbaren) Teilmengen von $X$ ist ein Ring. (eine Algebra, nur falls $X$ selbst endlich bzw. abzählbar) \newline
Die Familie der Teilmengen, die endlich (bzw. abzählbar) sind oder endliches (bzw. abzählbares) Komplement haben, ist eine Algebra.
\end{enumerate}
Weitere Beispiele folgen nach der Diskussion vom Erzeugendensystem.
\end{example}
\textit{Beobachtung.} Der Durchschnitt beliebig vieler Ringe (bzw. Algebren) auf einer festen Menge ist wieder ein Ring (bzw. eine Algebra). Zu jeder Menge $\EE \subset \mathcal{P}(X)$ gibt es eine(n) bezüglich mengentheoretischer Inklusion kleinste(n) Ring (bzw. Algebra), der (die) $\EE$ umfasst, nämlich den Durchschnitt aller Ringe (bzw. Algebren), die $\EE$ umfassen.
\begin{definition}[\textbf{Erzeugendensystem}]
\begin{mdframed}
Der von einer Familie $\EE \subset \mathcal{P}(X)$ erzeugte Ring auf $X$ ist der kleinste Ring, der sie enthält. Man nennt $\EE$ ein \emph{Erzeugendensystem} dieses Rings, oder \emph{Erzeuger}. (Analog für Algebren)
\end{mdframed}
Die Algebra eines Erzeugendensystems ist oft die einfachste Art, eine(n) Ring bzw. Algebra zu beschreiben. 
\end{definition}
Ein Ring geht aus einem Erzeugendensystem $\EE \subset \mathcal{P}(X)$ \emph{konstruktiv} durch einen \emph{abzählbaren} (induktiv!) Prozess hervor, ebenso eine Algebra. \vspace{0.5pc}
\newline
\textbf{Ring.} Definiere induktiv eine Folge von Familien $\FF_0 \subset \FF_1 \subset ... \subset \FF_n \subset ... \subset \mathcal{P}(X)$ mit
\begin{align*}
\FF_0 : = &\EE\cup \{\emptyset\}  \\
\FF_n :=  &\{ A\setminus B, A \cup B \mid A,B \in \FF_{n-1} \},\  n \geq 1
\end{align*}
So ist $\bigcup_{n\in \N} \FF_n \subset \mathcal{P}(x)$ $\setminus$- und $\cup$-stabil, also ein Ring.  \vspace{0.5pc} \newline 
\textbf{Algebra.} analog.
\subsubsection{Halbringe}
Hat ein Erzeugendensystem strukturelle Eigenschaft, so ist die Beschreibung des erzeugenden Rings einfach. Eine natürliche auftretende Bedingung ist:
\begin{definition}[\textbf{Halbringe}]
\begin{mdframed}
Eine Familie $\HH\subset \mathcal{P}(X)$ heißt ein \emph{Halbring} auf $X$, falls
\begin{enumerate}[(\roman*), topsep=3pt, itemsep=0pt]
	\item $\emptyset \in \HH$,
	\item $\HH$ ist $\cap$- stabil,
	\item Für $A,B \in \HH$ existieren \emph{disjunkte} Teilmengen $C_1,...,C_n \in \HH$ mit $A\setminus B = C_1 \sqcup ...\sqcup C_n$.
\end{enumerate}
\end{mdframed}
\end{definition}
\begin{remark}
Halbring ist eine Verallgemeinerung des Begriffs Ring, Ringe sind also Halbringe.
\end{remark}
\begin{example} \
\begin{enumerate}
	\item[(o)] $\{ \emptyset\} \subset \mathcal{P}(X)$ ist ein Halbring auf $X$.
	\item[(i)] Die Familie bestehend aus $\emptyset$ und allen (einelementigen) Teilmengen $\{\emptyset\} \cup \{ \{a\} \mid a \in X \}$  ist ein Halbring auf $X$, sie erzeugt den Ring der endlichen Teilmengen von $X$.
\end{enumerate}
Der Grundbaustein für später:
\begin{mdframed}
\begin{enumerate}
	\item[(ii)] Die Familie der \emph{halboffenen} Intervalle $[a,b) \subset \R$, falls $a<b$, also ist $$\II := \{[a,b) \mid a,b \in \R, a<b\}$$  ein Halbring auf $\R$.
\end{enumerate}
\end{mdframed}
\end{example}
Beschreibe den von einem Halbring erzeugenden Ring, die folgende Beobachtung wird darüber hinaus nützlich sein:

\begin{lemma}[Simultane Zerlegung]
\begin{mdframed}
Zu beliebigen Teilmengen $H_1,...,H_m \in \HH$ existieren paarweise disjunkte Teilmengen $H_1',...,H_n' \in \HH$, sodass jedes $H_i$ sich als die Vereinigung einiger $H_j'$'s darstellen lässt.
 \label{lemmaA}
\end{mdframed}
\begin{proof}
Betrachte die $2^m-1$ Durchschnitte der Form $G_1 \cap ... \cap G_m$, wobei $G_i = H_i$ oder $\complement H_i$ und nicht alle gleich $\complement H_i$. Sie sind paarweise disjunkt und zerlegen $H_1 \cup ... \cup H_m$. Jedes $H_i$ ist die Vereinigung von $2^{m-1}$ von ihnen. Es genügt zu zeigen, dass diese Durchschnitte disjunkte Vereinigungen von Teilmengen aus $\HH$ sind. Da Halbringe $\cap$-stabil sind, reicht es zu zeigen, dass die Teilmengen der Form
\begin{equation*}
H \cap \complement \widetilde{H}_l \cap ... \cap \complement \widetilde{H}_1	\quad \text{mit }H,\widetilde{H}_1,...,\widetilde{H}_l \in \HH
\end{equation*}
disjunkte Vereinigungen von Teilmengen in $\HH$ sind. \\
Da für $H \cap \complement \widetilde{H}_l = H \setminus \widetilde{H}_l$ (Axiom (iii)) gilt, reduziert die Behauptung für $l$ auf Behauptung für $l-1$, mit Induktion liefert dann die Behauptung.
\end{proof}
\end{lemma}

\begin{proposition}\marginpar{\tiny{21.10.2019}}
\begin{mdframed}
Jede Teilmenge im von einem Halbring $\HH$ erzeugten Ring $\RR$ ist eine endliche disjunkte Vereinigung von Teilmengen in $\HH$, d.h. \begin{scriptsize}(ein einfacher Erzeugungsprozess!)\end{scriptsize} 
\begin{equation}
\RR = \left\{ \bigsqcup_{k=1}^n H_k \  \Big\rvert \ n\in \N, H_1,...,H_n \in \HH \right\}
\label{eqI14}
\end{equation}
\end{mdframed}
\begin{proof}
Sei $\RR$ die Familie der endlichen \emph{disjunkten} Vereinigungen von Teilmengen in $\HH$. Mit dem letzten \hyperref[lemmaA]{Lemma ``Simultane Zerlegung''} ist $\RR$ gleich der Familie aller endlichen Vereinigungen von Teilmengen in $\HH$. Sie ist offensichtlich $\cup$-stabil. Zu verifizieren bleibt die $\setminus$- Stabilität. Seien hierzu $A = A_1 \cup ... \cup A_m$ und $B = B_1 \cup ... \cup B_n$, $A_i, B_j \in \HH$. Aus dem \hyperref[lemmaA]{Lemma ``Simultane Zerlegung''} folgt, dass es endlich viele nichtleere, \emph{paarweise disjunkte} $H_k' \in \HH$ existieren, sodass jedes $A_i$ und $B_j$ eine Vereinigung einiger $H_k'$'s ist. Daraus folgt, dass auch $A$ und $B$ Vereinigungen einiger $H_k'$'s sind. So ist auch $A\setminus B$ die Vereinigung einiger $H_k'$'s, nämlich derer, die in $A$, aber nicht in $B$ enthalten sind. Also ist $\RR$ ein Ring, enthalten in von $\HH$ erzeugendem Ring\begin{scriptsize}(denn Ringe sind $\cup$-stabil)\end{scriptsize}, also gleich.
\end{proof}
\end{proposition}

\begin{remark}
Man kann den von einer Familie $\EE$ erzeugten Halbring nicht (analog zu Ringen und Algebren) definieren, denn das Halbring-Axiom (iii) vererbt nicht auf Durchschnitte von Familien. Es gibt Durchschnitte von Halbringen, die keine Halbringe sind. M.a.W. existieren Familien, die nicht in einem eindeutigen kleinsten Halbring enthalten sind.
\end{remark}

\subsubsection{Produkte von Halbringen und Ringen}
Sind $\FF_i \subset \mathcal{P}(X_i), i=1,...,n$ Familien von Teilmengen, so entsteht das Produkt von ``Quadern''
\begin{equation*}
\begin{split}
\FF_1 \ast ... \ast \FF_n := \{ \underbrace{M_1 \times ... \times M_n}_{\subset X_1 \times ... \times X_n} \mid M_i \in \FF_i \text{ für } i=1,...,n\}  \\ \subset  \mathcal{P}(X_1 \times ... \times X_n)
\end{split}
\end{equation*}
und die $\cup$-stabile Hülle, die Familie $\FF_1 \boxtimes ... \boxtimes \FF_n$ der endlichen Vereinigungen von ``Quadern'' in $\FF_1 \ast ... \ast \FF_n$, die Figuren, 
\begin{equation*}
\FF_1 \boxtimes ... \boxtimes \FF_n = \left\{\text{endlcihe Vereinigungen von Teilmengen in } \FF_1 \ast ... \ast \FF_n	\right\}
\end{equation*}
Beide Produkte $\ast$ und $\boxtimes$ sind \emph{assoziativ}, d.h.\begin{small}
\begin{equation*}
(\FF_1 \ast \FF_2) \ast \FF_3 = \FF_1 \ast \FF_2 \ast \FF_3 = \FF_1 \ast (\FF_2 \ast \FF_3)
\text{ \scriptsize und }
(\FF_1 \boxtimes \FF_2) \boxtimes \FF_3 = \FF_1 \boxtimes \FF_2 \boxtimes \FF_3 = \FF_1 \boxtimes (\FF_2 \boxtimes \FF_3).
\end{equation*}
\end{small}
Wir definieren weiter
\begin{equation*}
\mathcal{Z} = 
\mathcal{Z} (\FF_1,...,\FF_n) \subset \PP(X_1 \times ...\times X_n) %= \bigcup_{k=1}^n \pi_k^{-1}(M_k)\subset \FF_1 \ast ... \ast \FF_n 
\end{equation*}
die Familie der \textbf{Zylindermengen}
\begin{equation*}
\begin{split}
\pi_k^{-1}(M_k) = X_1 \times ... \times X_{k-1} \times M_k \times X_{k+1} \times ... \times X_n \quad\\ \text{mit } 1 \leq k \leq n, M_k \in \FF_k
\end{split}
\end{equation*}
wobei $\pi_k: X_1 \times ... \times X_n \to X_k, (x_1,...,x_n) \mapsto x_k$ die natürliche Projektion ist. \\
Falls $X_i \in \FF_i \ \forall i$, so ist $\mathcal{Z}(\FF_1,...,\FF_n) \subset \FF_1 \ast ... \ast \FF_n$.%\newpage
\begin{proposition}
\begin{mdframed} \
\begin{enumerate}[(\roman*),topsep=5pt, itemsep= 0pt]
	\item Seien $\HH_i \subset \mathcal{P}(X_i)$ Halbringe $(i=1,...,n)$ und $\RR_i \subset \mathcal{P}(X_i)$ die von ihnen erzeugten Ringe. Dann ist $\HH_1 \ast ... \ast \HH_n$ ein Halbring auf $X_1 \times ... \times X_n$ und $\HH_1 \boxtimes ... \boxtimes \HH_n =\RR_1 \boxtimes ... \boxtimes \RR_n$ der von ihm erzeugte Ring.
	\item Sind $\RR_i \subset \mathcal{P}(X_i)$ Ringe und $\EE_i \subset \RR_i$ Erzeugendensysteme für $i=1,...,n$, so wird der Produktring $\RR_1 \boxtimes ... \boxtimes \RR_n$ von der Familie von Quandern $\EE_1 \ast ... \EE_n$ erzeugt.
	\item Sind die $\RR_i$ Algebren, so wird der Produktring $\RR_1 \boxtimes ... \boxtimes \RR_n$ von den Zylindermengen $\ZZ = \ZZ(\EE_1,...,\EE_n)$ erzeugt.
	\item Sind $\AA_i \subset \PP(X_i)$ $\sigma$-Algebren und $\EE_i \subset \AA_i$ Erzeugendensysteme für $i=1,...,n$, so wird die Produkt-$\sigma$-Algebra von $\ZZ(\EE_1,...,\EE_n)$ sowie von $\EE_1 \ast ... \ast \EE_n$ erzeugt.
	%\item Sind $\EE_i \subset \HH_i$ Erzeugendensysteme der Halbringe, so ist $\mathcal{Z}(\EE_1,...,\EE_n)$ ein Erzeugendensystem von $\HH_1 \ast ... \ast \HH_n$ als Halbring sowie (folglich) ein Erzeugendensystem von $\HH_1 \boxtimes ... \boxtimes \HH_n$ als Ring.
\end{enumerate}
\end{mdframed}
\label{propC}
\begin{proof} \
\begin{enumerate}[(\roman*),topsep=5pt]
	\item Zunächst im Fall $n=2$. Klar enthält $\HH_1 \ast \HH_2$ auch $\emptyset$ und ist $\cap$-stabil. Wir betrachten die disjunkte Zerlegung
	\begin{equation*}
	\begin{split}
		&(A_1 \times A_2) \setminus (B_1 \times B_2) =\\
	 \big( \underbrace{(A_1 \cap B_1)}_{\in \HH_1} \times \underbrace{(A_2 \setminus B_2)}_{\substack{\text{zerlegbar in} \\ \text{Teilmengen}\\ \text{aus }\HH_2}} \big) & \sqcup \underbrace{\big( \underbrace{(A_1 \setminus B_1)}_{\substack{\text{zerlegbar in} \\ \text{Teilmengen}\\ \text{aus }\HH_1}} \times \underbrace{(A_2 \setminus B_2)}_{\substack{\text{zerlegbar in} \\ \text{Teilmengen}\\ \text{aus }\HH_2}} \big)}_{\text{zerlegbar in Teilmengen aus } \HH_1 \times \HH_2}
		\sqcup \underbrace{\big( (A_1 \setminus B_1) \times (A_2 \cap B_2)\big)}_{\text{analog zerlegbar}}
	\end{split}
	\end{equation*}
	Also ist $(A_1 \times A_2) \setminus (B_1 \times B_2) $ disjunkt zerlegbar in Teilmengen aus $\HH_1 \ast \HH_2$, also erfüllt Axiom (iii) für Halbringe, d.h.  $\HH_1 \ast \HH_2$ ist ein Halbring. \\
	Mit Induktion liefert dann die Behauptung auch für $\HH_1 \ast ... \ast \HH_n, n\geq 1$. \newline
	Aus \eqref{eqI14} folgt, dass $\HH_1 \boxtimes ... \boxtimes \HH_n$ der von $\HH_1 \ast ... \ast \HH_n$ erzeugte Ring ist.
	\item Für jedes $i$ wird der Ring auf $X_1 \times ... \times X_n$ bestehend aus den Zylindermengen $\pi_i^{-1}(M_i)$ für $M_i \in \RR_i$ von der Familie der Zylindermengen $\pi_i^{-1}(E_i)$ für $E_i \in \EE_i$ erzeugt, denn jede Teilmenge $M_i \in \RR_i$ kann durch endlich viele Mengenoperationen aus Teilmengen $E_{ij} \in \EE_i$ hergestellt werden und $\pi^{-1}_i(M_i)$ entsprechend aus den $\pi_i^{-1}(E_{ij})$. \\
	Daraus folgt, dass $\mathcal{Z}(\EE_1,...,\EE_n)$ erzeugt denselben Ring wie $\mathcal{Z}( \RR_1,...,\RR_n)$ und denselben wie $\RR \ast ... \ast \RR_n$, also den Produktring $\RR_1 \boxtimes ... \boxtimes \RR_n$. 
	\item Übung.
	%\item Für jedes $k$ gilt: Der Halbring auf $X_1 \times ... \times X_n$ bestehend aus den Zylindermengen $\pi_k^{-1}(H_k)$ für $H_k \in \HH_k$ wird erzeugt von den Zylindermengen $\pi_k^{-1}(E_k)$ für $E_k \in \EE_k$. Da
	%\begin{equation*}
	%\begin{split}
	%& \EE_k \subset \\ 
	%& \underbrace{\{ H_k \in \HH_k \mid \pi^{-1}_k (H_k) \text{ gehört zum von den } \pi^{-1}_k(E_k)   \text{ für } E_k \in \EE_k \text{ erzeugten Halbring} \}}_{\text{ist Halbring, deshalb Gleicheit}}
	% \\ & \subset \HH_k
	%\end{split}
	%\end{equation*}
	%folgt, dass $\mathcal{Z}(\EE_1,...,\EE_k)$ erzeugen denselben Halbring $\mathcal{Z}(\HH_1,...\HH_n)$ und damit denselben wie $\HH_1 \ast ... \ast \HH_n$.
\end{enumerate}
\end{proof}
\end{proposition}
\begin{definition}
\begin{mdframed}
Wir nennen den Halbring $\HH_1 \ast ... \ast \HH_n$ das Produkt der Halbringe $\HH_i$, den Ring $\RR_1 \boxtimes ... \boxtimes \RR_n$ das Produkt der Ringe $\RR_i$.
\end{mdframed}
\end{definition}

\begin{hauptbsp}[Quader und Figuren in $\R^d$]
\begin{mdframed}
Ist $a,b \in \R^d$ mit $a_i < b_i \forall i$, so entsteht ein achsenparalleler halboffener Quader
\begin{equation*}
[a,b) := [a_1, b_1) \times ... \times [a_d,b_d)
\end{equation*}
Wir bezeichnen
\begin{equation*}
\QQ^d := \text{Familie dieser Quader}
\end{equation*}
und
\begin{equation*}
\mathcal{I} := \QQ^1, \text{Familie der halboffenen Intervalle}
\end{equation*}
Also gilt
\begin{equation*}
\QQ^d = \underbrace{\mathcal{I} \ast ... \ast \mathcal{I}}_{d\text{-Mal}}
\end{equation*}
Aus der \hyperref[propC]{letzten Proposition} folgt, dass $\QQ^d$ ein Halbring auf $\R^d$ ist. Es folgt ebenfalls, dass
\begin{equation*}
\FF^d := \underbrace{\mathcal{I} \boxtimes ... \boxtimes \mathcal{I}}_{d\text{-Mal}}
\end{equation*}
der Ring erzeugt von $\QQ^d$ ist, also der Ring der $d$-dimensionale ``Figuren''.
Wir haben gesehen: Figuren sind disjunkte Vereinigungen von Quadern.
\end{mdframed}
\end{hauptbsp}

Wir arbeiten aus technischen Gründen mit halboffenen Intervallen und Quadern. Besonders übersichtliche sind Halbringe, die abgeschlossen unter Produktbildung sind,  jedoch nicht die Teilmengen enthalten, die uns geometrisch primär interessieren: die offenen und abgeschlossenen Quader, die nicht achsenparallele sind, Polygone und Polytope, gekrümmte ``elementare Geometrie'' sowie Gebilde: Scheiben, Bälle, Zylinder und Kegel. Deshalb müssen wir unsere Ringe weiter anreichern und flexibilisieren, damit sie stabil unter abzählbaren Vereinigungen sind. $\leadsto$ $\sigma$-Algebra.

\subsection{Inhalte und Prämaße}
Wir beginnen mit der Untersuchung von Volumenfunktionen. Die grundlegende Forderung ist die \emph{Additivität}. Volumina dürfen nicht negativ sein, d.h. $\in [0,\infty] := [0, \infty) \cup \{ \infty\}$, also die erweiterte positive Halbgerade. 
\begin{remark}
Die erweiterten reellen Zahlen ist definiert als $\ol{\R} := \{- \infty\} \cup \R \cup \{ \infty \}$ mit natürlichen Konventionen
\begin{align*}
	x + \infty = \infty & \text{ für } x > - \infty	\\
	x \cdot \infty = \infty &\text{ für } x> 0
\end{align*}
Später werden wir außerdem sehen 
%\begin{equation*}
	$0 \cdot \infty = 0$
%\end{equation*}
(da $ 0 \cdot n \longrightarrow 0$ für $n \longrightarrow \infty$).
\end{remark}
\subsubsection{Inhalte auf Halbringen und Ringen}
Die allgemeinste Sorte von uns betrachteter Volumenfunktion ist endlich additiv und definiert auf Halbringen.

\begin{definition}[\textbf{Inhalt}]
\begin{mdframed}
Ein Inhalt auf einer Halbring $\HH$ ist eine Funktion $\mu: \HH \to [0,\infty]$ mit den Eigenschaften:
\begin{enumerate}[(\roman*),topsep=5pt, itemsep = 0 pt]
	\item $\mu (\emptyset) = 0$,
	\item \emph{Additivität}: Sind $A_1,...,A_n \in \HH$ paarweise disjunkt mit $A_1 \sqcup ... \sqcup A_n \in \HH$ \tablefootnote{Die Voraussetzung ist redundant, falls $\HH$ ein Ring ist.}, so gilt $\mu(A_1 \sqcup ... \sqcup A_n) = \mu(A_1) + ... +\mu(A_n)$.
\end{enumerate}
\end{mdframed}
\end{definition}

\begin{example} \
\begin{enumerate}
	\item[(o)] $\mu \equiv 0$ ``der Nullinhalt'' und\\
				$\nu(A) = \begin{cases}
					0,       & A = \emptyset \\
					\infty, & \text{sonst}
				\end{cases}$
				sind stets Inhalte auf beliebigen Halbringen.
	\item[(i)] Sei $X$ nichtleer. Betrachte die Algebra $\{ \emptyset, X\} \subset \mathcal{P}(X)$. So wird ein Inhalt \\
			$\begin{cases}
				\emptyset \mapsto 0 \\
				x \mapsto v \in [0,\infty] \text{ beliebig}
			\end{cases}$
			definiert.
\end{enumerate}
\end{example}

\begin{example}[Halboffene Intervalle in $\R$, der Grundbaustein für Lebesgue-Maß]
\begin{mdframed}
Auf dem Halbring $\II = \mathcal{Q}^1 \subset \mathcal{P}(\R)$ wird ein Inhalt gegeben durch die euklidische Länge
\begin{align*} %\label{I23}
\lambda^1_\II : \II \to [0, \infty), \quad \lambda^1_\II \big([a,b)\big) := b-a \ (a<b)
\end{align*}
\end{mdframed}
Wir überprüfen die Additivität: Sei $a = x_0 < x_1 < ... < x_{n} = b$ eine Unterteilung von $[a,b)$. So entsteht die disjunkte Zerlegung $[a,b) = [a,x_1) \sqcup ... \sqcup [x_{n-1},b)$. Es folgt 
\begin{equation*}
\underbrace{\lambda^1_\II \big([a,b)]}_{b-a} = \underbrace{\lambda^1_\II \big( [a,x_1)\big)}_{x_1-a} + \underbrace{\lambda^1_\II \big([x_1,x_2)\big)}_{x_2 -x_1} + ... + \underbrace{\lambda^1_\II \big( [x_{n-1},b ) \big)}_{b-x_{n-1}}
\end{equation*}
\end{example}
\marginpar{\tiny{24.10.2019}}
\begin{lemma}[Einfache Eigenschaften von Inhalten]
\begin{mdframed}
Seien $\HH$ ein Halbring und $\mu:\HH \longrightarrow [0,\infty]$ ein Inhalt. Dann gilt:
\begin{enumerate}[(\roman*),topsep=5pt, itemsep = 0 pt]
	\item \emph{Monotonie}: Ist $A, B \in \HH$ mit $A \subset B$, so ist $\mu (A) \leq \mu (B)$,
	\item \emph{Subadditivität:} Seien $A_1, ..., A_n \in \HH$ \emph{(nicht notwendigerweise disjunkt!)} mit $A_1 \cup ... \cup A_n \in \HH$, dann gilt
	\begin{equation*}
	\mu(A_1 \cup .... \cup A_n) \leq \mu (A_1) + ... + \mu (A_n).
	\end{equation*}
\end{enumerate}
\end{mdframed}
\begin{proof} \
\begin{enumerate}[(\roman*),topsep=5pt, itemsep = 0 pt]
\item Setze $B \setminus A = C_1 \sqcup ... \sqcup C_n$ mit $C_i \in \HH$ paarweise disjunkt, bzw. $B= A \sqcup C_1 \sqcup ... \sqcup C_n$. Aus der Additivität folgt dann $\mu(B) = \mu (A) + \underbrace{\mu(C_1) + ... + \mu (C_n)}_{\geq 0} \geq \mu (A)$.
\item Aus dem \hyperref[lemmaA]{Lemma ``simultane Zerlegung''} folgt, dass es paarweise disjunkte $H_i \in \HH$ existieren, sodass jedes $A_j$ die Vereinigung einiger von $H_i$ ist. Entsprechend summieren sich die Volumina auf. Die Ungleichung folgt, denn jedes $\mu(H_j)$ \emph{genau einmal} auf der linken Seite und je \emph{mindestens einmal} auf der rechten Seite.
\end{enumerate}
\end{proof}
\end{lemma}

\subsubsection{Fortsetzung von Inhalten von Halbringen auf Ringe}
\begin{satz}
\begin{mdframed}
Jeder Inhalt $\mu$ auf einem Halbring $\HH$ besitzt eine eindeutige Fortsetzung zu einem Inhalt $\ol{\mu}$ auf dem von $\HH$ erzeugten Ring $\RR$.
\end{mdframed}
\begin{proof} \
\begin{enumerate}[-,topsep=5pt, itemsep = 0 pt]
	\item \emph{Eindeutigkeit} folgt aus der Additivität von Inhalten und Beschreibung des erzeugten Rings $\RR$. Jede Teilmenge in $\RR$ ist eine disjunkte Vereinigung (wegen \eqref{eqI14}) $A_1 \sqcup ... \sqcup A_n$ mit $A_i \in \HH$. Daher notwendig
	\begin{equation} \label{eqI25}
	\ol{\mu} (A_1 \sqcup ... \sqcup A_n) = \underbrace{\mu (A_1)}_{= \ol\mu(A_1)} + ... + \ol\mu (A_n)
	\end{equation}
	\item \emph{Existenz} bzw. \emph{Wohldefiniertheit} von $\ol\mu$ durch \eqref{eqI25}: Wir betrachten eine weitere disjunkte Zerlegung derselben Teilmengen
	\begin{equation*}
	A_1 \sqcup ... \sqcup A_n = B_1 \sqcup ... \sqcup B_m \in \RR, A_i, B_j \in \HH
	\end{equation*}
	so entstehen Zerlegungen 
	\begin{align*}
		A_i = \bigsqcup_{j=1}^m (A_i \cap B_j)	\\
		B_j = \bigsqcup_{i=1}^n (A_i \cap B_j)
	\end{align*}
	Daraus folgt \emph{(Wir bemerken, dass $A_i \cap B_j \in \HH$, denn $\HH$ $\cap$-stabil ist.)}
	\begin{equation*}
	\sum_i \mu (A_i ) = \sum_i \underbrace{\sum_j \mu (A_i \cap B_j)}_{\mu (A_i)}
	= \sum_j \underbrace{\sum_i \mu (A_i \cap B_j)}_{\mu (B_j)} = \sum_j \mu (B_j)
	\end{equation*}
	Also ist $\ol\mu$ wohldefiniert.
	\item Es bleibt zu zeigen, dass $\ol\mu$ tatsächlich ein Inhalt ist. $\ol\mu$ ist laut der Definition \eqref{eqI25} offensichtlich additiv und somit ein Inhalt.
\end{enumerate}
\end{proof}
\begin{remark}
Hat $\mu$ endliche Werte, so hat $\ol\mu$ auch endliche Werte.
\end{remark}
\end{satz}

\begin{example}
\begin{mdframed}
Wir setzen den auf dem Halbring $\mathcal{I}$ definierten Inhalt
$\lambda^1_{\QQ^1}: \underset{\text{\tiny{Halbring}}}{\mathcal{I} = \QQ^1} \to [0,\infty)$  fort zu
\begin{equation*}
\lambda^1_{\FF^1}:\FF^1 \longrightarrow [0,\infty)
\end{equation*}
wobei $\FF^1$ den von $\QQ^1$ erzeugten Ring der 1-dimensionale Figuren bezeichnen, also ist $\lambda^1_{\FF}$ definiert als die Summe der Längen der Teilintervalle.
\end{mdframed}
\end{example}

\begin{remark}
Die Fortsetzung von Inhalten von Ringen auf Algebren ist nicht eindeutig. Z.B. betrachten wir die vom Ring $\RR=\{ \emptyset\}$ erzeugte Algebra $\AA = \{ \emptyset, X\}$, so können wir den Inhalt von $X$ beliebig $\in [0,\infty)$ wählen.
\end{remark}
\subsubsection{Prämaße}
Wir betrachten Verhalten von Volumina bei gewissen Grenzprozessen. (Approximation von innen und außen) Wir arbeiten mit Teilmengen einer festen Menge $X$.\newline \newline
Falls $(A_n)_{n \in \N}$ eine aufsteigende Folge von Teilmengen von $X$ mit
\begin{equation*}
\bigcup_{n \in \N} A_n =A, \ \ A_1 \subset A_2 \subset ... \subset A_n \subset ... \subset X
\end{equation*}  so schreiben wir $A_n \nearrow A$. \newline \newline
Falls $(A'_n)_{n \in \N}$ absteigend mit 
\begin{equation*}
\bigcap_{n \in \N} A'_n = A, \ \ X \supset A_1' \supset A_2' \supset ... \supset A_n' \supset ...
\end{equation*}
so schreiben wir $A_n' \searrow A$.
\begin{beobachtung}
Es gelten
\begin{align*}
	A_n \nearrow A \iff A \setminus A_n \searrow \emptyset	\\
	A_n' \searrow A \iff A_n'	\setminus A \searrow \emptyset
\end{align*}
Sei $\mu: \RR \longrightarrow [0,\infty]$ ein Inhalt. Dann gilt
\begin{equation*}
A_n \nearrow A \swarrow A_n' \xRightarrow[]{\text{Monotonie}} \underset{\text{wächst}}{\mu(A_n)} \leq \mu(A) \leq \underset{\text{fällt}}{\mu(A_n')}
\end{equation*}
Da $\mu(A_n)$ und $\mu(A'_n)$ nur \emph{schwach} monoton sind, folgt nur die Ungleichung
\begin{equation}	\label{I33}
\lim\limits_{n\to\infty} \mu(A_n) \leq \mu(A) \leq \lim\limits_{n \to \infty} \mu (A_n')
\end{equation}
\end{beobachtung}
Wir formulieren nun einen disjunkte Zerlegung für $A_n$ durch einen Induktiven Prozess: (Man beachte, dass $(A_n)_{n\in \N}$ eine aufsteigende Folge ist.)
\begin{equation*}\begin{split}
	A_0 & := \emptyset	\\
	\widetilde{A}_{n} &:= {A}_{n} \setminus A_{n-1} \end{split}
\end{equation*}
so entsteht die disjunkte Zerlegung
\begin{equation*}
	A = \bigsqcup_{n\in \N} \widetilde{A}_n
\end{equation*}
Dann ist \eqref{I33} äquivalent zu: Für Folgen $(\widetilde{A}_n)_{n \in \N}$ paarweise disjunkter Teilmengen mit $A:=\bigsqcup_{n\in\N}\widetilde{A}_n \in \RR$ gilt
\begin{equation*}
\mu\bigg( \underbrace{\bigsqcup_{n\in \N} \widetilde{A}_n}_{=A} \bigg) \geq \sum_{n=1}^\infty \mu(\widetilde{A}_n)
\tag{$\sigma$-Supadditivität}
\end{equation*}
Gilt in einer der Gleichung \eqref{I33} die Gleichheit, so fassen wir das als \textbf{Stetigkeitseigenschaften} auf. Wir vergleichen nun die Stetigkeitseigenschaften:

\begin{proposition}
\begin{mdframed}
Für einen Inhalt $\mu : \RR \longrightarrow [0,\infty]$ auf einem Ring $\RR$ sind die beiden folgenden Eigenschaften äquivalent:
\begin{enumerate}[(\roman*),topsep=5pt, itemsep = 0 pt]
	\item \emph{$\sigma$-Additivität:} Ist $(A_n)_{n \in \N}$ eine Folge paarweise disjunkter Teilmengen in $\RR$ mit $\bigsqcup_nA_n \in \RR$, so gilt
	$$
	\mu \bigg(\bigsqcup_{n=1}^\infty A_n \bigg) = \sum_{n=1}^\infty \mu (A_n).
	$$
	\item \textit{Stetigkeit von unten:} Ist $(B_n)_{n\in\N}$ eine aufsteigende Folge in $\RR$, so gilt %$B_n \in \RR$ mit $B_n \nearrow B \in \RR$, so gilt
	$$
	B_n \nearrow B \in \RR \implies \mu (B_n) \nearrow \mu (B).
	$$
\end{enumerate}
	Sie implizieren die beiden folgenden, ebenfalls zueinander äquivalenten, Eigenschaften:
	\begin{enumerate}[(\roman*),topsep=5pt, itemsep = 0 pt]
	\item[(iii)] \emph{Stetigkeit von oben:} Ist $(C_n)_{n \in \N}$ eine absteigende Folge in $\RR$ \textbf{mit} $\mu(C_n) < \infty$, so gilt
	$$C_n \searrow C \in \RR \implies \mu (C_n) \searrow \mu (C).$$
	%Ist $(C_n)_{n \in \N}$ absteigend, $C_n \in \RR$ \textbf{mit} $\mu(C_n) < \infty$ {und} $C_n \searrow C \in \RR$, so gilt
	%$
	%\mu(C_n) \searrow \mu (C)	
	%$
	\item[(iv)] \emph{Stetigkeit von $\emptyset$:} Ist $(D_n)_{n \in \N}$ eine absteigende Folge in $\RR$ \textbf{mit} $\mu(D_n)<\infty$, so gilt
	$$D_n \searrow \emptyset \implies \mu (D_n) \searrow 0.$$
	%Ist $(D_n)_{n \in \N}$ absteigend, $D_n \in \RR$ \textbf{mit} $\mu(D_n) < \infty$ und $D_n \searrow \emptyset$, so gilt
	%$ \mu (D_n) \searrow 0$
	\end{enumerate}
	Falls $\mu$ endliche Werte hat, gilt umgekehrt (iii), (iv) $\implies$ (i), (ii).
\end{mdframed}
\begin{proof}\
\begin{enumerate}[(\roman*),topsep=5pt, itemsep = 0 pt]
	\item[-] (i) $\iff$ (ii). Übergang durch $B_n = A_1 \sqcup ... \sqcup A_n$, bzw. $A_n = B_n \setminus B_{n-1}$, $B = \bigsqcup_n A_n$. Aus der endlichen Additivität folgt $\mu (B_n) = \sum_{i=1}^n \mu(A_i)$. Daraus folgt $\lim\limits_{n \to \infty} \mu (B_n) = \sum_{n \in \N} \mu (A_n)$. Außerdem gilt $\mu(B) = \mu (\bigsqcup_n A_n)$. \checkmark
	\item[-] (ii) $\implies$ (iii). Sei $C_n \searrow C$ mit endlichen Inhalten. Setze $B_n := C_1 \setminus C_n \in \RR$ und $B := C_1 \setminus C$, d.h. $C_1 = B_n \sqcup C_n$ und $C_1 = B \sqcup C$. Daraus folgt wegen der Additivität des Inhalts $\mu(C_1) = \mu (B_n) + \mu(C_n)=\mu(B) + \mu (C)$. Es gilt n.V. $\mu(B_n) \nearrow \mu (B)$, da $B_n \nearrow B$. Da alle Inhalte endlich sind, gilt
		$\mu (B_n) = \mu(C_1) - \mu (C_n)$ und
		$\mu (B) = \mu(C_1) - \mu (C)$. Also gilt $\mu(C_1) - \mu (C_n) \nearrow \mu(C_1) - \mu (C)$.
	Daraus folgt, dass $\mu(C_n) \searrow \mu(C)$, also (iii).
	\item[-] (iii) $\Longleftarrow$ (iv).  (Die andere Richtung ist klar, da (iv) ist Spezialfall von (iii)!) \newline
	Sei $C_n \searrow C$ mit endlichen Inhalten. Setze $D_n := \underbrace{C_n \setminus C}_{\in \RR} \searrow \emptyset$. Dann ist $C_n = D_n \sqcup C$ und somit $\mu(C_n) = \mu(D_n) + \mu(C)$. Da alle Inhalte endlich sind, gilt $\underbrace{\mu(D_n)}_{\searrow 0 \text{ wegen (iv)}} = \mu(C_n) -\mu (C)$. Daraus folgt $\mu(C_n) \searrow \mu (C)$, also (iii).
	\item[-] $\mu$ habe endliche Werte, es gelte (iv). Zeige (ii). Sei $B_n \nearrow B$. Setze $D_n:=B \setminus B_n$. Da $\underbrace{\mu(D_n)}_{\searrow 0 \text{ wg (iv)}} = \mu(B) - \mu (B_n)$ und alle Inhalte endlich sind, gilt $\mu(B_n) \nearrow \mu(B)$, d.h. (ii).
\end{enumerate}
\end{proof}
\end{proposition}

\marginpar{\tiny{28.10.2019}}

Es ist natürlich und geboten, von Inhalten $\sigma$-Additivität zu verlangen. $\leadsto$ leistungsfähige Volumentheorie. (E. Borel, Lebesgue $\sim$1900)

\begin{definition}
\begin{mdframed}
Ein \textbf{Prämaß} auf einem \emph{Ring} ist ein $\sigma$-additiver Inhalt.
\end{mdframed}
\end{definition}

\begin{example}
Sei $X$ unendlich. Betrachte $\AA \subset \mathcal{P}(X)$ die Algebra erzeugt von endlichen Teilmengen, besteht aus endlichen Teilmengen und deren Komplementen. Definieren Inhalt $\mu$ auf $\AA$ durch
$$ \mu (A) := \begin{cases}
	0, & \text{falls } A \text{ endlich} \\
	1, & \text{falls } \complement A \text{ endlich}
	\end{cases}$$
Sind $A_1,...,A_n$ paarweise disjunkt und nicht alle $A_i$ endlich, so gibt es genau ein $A_i$ unendlich. Daraus folgt
$$\underset{\text{genau ein Beitrag }1}{\mu(A_1)+...\mu(A_n)} = 1 = \mu(A_1 \sqcup ... \sqcup A_n).$$
Also war die endliche Additivität. Die $\sigma$-Additivität wird genau dann verletzt, falls $X$ abzählbare Zerlegung in endlichen Teilmengen ($\iff$ $X$ abzählbar) zulässt.\footnote{Zum Beispiel betrachten wir die natürlichen Zahlen $\N$. Es gilt $\sum\limits_{n\in \N} \mu (\{n\}) =0 \neq 1 = \mu \left( \bigcup\limits_{n \in \N} \{n\} \right)$.} Daraus folgt: $\mu $ Prämaß $\iff$ $X$ überabzählbar (z.B. $X = \R$).
\end{example}

\begin{satz}
\begin{mdframed}
Der Inhalt $\lambda^1_{\FF^1}$ ist ein Prämaß.
\end{mdframed}
(Später, nach Produkten, Beweis in beliebiger Dimension.)
\begin{proof}
Zu zeigen ist die $\sigma$-Additivität. Dies ist wegen endlicher Werte von $\lambda^1_{\FF^1}$ äquivalent zu Stetigkeit in $\emptyset$. Diese weisen wir jetzt nach. Das Argument beruht auf \emph{topologischen} Eigenschaften von $\R$, nämlich Lokalkompaktheit. Wir betrachten eine Folge absteigender Figuren (d.h. \emph{endliche} Vereinigung halboffener Intervalle $[a,b)$), nämlich
$$F_1 \supset F_2 \supset ... \supset ... \supset F_n \supset ...  \quad  \text{mit } F_n \in \FF^1.$$
Stetigkeit in $\emptyset$ bedeutet: $\underbrace{F_n \searrow \emptyset}_{\iff \bigcap_n F_n = \emptyset} \implies \lambda^1_{\FF^1}(F_n) \searrow 0$. Wir notieren $\lambda = \lambda^1_{\FF^1}$.
Annahme: $\lim\limits_{n \to \infty} \lambda(F_n) \geq v_0 >0$. Zu zeigen: $\bigcap_n F_n \neq \emptyset$.\\
Dazu approximieren wir von innen durch eine geschachtelte Folge von Kompakta $K_n$.
Sei $\varepsilon_n \searrow 0$. Dann existieren $F_n' \in \FF^1$ und Kompakta $K_n \subset \R$ sodass $F_n' \subset K_n \subset F_n$ \begin{scriptsize} (\textit{Bem.} $K_n$ sind nicht notwendigweise geschachtelt. Wir brauchen $F_n'$, denn $\mu$ ist nicht auf Kompakta definiert.) \end{scriptsize}und $\lambda(F_n') > \lambda(F_n) -\varepsilon_n$. Wir vergleichen absteigende Folgen
$$ \bigcap_{i\leq n} F_i' \subset \bigcap_{i \leq n} K_i \subset \underbrace{\bigcap_{i \leq n} F_i}_{= F_n}.$$
Es genügt zu zeigen, dass
\begin{equation}\label{I5}
\bigcap_{i \leq n} K_i \neq \emptyset \  \ \forall n,
\end{equation}
denn \eqref{I5} $ \xRightarrow{\text{Topologie}} \bigcap_{i=1}^\infty K_i \neq \emptyset \implies \bigcap_{i=1}^n F_i \neq \emptyset$. \newline
Es gilt
$$
\underbrace{\left( \bigcap_{i \leq n} F_i \right)}_{F_n} \setminus \left( \bigcap_{i \leq n} F_i' \right) = \bigcup_{i \leq n} (\underbrace{F_n}_{\subset F_i} \setminus F_i') \subset \bigcup_{i \leq n}(F_i \setminus F_i'),
$$
also
$$F_n = {\bigcap_{i\leq n} F_i} \subset \left( \bigcap_{i \leq n} F_i' \right) \cup \bigcup_{i \leq n} (F_i \setminus F_i'),$$
d.h. Überdeckung von $F_n$ durch $n+1$ Teilmengen aus $\FF^1$. Aus der $\sigma$-Subaddtivität folgt dann
$$
\lambda (F_n) \leq \lambda \left( \bigcap_{i\leq n} F_i' \right) + \sum_{i \leq n} \underbrace{\lambda (F_i \setminus F_i')}_{< \varepsilon_i}
$$
Da $\lambda$ endlich-wertig ist, können wir die Gleichung umstellen zu
$$
\lambda\left( \bigcap_{i \leq n} F_i' \right) \geq \underbrace{\lambda(F_n)}_{\geq v_0} - \sum_{i \leq n} \varepsilon_i > v_0 - \sum_{n \in \N} \varepsilon_n
$$
Wähle $(\varepsilon_n)$ so, dass $\sum_{n\in \N} \varepsilon_n < v_0$. Dann folgt
$$\lambda \left( \bigcap_{i \leq n } F_i' \right) > 0 \implies \bigcap_{i \leq n} F_i' \neq \emptyset \implies \bigcap_{i \leq n }K_i \neq \emptyset $$
Also haben wir \eqref{I5} gezeigt und somit war die Behauptung.
\end{proof}
\end{satz}

\begin{definition}
\begin{mdframed}
$\lambda^1_{\FF^1}$ heißt \textbf{das eindimensionale Lebesgue-Prämaß}.
\end{mdframed}
\end{definition}

\subsubsection{Produkte von Inhalten und Prämaßen}
Man kann in natürlicher Weise Produkte von Inhalten auf Halbringen und Ringen bilden. Zunächst Halbringe:
\begin{satz}
\begin{mdframed}
Seien $\mu_i$ Inhalte auf Halbringen $\HH_i \subset \mathcal{P}(X_i), i=1,...,n$. Dann wird durch
$$
(\mu_1 \times ... \times \mu_n) \big(H_1 \times ... \times H_n \big) := \mu(H_1) \cdot ... \cdot \mu(H_n), H_i \in \HH_i
$$
ein Inhalt $\mu_1 \times ... \times \mu_n$ auf dem Produkthalbring $\HH_1 \ast ... \ast \HH_n \subset \mathcal{P}(X_1 \times ... \times X_n)$ definiert. \\
Zusätzliche \emph{Konvention} für Multiplikation auf $\R$: $0 \cdot \infty = 0$, weil $ 0 \cdot n \xrightarrow{n \to \infty} 0$. 
\end{mdframed}
\begin{proof} (Für $n=2$, der allgemeiner Fall folgt mit Induktion.) \\
Wir müssen die \emph{Additivität} nachweisen. Betrachte die endliche Zerlegung eines ``Rechtecks'' $H_1 \times H_2 \in \HH_1 \ast \HH_2$ in endlich viele paarweise disjunkte ``Rechtecken'' $H_1^{(j)} \times H_2^{(j)} \in \HH_1 \ast \HH_2$:
$$
H_1 \cdot H_2 = \bigsqcup_{j} (H_1^{(j)} \times H_2^{(j)}).
$$
Aus dem \hyperref[lemmaA]{Lemma ``simultane Zerlegung''} folgt, dass wir $H_i$ in $H_i'^{(k)} \in \HH_i, k \in \underset{\text{endlich!}}{K_i}$ zerlegen können, sodass jedes $H_i^{(j)}$  die disjunkte Vereinigung einiger $H_i'^{(k)}$ ist, also
$$
H_i^{(j)} = \bigsqcup_{k \in K_i^{(j)}} H_i'^{(k)}, K_i^{(j)}\subset K_i.
$$
So entsteht die ``karierte'' Zerlegung
$$
H_1 \times H_2 = \bigsqcup_{(k_1,k_2)\in K_1\times K_2} \Big (H_1'^{(k_1)} \times H_2'^{(k_2)} \Big)	
$$
bzw.
$$
H_1^{(j)} \times H_2^{(j)} = \bigsqcup_{(k_1,k_2)\in K_1^{(j)}\times K_2^{(j)}} \Big (H_1'^{(k_1)} \times H_2'^{(k_2)} \Big).
$$
Es gilt die Zerlegung der Indexmenge
$$
K_1 \times K_2 = \bigsqcup_j \Big( K_1^{(j)} \times K_2^{(j)} \Big).
$$
Da $\mu_i$ additiv ist, folgt
$$
\mu_i (H_i) = \sum_{k \in K_i} \mu_i (H_i'^{(k)})
$$
bzw. 
$$
\mu_i (H_i^{(j)}) = \sum_{k \in K_i^{(j)}} \mu_i (H_i'^{(k)}).
$$
Es folgt \begin{scriptsize}
(Die Additivität von $\mu_1 \times \mu_2$ ist klar für ``karierte'' Zerlegung)
\end{scriptsize}
\begin{equation*}
	\begin{split}
	(\mu_1 \times \mu_2) \big( H_1 \times H_2 \big) & = \sum_{(k_1,k_2) \in K_1 \times K_2} (\mu_1 \times \mu_2) \big(H_1'^{(k_1)} \times H_2'^{(k_2)} \big) \\
	& = \sum_j \underbrace{ \sum_{(k_1,k_2) \in K_1^{(j)} \times K_2^{(j)}} (\mu_1 \times \mu_2) \big(H_1'^{(k_1)} \times H_2'^{(k_2)} \big)}_{= \mu_1 \times \mu_2 (H_1^{(j)} \times H_2 ^{(j)})}.
	\end{split}
\end{equation*}
Also war die Additivität von $\mu_1 \times \mu_2$.
\end{proof}
\end{satz}

\begin{definition}
\begin{mdframed}
Wir nennen den Inhalt $\mu_1 \times ... \times \mu_n$ auf dem Halbring $\HH_1 \ast ... \ast \HH_n$ das Produkt der Inhalte $\mu_i$.
\end{mdframed}
\end{definition}

\begin{hauptbsp}[\textbf{Elementarinhalt}]
Der Elementarinhalt achsenparalleler Quader $[a,b) \subset \R^d$ für $a,b \in \R^d$ mit $a_i < b_i$ ist definiert als ihr $d$-dimensionales euklidisches Volumen
$$
\lambda^d_{\QQ^d}\big([a,b)\big) := (b_1 -a_1) \cdot ... \cdot (b_d-a_d)
$$
Der Elementarinhalt $\lambda^d_{\QQ^d}$ ist ein Inhalt, denn
$$
\lambda^d_{\QQ^d} = \underbrace{\lambda^1_\mathcal{I} \times ... \times \lambda^1_\mathcal{I}}_{d\text{-mal}}
$$
bezüglich der Zerlegung $\QQ^d = \mathcal{I} \times ... \times \mathcal{I}$.
\end{hauptbsp}

Für Ringe folgt mit dem Fortsetzungsresultat für Inhalte von Halbringen auf (die von den erzeugte) Ringe:

\begin{korollar}
Seien $\mu_i$ Inhalte auf Ringen $\RR_i \subset \mathcal{P}(X_i)$, $(i=1,...,n)$. Dann existiert einen eindeutigen Inhalt $\mu_1 \boxtimes .... \boxtimes \mu_n$ auf dem Produktring $\RR_1 \boxtimes ... \boxtimes \RR_n \subset \mathcal{P}(X_1 \times ... \times X_n)$ mit
$$(\mu_1 \boxtimes ... \boxtimes \mu_n) \big( R_1 \times ... R_n\big) = \mu_1 (R_1) \cdot ... \mu_n(R_n), R_i \in \RR_i$$
\end{korollar}

\begin{hauptbsp}
Der Elementarinhalt $\lambda^d_{\QQ^d}: \QQ^d \longrightarrow [0,\infty)$ setzt sich eindeutig fort zu einem Inhalt
$$
\lambda^d_{\FF^d}: \FF^d \longrightarrow [0,\infty)
$$
auf $d$-dimensionalen Figuren, den wir als \emph{$d$-dimensionales euklidisches Volumen} auffassen.
\end{hauptbsp}

Auch Prämaße verhalten sich gut unter Produkten:

\begin{satz}
\begin{mdframed}
Endliche Produkte von Prämaßen sind wieder Prämaße.
\end{mdframed}
\begin{proof}
($n=2$, der allgemeiner Fall folgt mit Induktion.)   \newline
Seien $\mu_i$ Prämaße auf Ringen $\RR_i \subset \mathcal{P}(X_i)$, $i=1,2$. und der Produktinhalt $\mu_1 \boxtimes \mu_2$ auf $\RR_1 \boxtimes \RR_2$ gegeben durch
$$(\mu_1 \boxtimes \mu_2) (R_1 \times R_2) = \mu_1 (R_1) \cdot \mu_2(R_2), R_i \in \RR_i.$$ Jetzt müssen wir die Additivität des Volumens nachweisen bei \emph{abzählbaren} Zerlegungen von Figuren (d.h. Teilmengen aus $\RR_1 \boxtimes \RR_2$) in Figuren. Weil Figuren endliche disjunkte Vereinigungen von Quadern sind und der Inhalt $\mu_1 \boxtimes \mu_2$ (endlich) additiv ist, genügt es, abzählbare disjunkte Zerlegungen
$$
A_1 \times A_2 = \bigsqcup_{\substack{m\in \N \\ \text{\tiny abzählbar!}}} (A_{1,m} \times A_{2,m})
$$
mit $A_i, A_{i,m} \in \RR_i$ zu betrachten. Zu zeigen ist also
$$
\sum_{m\in \N} \underbrace{\mu_1(A_{1,m}) \cdot \mu_2(A_{2,m})}_{(\mu_1 \boxtimes \mu_2) (A_{1,m}\times A_{2,m})} = \mu_1 (A_1) \cdot \mu_2(A_2)
$$
Wegen der Monotonie von Inhalten gilt die Richtung ``$\leq$'', zu zeigen ist ``$\geq$''.
\marginpar{\tiny{31.10.2019}}
Hierbei wird verwendet, dass $\mu_i$ Prämaße sind. Wir brauchen eine \emph{untere} Abschätzung von der Summe der Teilvolumina. Betrachte eine Folge approximierender Figuren
$$F_m := \bigsqcup_{i \leq m}(A_{1,m} \times A_{2,m}).$$
Wir schätzen $(\mu_1 \boxtimes \mu_2) (F_m)$ nach unten ab. Die $X_2$-Querschnitte für $x_1 \in A_1$ 
$$S_{2,m}(x_1) \subset A_2$$
sind definiert durch
$$\{x_1\} \times S_{2,m}(x_1) = F_m \cap ( \{x_1\} \times A_2).$$
Dann folgt
$$S_{2,m} (x_1) \in \RR_2$$
da $S_{2,m}(x_1)$ die Vereinigung einiger $A_{2,i}$ ist. $S_{2,m}(x_1)$ wachsen mit $m$ und $S_{2,m}(x_1) \nearrow A_2$ für $m \to \infty$ und festes $x_1$. \\
Für festes $m$ treten endlich viele Querschnitte auf. Die Punkte $x_1 \in A_1$, für die $S_{2,m}(x_1)$ ein bestimmter Querschnitt ist, bilden eine Figur $\in \RR_1$. (Liegt im Unterring erzeugt von $A_{1,i}$.) \\
Für $0<v_2<\mu_2(A_2)$ setze
$$
B_{1,m}(v_2) := \{x_1 \in A_1 \mid \mu_2(S_{2,m} (x_1) ) \geq v_2 \} \in \RR_1.
$$
Dann erhalten wir die Volumenabschätzung:
$$(\mu_1 \boxtimes \mu_2) (F_m) \geq \mu_1 (B_{1,m}(v_2))\cdot v_2.$$
Somit:
\begin{equation*}
	\begin{split}
				&	S_{2,m}(x_1) \nearrow A_2	\\
			\xRightarrow{\mu_2 \text{ Prämaß}} & \mu(S_{2,m}(x_1) \nearrow \mu_2(A_2) \\
			\xRightarrow{\quad \quad\quad\  \,} & B_{1,m}(v_2) \nearrow A_1 \\
			\xRightarrow{\mu_1 \text{ Prämaß}} & \mu_1 (B_{1,m}(v_2)) \nearrow \mu_1(A_1) \\
			\xRightarrow{\quad \quad\quad\  \,} & \lim_{m \to \infty} (\mu_1 \boxtimes \mu_2) (F_m) \geq \mu_1(A_1) \cdot v_2\ \  \forall v_2 \in (0,\mu_2(A_2)) \\
			\xRightarrow{\quad \quad\quad\  \,} & \lim_{m \to \infty} (\mu_1 \boxtimes \mu_2) (F_m) \geq \mu_1(A_1) \cdot \mu_2(A_2) \\
	\end{split}
\end{equation*}
Also die Gleichheit, d.h. $\mu_1 \boxtimes \mu_2$ ist ein Prämaß.
\end{proof}
\end{satz}

\begin{remark}
Der letzte Satz wird häufig erst im Rahmen der Integrationstheorie bewiesen (mit Satz von der monotonen Konvergenz).
\end{remark}

Zurück zum unsren Hauptbeispiel: Weil der Inhalt $\lambda^d_{\FF^d}$ die $d$-te Potenz des Prämaßes $\lambda^1_{\FF^1}$ ist, folgt

\begin{korollar}
Der Inhalt $\lambda^d_{\FF^d}$ ist ein Prämaß.
\end{korollar}

\begin{definition}
	\begin{mdframed} 
		Das Prämaß $\lambda^d_{\FF^d}$ heißt \textbf{das $\boldsymbol{d}$-dimensionale Lebesgue-Prämaß}.
	\end{mdframed}
\end{definition}

\subsection{$\sigma$-Algebren}
\subsubsection{$\sigma$-Algebren}
Wir fordern von Teilmengenfamilien jetzt, dass sie \emph{reichhaltiger} und \emph{flexibel} sind, d.h. dass man innerhalb von ihnen zusätzlich zu den genannten endlichen Mengenoperationen auch gewisse \emph{abzählbar unendliche} Operationen (nämlich Vereinigungen) durchführen kann. D.h. Sie sollen abgeschlossen sein unter gewissen \emph{Grenzprozessen.}

\begin{definition}[\textbf{$\boldsymbol\sigma$-Algebra}]
\begin{mdframed}
Eine Familie $\AA\subset \mathcal{P}(X)$ heißt $\sigma$-Algebra auf $X$, falls:
\begin{enumerate}[(\roman*),topsep=5pt, itemsep = 0 pt]
	\item $\emptyset \in \AA$,
	\item \emph{$\sigma$-$\cup$-stabil}, d.h. Für jede Folge $(A_n)_{n\in \N}$ von Teilmengen mit $A_n \in \AA$ gilt $\bigcup\limits_{n\in\N} A_n \in \AA$,
	\item \emph{$\complement$-stabil}, d.h. $A \in \AA \implies \complement A \in \AA$.
\end{enumerate}
\end{mdframed}
Man kann (i) bzw. (iii) ersetzen durch die bezüglich Komplementbildung duale Forderung:
\begin{enumerate}[(\roman*),topsep=5pt, itemsep = 0 pt]
	\item[(i*)] $X \in \AA$,
	\item[(ii*)] \emph{$\delta$-$\cap$-stabil}, d.h. Für jede Folge $(A_n)_{n\in \N}$ von Teilmengen mit $A_n \in \AA$ gilt $\bigcap\limits_{n\in\N}A_n \in \AA$.
\end{enumerate}
Leicht können wir einsehen, dass $\sigma$-Algebren Algebren sind.
\begin{remark}
\begin{enumerate}[(\roman*),topsep=5pt, itemsep = 0 pt]
	\item Beachte formale Ähnlichkeit von (iii) bzw. (iii*) mit Definition einer Topologie: Die Topologie ist die Familie aller offenen Teilmengen, sie ist stabil unter beliebigen Vereinigungen und endlichen Durchschnitten.
	\item In $\sigma$-Algebren gilt: Alle von innen bzw. außen durch Teilmengen in der $\sigma$-Algebra approximierbaren Teilmengen gehören wieder zur $\sigma$-Algebra.
\end{enumerate}
\end{remark}
\end{definition}

Wie im Fall von Ringen und Algebren: Beliebige Durchschnitte von $\sigma$-Algebren sind wieder $\sigma$-Algebren, deswegen ist die folgende Definition sinnvoll:

\begin{definition}[\textbf{Erzeugendensystem von $\boldsymbol\sigma$-Algebren}]
\begin{mdframed}
Die von einer Familie $\EE\subset \mathcal{P}(X)$ \emph{erzeugte} $\sigma$-Algebra $\sigma(\EE)$  ist die eindeutige kleinste sie enthaltende $\sigma$-Algebra. Man nennt $\EE$ ein(en) \emph{Erzeugendensystem} oder \emph{Erzeuger} von $\sigma(\EE)$.
\end{mdframed}
\end{definition}
Die Definition einer $\sigma$-Algebra durch ein Erzeugendensystem ist nicht \emph{konstruktiv}! Im Fall von Ringen und Algebren kann man die enthaltenen Teilmengen ausgehend von einem Erzeugendensystem noch explizit angeben, nämlich \emph{induktiv} in \emph{abzählbaren vielen} Schritten konstruieren. Diese ist bei $\sigma$-Algebren abgesehen von Ausnahmenfällen nicht mehr möglich, sondern man benötigt zur Konstruktion aus einem Erzeugendensystem einen Prozess mit \emph{überabzählbaren vielen} Schritten. (siehe transfinite Induktion) Dieser Umstand reflektiert, dass die in $\sigma$-Algebren enthaltenen Teilmengen i.A. so kompliziert sind, dass man sie nicht mehr explizit beschreiben kann. \newline \newline
\begin{small}
\textit{Ergänzend:} Ist $\EE\subset\mathcal{P}(X)$, so kann man durch Anwenden abzählbarer Mengenoperationen immer größere Teilmengenfamilien angeben, die zu $\sigma(\EE)$ gehören müssen und deren ``Verwandtschaftsgrad'' zu $\EE$ gleichzeitig sukzessive abnimmt.\\
Zu $\FF \subset \mathcal{P}(X)$ definiere 
\begin{equation*}
\begin{split}
\FF \subset \FF^\nu & := \left\{
\substack{\text{\small abzählbare Vereinigungen von Teilmengen in $\FF$}  \\ \text{\small sowie deren Komplementen}}
\right\} \subset \mathcal{P}(X) 	\\
\EE_0 & := \EE \cup \{\emptyset\} \\
\EE_n & := \EE^\nu_{n-1}	\text{ für } n\in \N
\end{split}
\end{equation*}
Anfang der ``Borel-Hierarchie'', fortsetzen mit transfiniter Induktion.
\end{small} 

\paragraph{Einfachste Beispiele.} 
\begin{enumerate}[(\roman*),topsep=5pt, itemsep = 0 pt]
	\item[(o)]  $\{\emptyset, X\} \subset \mathcal{P}(X)$ ist eine $\sigma$-Algebra auf $X$, erzeugt von $\emptyset$.\\
	$\PP (X) \subset \PP (X)$ ist ebenfalls eine $\sigma$-Algebra.
	\item Ist $A \subset X$, so ist $\{\emptyset, A, \complement A, X\}$ eine $\sigma$-Algebra auf $X$, erzeugt von $\EE=\{A\}$.
	\item Die Teilmengen von $X$, die abzählbar sind oder abzählbares Komplement haben, bilden eine $\sigma$-Algebra $\AA$ auf $X$, erzeugt von den \emph{einelementigen} Teilmengen. Es gilt: $\AA \neq \PP (X) \iff X $ überabzählbar. 
\end{enumerate}
Von zentraler Wichtigkeit sind Beispiele \emph{topologischen} Ursprungs:

\begin{example} \ 
\begin{enumerate}[(\roman*),topsep=5pt, itemsep = 0 pt]
	\item (\textbf{Borelsche $\boldsymbol\sigma$-Algebra, Borel-Mengen}) Jeder topologische Raum $(X,\TT)$ trägt eine natürliche $\sigma$-Algebra, nämlich die von seiner Topologie erzeugte $\sigma$-Algebra
	$$\BB(X):=\sigma(\TT).$$
	Wir nennen $\BB(X)$ die \emph{Borelsche $\sigma$-Algebra} von $X$. In ihnen enthaltenen Teilmengen heißen \emph{Borel-Mengen}. \\
	\textit{Notation}: 
	\begin{equation*}
		\begin{split}
			\BB(\R^d) & =: \BB^d \\
			\BB(\R) & =: \BB
		\end{split}
	\end{equation*}
	\item Jeder \emph{metrische} Raum $(X,d)$ trägt eine natürliche Topologie, die von den offenen metrischen Bällen erzeugte Topologie $\TT_d$ und damit eine natürliche $\sigma$-Algebra
	$$\BB(X):=\sigma(\TT_d).$$
\end{enumerate}
\end{example}

\begin{remark} \
\begin{enumerate}[(\roman*),topsep=5pt, itemsep = 0 pt]
	\item Die Borelsche $\sigma$-Algebra eines topologischen Raumes wird ebenfalls von den abgeschlossenen Teilmengen erzeugt. (vgl. die Stabilität unter Komplementbildung)
	\item Ist die Topologie \emph{Hausdorffsch}, so sind kompakte Teilmengen abgeschlossen\footnote{In metrischen Räumen sind Kompakta abgeschlossen und beschränkt. Allgemeiner: In topologischen Räumen mit Hausdorff-Eigenschaft sind Kompakta abgeschlossen. (Ana II, S.30)}, und daher die von den Kompakta erzeugte $\sigma$-Algebra in der Borelschen enthalten. Besitzt der Raum eine abzählbare Überdeckung durch Kompakta, so sind beide $\sigma$-Algebren gleich.
\end{enumerate}
\end{remark}
\marginpar{\tiny{4.11.2019}}
Wir haben also gesehen:
$$
\underset{\text{topologischer Raum}}{(X,\TT)} \leadsto \text{Borelsche }\sigma\text{-Algebra } \BB(X):= \sigma(\TT).
$$
$\BB^d=\BB(\R^d)$ hat verschiedene natürliche Erzeuger, z.B.
\begin{itemize}
	\item die Familie der offenen Teilmengen,
	\item die Familie der abgeschlossenen Teilmengen,
	\item die Familie der Kompakta,
	\item die Familie der offenen Bälle, bzw. offenen Bälle mit Radien $\in \Q^+$ und Zentren $\in \Q^d$, denn jede offene Teilmenge ist eine abzählbare Vereinigung von offenen Bällen,
	\item die Familie der abgeschlossenen Bälle,
	\item die Familie der offenen (achsenparallelen) Quader,
	\item die Familie der abgeschlossenen (achsenparallelen) Quader.
\end{itemize}

\begin{lemma}
\begin{mdframed}
$\BB^d$ wird auch von $\QQ^d$ (halboffenen, achsenparallelen Quadern) erzeugt.
\end{mdframed}
\begin{proof}
Einerseits: Jede offene Teilmenge von $\R^d$ ist (notwendigerweise) wegen der Dichtheit von $\Q$ in $\R$ eine abzählbare Vereinigung von Quandern $[a,b)$ aus $\QQ^d$ mit rationalen Ecken, d.h. $a_i<b_i \ \forall i, a,b\in \Q^d$. Also $\BB^d \subset \sigma (\QQ^d)$.\\
Anderseits ist jeder Quader $[a,b)$ mit $a_i<b_i\ \forall i$ ein abzählbarer Durchschnitt offener Quader, gehört also zu $\BB^d$, denn
$$
[a,b) = \underbrace{\bigcap_{n\in \N} \underbrace{\left(a_1-\frac{1}{n}, b_1\right) \times ... \times \left(a_d-\frac{1}{n},b_d \right)}_{\in \TT}}_{\in \ \sigma(\TT) = \BB^d}
$$
Also die Gleichheit.
\end{proof}
\end{lemma}
$\BB^d$ ist so reichhaltig, dass sie alle ``denkbaren'' geometrischen Gebilde in $\R^d$ enthält. Genauerer Vergleich von $\BB^d$ mit $\TT_{\R^d}$ folgt später.

\subsubsection{Dynkin-Systeme}
Wir betrachten jetzt eine Abschwächung von $\sigma$-Algebren, die natürlich auftritt.
\begin{definition}
\begin{mdframed}
Eine Familie $\DD \subset \PP(X)$ heißt ein Dynkin-System auf $X$, falls
\begin{enumerate}[(\roman*),topsep=5pt, itemsep = 0 pt]
	\item $\emptyset \in \DD$,
	\item $\complement$-stabil,
	\item Für jede Folge $(A_n)_{n \in \N}$ \emph{paarweise disjunkter Teilmengen} $A_n \in \DD$ gilt $\bigcup\limits_{n\in \N} A_n \in \DD$.
\end{enumerate}
\end{mdframed}
Im Vergleich zu $\sigma$-Algebren fordern wir zusätzlich die (paarweise) Disjunktheit der Folgenglieder. $\sigma$-Algebren sind Dynkin-Systeme.
\end{definition}

\begin{proposition}
\begin{mdframed}
$\cap$-stabile Dynkin-Systeme sind $\sigma$-Algebren.
\end{mdframed}
\begin{proof}
%$\sigma$-Algebren sind Dynkin-Systeme und $\cap$-stabil, also $\cap$-stabile Dynkin-Systeme. 
%Ist $\DD$ ein $\cap$-stabiles Dynkin-System, so lassen sich abzählbare Vereinigungen als abzählbare \emph{disjunkte} Vereinigungen ausdrücken:
%\begin{equation*}
%	\begin{split}
%		     \bigcup_{n \in \N}A_n
%		=    A_1 \sqcup (A_2 \cap \complement A_1) \sqcup  ... \sqcup \underbrace{\left(A_n \cap \bigcap\limits_{i=1}^{n-1}\complement A_i\right)}_{\in \DD} \sqcup ... \in \DD
%	\end{split}
%\end{equation*}
%Also ist $\DD$ eine $\sigma$-Algebra.
Sei $\DD$ ein $\cap$-stabiles Dynkin-System. Aus der $\complement$-Stabilität und $\cap$-Stabilität folgt die $\cup$-Stabilität wegen $A\cup B=\complement(\complement A \cap \complement B)$. $\DD$ ist also eine Algebra. In Algebren lassen sich abzählbare Vereinigungen als abzählbare \emph{disjunkte} Vereinigungen ausdrücken:
\begin{equation*}
A_1 \cup A_2 ... \cup A_n \cup ... = A_1 \sqcup \underbrace{(A_2 \setminus A_1)}_{\in \DD} \sqcup \underbrace{(A_3 \setminus (A_1\sqcup A_2))}_{\in \DD} \sqcup ... \in \DD
\end{equation*}
D.h. $\DD$ enthält beliebige abzählbare Vereinigungen von Teilmengen aus $\DD$, ist also eine $\sigma$-Algebra.
\end{proof}
\end{proposition}

\begin{lemma}
\begin{mdframed}
Ist $\DD \subset \PP(X)$ ein Dynkin-System und $D_1, D_2 \in \DD$ mit $D_1 \subset D_2$, so ist $D_2 \setminus D_1 \in \DD$.
\end{mdframed}
\begin{proof}
Wegen $\complement(D_2\setminus D_1) = D_1 \sqcup \complement D_2 \in \DD$ gilt $D_2 \setminus D_1 \in \DD$.
\end{proof}
\end{lemma}
$\leadsto$ Die Dynkin-Eigenschaft vererbt sich bei gewissen Konstruktionen:
\begin{folgerung}
\begin{mdframed}
Für $D \in \DD$ ist die Spur
$$\DD |_D := \{ M\subset D \mid M \in \DD\}\subset \PP(D)$$
von $\DD$ auf $D$ ein Dynkin-System auf $D$. Ebenso ist
$$\DD_D := \{M \subset X \mid M \cap D \in \DD \} \subset \PP (X)$$
ein Dynkin-System auf $X$.
\end{mdframed}
\begin{proof}
%Die $\complement$-Stabilität von $\DD|_D$ folgt aus dem Lemma, und die Stabilität unter abzählbaren disjunkten Vereinigungen ist klar. Genauso im zweiten Fall.
Ist $M \in \DD|_D$, so ist $M \in \DD$ und wegen des Lemmas $\complement M = D\setminus M \in \DD$, also $\complement M \in \DD|_D$. Ist $(A_n)_{n\in \N}$ eine Folge paarweise disjunkter Teilmengen in $\DD|_D$, so ist $A_n \in \DD\ \forall n$, folglich $\bigcup\limits_{n \in \N}A_n \in \DD$, also $\bigcup\limits_{n\in\N} A_n \in \DD|_D$.\\Ist $M\in\DD_D$, so ist $M\cap D \in \DD$ und wegen des Lemmas $\underbrace{D \setminus (M\cap D)}_{\complement M \cap D} \in \DD$, also $\complement M \in \DD_D$. Ist $(A_n)_{n\in \N}$ eine Folge paarweise disjunkter Teilmengen in $\DD_D$, so ist $\left(\bigcup\limits_{n\in \N}A_n\right)\cap D \in \DD$, also $\bigcup\limits_{n\in\N} (A_n \cap D) \in \DD$, d.h. $\bigcup\limits_{n\in\N}A_n \in \DD_D$.
\end{proof}
\end{folgerung}
Da beliebige Durchschnitte von Dynkin-Systemen wieder solche sind, gibt es für jede Familie $\EE\subset\PP(X)$ ein eindeutiges kleinstes sie enthaltendes Dynkin-System $\DD(\EE)\subset\PP(X)$, genannt \textbf{das von $\EE$ erzeugte Dynkin-System}. Klar gilt $\DD(\EE)\subset\sigma(\EE)$.
\begin{satz}
\begin{mdframed}
Von $\cap$-stabilen Familien erzeugte Dynkin-Systeme sind $\cap$-stabil, also $\sigma$-Algebren.
\end{mdframed}
\begin{proof}
Sei $\EE \subset \PP(X)$ $\cap$-stabil. Zu zeigen: $\DD(\EE)$ ist $\cap$-stabil. Sei $E \in \EE$. Fixiere noch ein $E' \in \EE$. Dann gilt nach Voraussetzung $E' \cap E \in \EE \subset \DD(\EE)$, d.h. $E' \in \DD(\EE)_E$ (vgl. oben). Daraus folgt $\EE \subset \DD(\EE)_E$ und somit $\DD(\EE) \subset \DD(\EE)_E$, da $\DD(E)$ das kleinste Dynkin-System ist, das $\EE$ enthält. Wir haben also gezeigt: $\DD(\EE)\subset \DD(\EE)_E \ \forall E \in \EE$. Es folgt
\begin{equation*}
\begin{rcases}
	E \in \EE \\
	D \in \DD(\EE)
\end{rcases}
\implies 
D \in \DD(\EE)_E\implies
D \cap E \in \DD(\EE) \implies E \in \DD(\EE)_D
\end{equation*}
Wir haben also gezeigt: $\EE \subset \DD(\EE)_D \ \forall D \in \DD(\EE)$, weiter gilt $\DD(\EE)\subset \DD(\EE)_D \ \forall D \in \DD(\EE)$. Dies wiederum bedeutet:
\begin{equation*}
	D,D' \in \DD(\EE) \implies D \cap  D' \in \DD(\EE)
\end{equation*}
Also ist $\DD(\EE)$ $\cap$-stabil, wird aufgrund der letzten Proposition eine $\sigma$-Algebra.
\end{proof}
\begin{remark}
Ist $\EE$ $\cap$-stabil, so gilt $\DD(\EE) =\sigma (\EE)$.
\end{remark}
\end{satz}
\subsubsection{Die messbare Kategorie}
``Messbare Objekte'' sind Mengen versehen mit $\sigma$-Algebren als ``messbaren Zusatzstrukturen''.
\begin{definition}[\textbf{Messraum, Messbarkeit von Teilmengen}]
\begin{mdframed}
Ein \textbf{Messraum} ist ein Paar $(X,\AA)$ bestehend aus einer Menge $X$ und einer $\sigma$-Algebra $\AA \subset \PP(X)$. Die Teilmengen in $\AA$ heißen \textbf{messbar}.
\end{mdframed}
\end{definition}
$\leadsto$ werden Definitionsbereiche unserer Volumenfunktionen (Maße) sein.

\begin{definition}[\textbf{Messbarkeit von Abbildungen}]
\begin{mdframed}
Eine Abbildung $f:(X,\AA) \longrightarrow (Y,\BB)$ von Messräumen heißt \textbf{messbar}, falls $f^{-1}(B) \in \AA \ \forall B \in \BB$, d.h. Urbilder messbarer Teilmengen sollen messbar sein. Man sagt auch $f:X\longrightarrow Y$ ist \textbf{$\AA$-$\BB$-messbar}.
\end{mdframed}
\end{definition}

\begin{beobachtung}
Kompositionen messbarer Abbildungen sind messbar, denn: ist
$$(X,\AA)\xrightarrow[\text{messbar}]{f} (Y,\BB) \xrightarrow[\text{messbar}]{g} (Z,\CC),$$
so gilt
$$C\in \CC \implies g^{-1}(C) \in \BB \implies \underbrace{f^{-1}(g^{-1}(C))}_{(g\circ f)^{-1}(C)} \in \AA.$$
Also $g\circ f$ ist $\AA$-$\CC$-messbar.
\end{beobachtung}

\paragraph{Zurückziehen von $\sigma$-Algebren:}
Eine Abbildung von Mengen $f:X \longrightarrow Y$ induziert kontravariant eine Abbildung $F^*:\PP(Y)\longrightarrow\PP(X), M \longmapsto f^{-1}(M)$. \\
Für $\FF \subset \PP(Y)$ nennt man
$$f^* \FF := \{f^{-1}(M) \mid M \in \FF \} \subset \PP(X)$$
die mit $f$ zurückgezogene Familie.\\
Ist $\BB\subset \PP(Y)$ eine $\sigma$-Algebra, so ist $f^*\BB \subset \PP(X)$ auch eine $\sigma$-Algebra, denn
\begin{equation*}
\begin{split}
	f^{-1} \left(\bigcup M_k\right) & = \bigcup f^{-1}(M_k) \\
	f^{-1}\left(\complement M\right) & = \complement f^{-1}(M).
\end{split}
\end{equation*}

\begin{beobachtung}
Gegeben eine $\sigma$-Algebra $\BB$ auf $Y$, ist $f^*\BB$ die kleinste $\sigma$-Algebra auf $X$, bezüglich derer $f$ messbar ist. Deswegen ist die Reformulierung von der Definition der Messbarkeit der Abbildungen sinnvoll:
\end{beobachtung}

\begin{definition}
Eine Abbildung $f:(X,\AA)\longrightarrow (Y,\BB)$ ist genau dann messbar, wenn $f^*\BB\subset\AA$. Ist speziell $X\subset Y$ eine Teilmenge und $\iota:X\longrightarrow Y$ die Inklusion, so ist $\iota^{-1}(M)=X\cap M$ für $M\subset Y$. Für eine $\sigma$-Algebra $\BB$ auf $Y$ nennt man
$$\BB|_X := \iota^*\BB=\{M\cap X \mid M \in \BB\} \subset \PP(X)$$
die \textbf{Spur-$\sigma$-Algebra} von $\BB$ auf $X$. Die Inklusion ist dann $\BB|_X$-$\BB$-messbar.
\end{definition}
Man kann Erzeugendensysteme von $\sigma$-Algebren zurückziehen:
\begin{lemma}
\begin{mdframed}
Ist $f: X \longrightarrow Y$ eine Abbildung und $\FF \subset \PP(Y)$, so gilt $\sigma(f^*\FF) = f^* (\sigma(\FF))$.
\end{mdframed}
\begin{proof}
Einerseits: $f^*(\sigma(\FF))$ ist $\sigma$-Algebra, enthält $f^*\FF$ $\implies$ $\sigma(f^*\FF) \subset  f^*(\sigma(\FF))$. \\
Anderseits ist
$
\{A \in \sigma(\FF) \mid f^{-1}(A) \in \sigma(f^* \FF) \}
$ eine Algebra, die $\FF$ enthält und in $\sigma(\FF)$ enthalten ist, somit Gleichheit. D.h. $A\in \sigma(\FF)\implies f^{-1}(A) \in \sigma(f^*\FF)$, somit $f^*(\sigma(\FF))\subset \sigma(f^*\FF).$ %Dies zeigt: $f^{-1}(A)\in \sigma(f^*\FF)\implies A \in \sigma(\FF)$.
\end{proof}
\marginpar{\tiny{7.11.2019}}
Das Lemma lässt sich reformulieren: Ist $f: X \to Y$ eine Abbildung, $\BB \subset \PP(Y)$ eine $\sigma$-Algebra und $\EE \subset \BB$ Erzeugendensystem, so wird $f^*\BB$ von $f^* \EE$ erzeugt. Wir können also Erzeuger zurückziehen.
\end{lemma}
\begin{equation*}
	\begin{array}{ccc}
		\text{topologische Kategorie} & \leadsto & \text{Klasse der Borelschen Messräume} \\
		(X,\TT) & \leadsto & (X, \BB(X))
	\end{array}
\end{equation*}

\begin{definition}
\begin{mdframed}
Sind $(X,\TT_X)$ und $(Y,\TT_Y)$ topologische Räume, so heißt eine Abbildung $f:X \longrightarrow Y$ \textbf{Borel-messbar}, falls sie $\BB(X)$-$\BB(Y)$-messbar ist.
\end{mdframed}
\end{definition}

\begin{lemma}
\begin{mdframed}
Stetige Abbildungen sind Borel-messbar.
\end{mdframed}
\begin{proof}
Mit Notation der Definition: Die Stetigkeit von $f:X \longrightarrow Y$ bedeutet, dass $f^* \TT_Y \subset \TT_X$. Es folgt
$$
f^*(\underbrace{\BB(Y)}_{\sigma(\TT_Y)}) \overset{\text{Lemma}}= \sigma(\underbrace{f^*\TT_Y}_{\subset \TT_X}) \subset \sigma(\TT_X) = \BB(X).
$$
\end{proof}
\end{lemma}

\subsubsection{Produkte von $\sigma$-Algebren}
Wir konstruieren endliche Produkte von $\sigma$-Algebren bzw. Messräumen. 
\begin{definition}
\begin{mdframed}
Seien $(X_i, \AA_i)$ Messräume ($i=1,...,n)$. \vspace{0.5pc} \\
Die \textbf{Produkt-$\boldsymbol\sigma$-Algebra} $\AA = \AA_1 \otimes ... \otimes \AA_n$ ist definiert als die $\sigma$-Algebra auf $X = X_1 \times ... \times X_n$ erzeugt von $\ZZ = \ZZ(\AA_1, ..., \AA_n)$, d.h. von $\pi_i^{-1}(A_i) \ \forall A_i \in \AA_i$\tablefootnote{Die $\pi_i^{-1}(A_i)$ sind spezielle Quader, da $X_i \in \AA_i$.}, $i=1,...,n$ und ebenfalls erzeugt vom Halbring $\QQ = \AA_1 \ast ... \ast \AA_n \supset \ZZ$ der Quader $A_1 \times ... \times A_n$ mit $A_i \in \AA_i$. 
\vspace{0.5pc}\\
Die $\sigma$-Algebra $\AA_1 \otimes ... \otimes \AA_n$ auf $X_1 \times ... \times X_n$ heißt das Produkt der $\sigma$-Algebren $\AA_i$, und der Messraum $(X_1\times ... \times X_n, \AA_1 \otimes ... \times \AA_n)$ das Produkt der Messräume $(X_i,\AA_i)$.
\end{mdframed}
\end{definition}
Das Produkt ist \emph{assoziativ}, z.B. gilt $(\AA_1 \otimes \AA_2) \otimes \AA_3 = \AA_1 \otimes \AA_2 \otimes \AA_3 = \AA_1 \otimes (\AA_2 \otimes \AA_3)$.\Ueb 
\vspace{0.5pc}\\
Die natürlichen Projektionen $\pi_k:X_1 \times ... \times X_n \longrightarrow X_k$ sind messbar. Die Produkt-$\sigma$-Algebra ist charakterisiert als \emph{die kleinste $\sigma$-Algebra} auf $X_1 \times ... \times X_n$, so dass die $\pi_k$ bezüglich der $\AA_k$ \emph{messbar} sind. 
\vspace{0.5pc}\\
Sind $\EE_i \subset \AA_i$ Erzeugendensysteme, so ist $\ZZ(\EE_1,...,\EE_n)$ ein Erzeugendensystem von $\AA_1 \otimes ... \otimes \AA_n$, ebenso wie $\EE_1 \ast ... \ast \EE_n$, denn
\begin{equation*}
	\begin{array}{ccc}
		\pi_k^{-1}(M_k) & \xleftrightarrow{\quad\text{bij.}\quad} & M_k \subset X_k \\
		\{ \pi^{-1}_k(E_k) \} & \xleftrightarrow{\quad\text{bij.}\quad} & \{E_k\} \\
		\text{erzeugt} & &\text{erzeugt} \\
		\underset{\sigma\text{-Algebra}}{\{\pi_k^{-1}(A_k)\}} & \xleftrightarrow{\quad\text{bij.}\quad}  & \AA_k
	\end{array}
\end{equation*}
\subsection{Maße}
\subsubsection{Beispiele und Definitionen}
\begin{definition}
\begin{mdframed}
Ein Maß ist ein auf einer $\sigma$-Algebra definiertes Prämaß.
\end{mdframed}
\end{definition}

\begin{example} \
\begin{enumerate}[label=(\roman*),topsep=3pt, itemsep=0pt]
	\item(Punkteinheitsmaße, Diracmaße) Sei $x\in X$. Die \emph{Dirac-Maße} $\delta_x$ auf $\PP(X)$ sind definiert als 
	$$
\delta_x(A):=\begin{cases}
	1, \quad &\text{falls }x \in A, \\
	0, \quad &\text{sonst}
\end{cases}	
	$$
	Durch Linearkombinationen $\sum\limits_{i=1}^k m_i \delta_{x_i}, m_i \in \R_0^+, x_i \in X$ erhält man weitere Maße, die ``endlichen Maßverteilungen''. Man kann ebenfalls allgemeinere Maßverteilungen $(\longleftrightarrow \text{ Fkt }X\to \R_0^+)$ mit beliebigem Träger $\subset X$ bilden.
	\item (Zählmaß) Das Zählmaß $\zeta:\PP(X) \longrightarrow \N_0 \cup \{\infty\} \subset [0,\infty]$ ist definiert durch
	$$\zeta(A)= \begin{cases}
	|A|, \quad &\text{falls $A$ endlich},\\
	\infty,\quad &\text{sonst}
	\end{cases}$$
	\item Auf $\PP(X)$ wird ein Maß $\mu$ definiert durch
	$$\mu(A)=\begin{cases}
		0, \quad &\text{falls } A = \emptyset, \\
		\infty, \quad &\text{sonst}
	\end{cases}$$
	\item Sei $X$ überabzählbar und $\AA$ die von den endlichen Teilmengen erzeugte $\sigma$-Algebra, d.h. abzählbare Teilmengen und deren Komplemente. Dann ist
	$$\nu(A)=\begin{cases}
	0, \quad &\text{falls } A \text{ abzählbar},\\
	1, \quad &\text{falls }\complement A \text{ abzählbar}
	\end{cases}$$
	ein Maß auf $\AA$.
\end{enumerate}
\end{example}

Teilmengen von $\sigma$-Algebren sind i.A. kompliziert und nicht explizit beschreibbar. Entsprechend sind Maße i.A. nicht beschreibbar durch explizite Angabe ihrer Werte. $\leadsto$ Konstruktion durch \emph{Fortsetzung}.
\\\\
\textbf{Ansatz/ Methode:}
\begin{equation*}
	\begin{array}{c}
	\text{Inhalt auf Halbring}\\
	\text{(Prämaß auf Ring)}
	\end{array}
	\underset{\text{Fortsetzung}}{\overset{\text{eindeutige}}\leadsto}
	\text{Maße auf $\sigma$-Algebren}
\end{equation*}
Daher brauchen wir Fortsetzungsresultate.

\subsubsection{Äußere Maße und Messbarkeit}
Sei $\HH \subset \PP(X)$ ein Halbring und $\mu:\HH \longrightarrow [0,\infty]$ ein Inhalt. Unser Ziel ist, $\mu$ auf $\sigma(\HH)$ fortzusetzen.\\\\
Wir geben für \emph{beliebige} Teilmengen von $X$ ein \textbf{äußeres $\boldsymbol\mu$-Volumen} \ (bzw. eine \textbf{obere $\boldsymbol\mu$-Volumenschranke}), indem wir \emph{abzählbare} Überdeckungen durch Teilmengen aus $\HH$ verwenden. Wir definieren für $M \subset X$
$$
\boxed{\mu^*(M):= \inf \left\{ \sum_{n=1}^\infty \mu(A_n) \ \bigg\vert\ (A_n)_{n\in \N} \text{ Folge in $\HH$ mit } M \subset {\bigcup_{n\in \N} A_n} \right\}}.
$$
$\bigcup\limits_{n\in\N} A_n$ ist also eine abzählbare Überdeckung von $M$, beinhaltet alle endlichen. Falls keine Überdeckung existiert, so $\mu^*(M)=\inf \emptyset = \infty$.
\\\\
Wir können immer zu disjunkter Überdeckung übergehen:
$$\bigcup_{n=1}^\infty A_n = A_1 \cup (A_2\setminus A_1) \cup ... \cup \underbrace{(A_n \setminus (A_1 \cup ... \cup A_{n-1}))}_{\text{disjunkt zerlegbar in Teilmengen aus }\HH} \cup ...$$
Daher genügt es, Infimum über disjunkte Überdeckung zu bilden.\\\\
Wir erinnern uns an unser erstes Fortsetzungsresultat: Ist $\mu$ ein Inhalt auf $\HH$, so existiert eine eindeutige $\ol{\mu}$ auf dem von $\HH$ erzeugten Ring $\RR$, der aus endlichen disjunkten Vereinigungen von Teilmengen aus $\HH$ besteht. D.h. Überdeckungen durch Teilmengen aus $\RR$ sind nicht effektiver als Überdeckungen durch Teilmengen aus $\HH$. Anders gesagt,
$$\boxed{\ol\mu^* = \mu^*}.$$
Ist $M \subset B \in \RR$, so genügt es, Infimum über Überdeckungen durch in $B$ enthaltene Teilmengen zu bilden. \Bild \ Analog: Ist $M\cap B = \emptyset$, so genügt es, Infimum über Überdeckungen durch zu $B$ disjunkte Teilmengen zu bilden.

\begin{lemma}
\begin{mdframed}
Die obere $\mu$-Volumenschranke $\mu^*:\PP(X) \longrightarrow [0,\infty]$ hat folgende Eigenschaften:
\begin{enumerate}[label=(\roman*),topsep=3pt, itemsep=0pt]
	\item $\mu^*(\emptyset) = 0$,
	\item \emph{Monotonie:} $M \subset M' \implies \mu^*(M) \leq \mu^*(M')$,
	\item \emph{$\sigma$-Subadditivität:} Für Folgen $(M_n)_{n \in \N}$ von (nicht notwendigerweise disjunkten) Teilmengen $M_n \subset X$ gilt: $\mu^*\big(\bigcup\limits_{n \in \N} M_n \big)\leq \sum\limits_{n\in\N}\mu^*(M_n)$.
\end{enumerate}
\end{mdframed}
\begin{proof}
Zu (iii): Sei $(\underbrace{A_{n,m}}_{\in \HH})_m$ eine Überdeckung von $M_n$. Dann ist $(A_{n,m})_{(n,m)}$ eine Überdeckung von $\bigcup\limits_{n \in \N}M_n$. Es folgt
$$
\mu^*\left( \bigcup_{n \in \N} M_n\right) \leq \sum_{\substack{m\in\N\\n\in\N}} \mu^*(A_{n,m}) \underset{\text{\tiny für Doppelreihen}}{\overset{\text{\tiny großer Umordnungssatz}}=}  \sum_{n\in\N}\sum_{m\in\N} \mu^* (A_{n,m})
$$
Die Summe $\sum\limits_{\substack{m\in\N\\n\in\N}} \mu^*(A_{n,m})$ ist wohldefiniert, da unabhängig von Summationsreihenfolge\footnote{siehe Umordnungssatz für Reihen mit nichtnegativen Summanden.}. Zu $\varepsilon_n >0$ können wir die Überdeckung $(A_{n,m})_m$ so wählen, dass $\sum\limits_{m\in \N} \mu^*(A_{n,m}) \leq \mu^*(M_n)+\varepsilon_n$. Dann folgt
$$
\sum_{n\in\N}\sum_{m\in\N} \mu^* (A_{n,m}) \leq \sum_{n \in \N} \mu^*(M_n) + \underbrace{\sum_{n \in \N} \varepsilon_n}_{\substack{>0, \\ \text{wird bel. klein}}}
$$
Also war die $\sigma$-Subadditivität.
\end{proof}
\end{lemma}

\begin{definition}
\begin{mdframed}
Eine Funktion $\mu^*: \PP(X) \longrightarrow [0,\infty]$ mit den oben genannten Eigenschaften (i)-(iii) heißt ein \textbf{äußeres Maß} auf $X$.
\end{mdframed}
\end{definition}

$\mu^*$ muss nicht von (zu) einem Inhalt $\mu$ induziert (assoziiert) sein. Falls $\mu^*$ zu Inhalt $\mu:\HH\longrightarrow [0,\infty]$ assoziiert, so gilt %wegen (iii)
\begin{equation*}
\boxed{\mu^*|_{\RR} \leq \ol{\mu}}.
\end{equation*}
\textbf{Partielle Additivität äußerer Maße:}  Seien $M_1,M_2\subset X$ disjunkt und durch $\RR$ trennbar, d.h. es existiert ein $B\in \RR$ mit $M_1\subset B$ und $M_2 \cap B = \emptyset$. Dann gilt\Ueb $$\mu^*(M_1\sqcup M_2)=\mu^*(M_1) + \mu^*(M_2).$$
Man kann auch ein \textbf{inneres $\mu$-Volumen} (bzw. \textbf{untere $\mu$-Volumenschranke} definieren\footnote{Dies ist nicht logisch notwendig für unsere Diskussion, motiviert aber die Diskussion von Messbarkeit, siehe unten.}. Die Approximation geschieht durch äußere Approximation des Komplements. Sei $M\subset X$ so, dass $M \subset R\in \RR$ mit $\ol\mu(R)<\infty$. Das innere $\mu$-Volumen ist definiert als
$$
\mu_*(M):= \underbrace{\mu^*(R)}_{\substack{\text{i.A. nur }\\\leq \ol\mu(R)}}-\mu^*(R\setminus M).
$$
Die Approximation von innen gelingt nicht: Ist z.B. $O\subset X$ offen und $A\subset O$ eine abzählbare dichte Teilmenge, so enthält $O\setminus A$ keine Quader, d.h. inneres Volumen $=0$.\\\\
Ist $R \subset R'$ mit $\ol\mu(R')<\infty$, so gilt wegen der partiellen Additivität
$$
\underbrace{\mu^*(R'\setminus M)}_{\mu^*(R\setminus M) + \mu^*(R'\setminus R)}-\mu^*(R\setminus M)=\mu^*(R'\setminus R) = \mu^*(R')-\mu^*(R),
$$
also ist $\mu_*$ wohldefiniert. \\\\
Diskrepanz zwischen äußerem und innerem Volumen:
$$
\mu^*(M)-\mu_*(M)=\underbrace{\mu^*(M)+\mu^*(R\setminus M)}_{\geq \mu^*(R)} - \mu^*(R)\geq 0
$$
Der ``Volumen-Exzess'' wird also erzeugt durch ``erzwungene Überlappung''. Intuition: der Exzess ist positiv, wenn der ``Rand'' von $M$ bzw. Übergangsbereich (Interface) von $M$ in $\complement M$ ``dick'' bzw. ``diffus'' ist.\\\\
\marginpar{\tiny{11.11.2019}}
Unser Ziel ist nun, die \emph{$\mu^*$-messbare} Teilmengen zu definieren und danach zeigen, dass diese eine \emph{$\sigma$-Algebra} $\AA^*\supset \RR$ bilden und dass $\mu^*|_{\AA^*}$ ein Maß ist. (Es setzt $\ol{\mu}$ fort, falls $\ol\mu$ ein Prämaß ist.)
\\\\
Wir betrachten $M$ als ``messbar'', falls es keinen ``Volumen-Exzess'' gibt, d.h. wir verlangen
$$
\mu^*(M)=\mu_*(M),
$$
also
$$
\mu^*(R)=\mu^*(M)+\mu^*(R\setminus M)		\quad \forall R \in \RR \text{ mit } R \supset M, \ol\mu(R)<\infty.
$$
Um auf \emph{beliebige} Teilmengen $M\subset X$ auszudehnen, \emph{lokalisiere} und \emph{verlange}:
\begin{equation}\label{I.6}
\mu^*(R)=\mu^*(R\cap M) + \mu^*(\underbrace{R\setminus M}_{R\cap \complement M}) \ \ \forall R \in \RR
\end{equation}
(\textit{Bemerkung:} Dies gilt wegen der Subadditivität von $\mu^*$ sowieso, falls $\mu^*(R)=\infty$.)\\
Es folgt, dass diese Gleichheit auch für Schnitte von $M$ mit \emph{beliebigen} Teilmengen $S \subset X$, d.h.
\begin{equation}\label{I.7}
\boxed{
\mu^*(S)=\mu^*(S\cap M) + \mu^*(S \setminus M)
}
\end{equation}
\textit{Begründung}: Ist $(A_n)_{n \in \N}$ eine Überdeckung von $S$ durch Teilmengen $A_n \in \RR$, so gilt 
\begin{equation*}
\begin{split}
\sum_{n\in\N}\ol\mu(A_n) \geq \sum_{n\in\N}\underbrace{\mu^*(A_n)}_{\overset{\eqref{I.6}}= \mu^*(A_n \cap M)+\mu^*(A_n \setminus M)} & = \underbrace{\sum_{n\in\N}\mu^*\underbrace{(A_n \cap M)}_{\text{Überdeckung } S\cap M}}_{\geq \mu^*(S\cap M)}+\underbrace{\sum_{n\in \N}\mu^*\underbrace{(A_n\setminus M)}_{\text{Überdeckung }S\setminus M}}_{\geq \mu^*(S\setminus M)} \\
& \geq  \mu^*(S\cap M) + \mu^*(S \setminus M) \overset{\text{Subadd.}}{\geq} \mu^*(S)
\end{split}
\end{equation*}
Nehmen wir nun Infimum über $(A_n)$, so ist die linke Seite der Ungleichung gleich $\mu^*(S)$ und es folgt
$$\mu^*(S)\geq \mu^*(S\cap M) + \mu^*(S \setminus M) \geq \mu^*(S),$$
also Gleichheit überall, d.h. \eqref{I.7} gilt. \hfill $\blacksquare$\\

Man gelangt so auf natürlicher Weise zur Schlüssel-Definition: (Carathéodory, 1914)

\begin{definition}
\begin{mdframed}
$M\subset X$ heißt \textbf{$\boldsymbol\mu^{\boldsymbol*}$-messbar}, falls 
\begin{equation*}
\mu^*(S)=\mu^*(S\cap M) + \mu^*(S\setminus M)
\end{equation*}
für alle $S\subset X$ gilt. \footnotesize{$S$ ist also eine ``Testmenge''.}
\end{mdframed}
\end{definition}
\textit{Vorstellungen:} $M$ und $\complement M$ ``durchdringen sich nicht''. $M$ hat ``dünnen Rand''. Jede Teilmenge $S$ wird durch ``sauberen Schnitt'' durch $M$ geteilt.\\\\
Wir setzen
$$\AA^*:=\{\mu^*\text{-messbare Teilmengen}\}\subset \PP(X).$$
\begin{small}
\begin{remark}
Ältere Ansätze (Jordan, Peano) arbeiten noch mit \emph{endlichen} Überdeckungen durch Teilmengen aus $\HH$, bzw. äquivalent, \emph{Einschließungen} durch Teilmengen in $\RR$. Das äußere Maß wird also so definiert:
\begin{equation*}
\begin{split}
\mu^{*,\text{fin}}&:= \inf \left\{ \sum_{i=1}^n \mu(A_i)\  \big\vert \ A_1,...,A_n \in \HH, \text{ mit } A_1\cup...\cup A_n \supset M\right\} \\
 &= \inf \left\{ \ol\mu(\ol{A}) \mid \ol{A} \in \RR \text{ mit } \ol{A} \supset M \right\}
\end{split}
\end{equation*}
$\mu^{*,\text{fin}}:\PP(X)\longrightarrow [0,\infty]$ ist wieder monoton, aber nur \emph{endlich} additiv. Es ist $\mu^{*,\text{fin}}|_{\RR}=\ol{\mu}$ und somit $\mu^{*,\text{fin}}\geq \mu^*$. Deswegen ist $\mu^{*,\text{fin}}$ nur nützlich auf kleineren Klassen von Teilmengen. Das innere Maß wird definiert durch
\begin{equation*}
	\mu_{*,\text{fin}}:=\sup \{\ol\mu(I) \mid I \in \RR \text{ mit } I \subset M\}
\end{equation*}
Also schöpft man $M$ durch Teilmenge $I\in \RR$ aus. Dies ist wiederum unflexibler, denn:
$$
\mu_{*,\text{fin}}\leq \mu_* (\leq \mu^* \leq \mu^{*,\text{fin}}).
$$
Dies führt zur einer größeren Diskrepanz $\mu^{*,\text{fin}}-\mu_{*,\text{fin}}\geq \mu^*-\mu_*$ und deswegen sind weniger Teilmengen $\mu^{*,\text{fin}}$-messbar (bzw. Jordan-messbar, Jordan-quadrierbar).\\
Die ältere Version der Theorie bleibt unbefriedigend, denn z.B. betrachten wir $\Q \cap [0.1]$ dicht in $[0,1]$ und $\lambda^1_\II$ (euklidische Länge $b-a$) auf $\II$ (Halbring halboffener Intervalle $[a,b)$). Es gilt $(\lambda_\II^1)^{*,\text{fin}}(\Q \cap [0,1])=1$, weil jede Teilmenge aus $\FF^1$, die $\Q \cap [0,1]$ enthält, auch $[0,1]$ enthält. Anderseits gilt $(\lambda^1_\II)^*(\Q \cap [0,1]) =0$, da es abzählbare Überdeckung durch abzählbare viele Teilintervalle beliebig kleiner Gesamtlänge gibt. Es folgt $\lambda_{*,\text{fin}}(\Q\cap[0,1])\leq \lambda_*(\Q\cap[0,1]) \leq \underbrace{\lambda^*(\Q\cap[0,1])}_{=0}\leq \underbrace{\lambda^{*,\text{fin}}(\Q\cap[0,1])}_{=1}$. Also ist $\Q\cap[0,1]$ nicht $(\lambda^1_{\II})^{*,\text{fin}}$-messbar, aber $(\lambda^1_\II)^*$-messbar.
\end{remark}
\end{small}

\subsubsection{Die $\sigma$-Algebra der messbaren Mengen und ihr Maß}
Sei $\AA^*\subset \PP(X)$ die Familie der $\mu^*$-messbaren Teilmengen. ($\mu^*$ nicht notwendigerweise induziert von einem Inhalt, fordere nur Axiome eines äußeren Maßes)

\begin{definition}
\begin{mdframed}
$N \subset X$ heißt \textbf{$\boldsymbol\mu^{\boldsymbol*}$-Nullmenge}, falls $\mu^*(N)=0$.
\end{mdframed}
\end{definition}

\begin{lemma}
\begin{mdframed}
Alle $\mu^*$-Nullmengen gehören zu $\AA^*$.
\end{mdframed}
\begin{proof}
Für $S\subset X$ gilt:
$$
\mu^*(S) \overset{\text{Subadd.}}\leq \underbrace{\mu^*(\underbrace{S\cap N}_{\subset N})}_{\overset{\text{Monot.}}\leq \mu^*(N)=0}+\underbrace{\mu^*({\underbrace{S\setminus N}_{\subset S}})}_{\leq \mu^*(S)} \leq \mu^*(S)
$$
Also Gleichheit in allen Ungleichungen. Also $N$ $\mu^*$-messbar.
\end{proof}
\end{lemma}

\begin{beobachtung}
$\AA^*$ ist $\complement$-stabil, d.h. $M \in \AA^* \implies \complement M \in \AA^*$, denn ``Rand von $M$ = Rand von $\complement M$''.
\end{beobachtung}

Äußere Maße sind i.A. nicht additiv, d.h. sie sind keine Inhalte auf $\PP(X)$, sondern nur partiell additiv (vgl. oben). Es ist aber plausibel zu erwarten, dass sie $\sigma$-additiv auf Teilmengen mit ``dünnem Rand'' sind, da keine ``Volumendurchdringung''.

\begin{satz}[\textbf{Carathéodory}]
\begin{mdframed}
$\AA^*$ ist eine $\sigma$-Algebra und $\mu^*|_{\AA^*}$ ist ein Maß.
\end{mdframed}
\end{satz}

\begin{lemma}\
\begin{enumerate}[label=(\roman*),topsep=3pt, itemsep=0pt]
\item Sind $M_1,...,M_n \subset X$ Teilmengen (nicht notwendigerweise disjunkt) mit $\mu^*(M_1\cup...\cup M_n)=\mu^*(M_1)+...+\mu^*(M_n)$, so auch $\mu^*(M_1 \cup ... \cup M_k) = \mu^*(M_1) + ... +\mu^*(M_k)$ für $1 \leq k <n$.
\item Sind $M_1,...,M_n \in \AA^*$ paarweise disjunkt und $S\subset X$, so gilt $\mu^*(S\cap(M_1 \cup ... \cup M_n)) = \mu^*(S \cap M_1) + ... +\mu^*(S \cap M_n)$.
\end{enumerate}
\begin{proof}
\begin{enumerate}[label=(\roman*),topsep=3pt, itemsep=0pt]
	\item Subadditivität von $\mu^*$ liefert
	\begin{equation*}
	\mu^*\left( \bigcup_{i=1}^n M_i \right) \overset{\text{nV}}= \sum_{i=1}^n \mu^*(M_i)= \underbrace{\sum_{i\leq k} \mu^*(M_i)}_{\geq \mu^*\left(\bigcup\limits_{i\leq k} M_i\right)}+\underbrace{\sum_{i>k}\mu^*(M_i)}_{\geq \mu^*\left(\bigcup\limits_{i> k} M_i\right)} \geq \mu^*\left( \bigcup_{i=1}^n M_i \right)
	\end{equation*}
	Also Gleichheit überall und insbesondere gilt $\sum\limits_{i\leq k} \mu^*(M_i)=\mu^* \left(\bigcup\limits_{i\leq k} M_i \right)$
	\item Für $1 \leq k \leq n$ gilt, da $M_i$ disjunkt:
	\begin{equation*}
		\begin{split}
			(S\cap (M_k \cup ... \cup M_n))\cap M_k &= S \cap M_k \\
			(S\cap (M_k \cup ... \cup M_n))\cap \complement M_k &= S \cap (M_{k+1}\cup...\cup M_n)
		\end{split}
	\end{equation*}
	Setze nun die Testmenge $S \cup (M_k \cup ... \cup M_n)$, und weil $M_k \in \AA^*$ ist, folgt nach der Definition der $\mu^*$-Messbarkeit:
	$$\mu^*(S \cup (M_k \cup ... \cup M_n)) = \mu^*(S \cap M_k) + \mu^*(S \cap (M_{k+1}\cup...\cup M_n))$$
	Induktion liefert die Behauptung.
\end{enumerate}
\end{proof}
\end{lemma}
\textbf{Beweis des Satzes von Carathéodory.} Zuerst zeigen wir, dass $\AA^*$ $\cup$-stabil ist: Hierzu seinen $M_1, M_2 \in \AA^*$. Sei $S \subset X$ beliebig. Dann folgt
\begin{equation*}
	\begin{split}
	\mu^*(S) & \underset{\text{messbar}}{\overset{M_1}=} 
	\underbrace{\mu^*(S \cap M_1)}_{ \substack{=
	\mu^*(S\cap M_1 \cap M_2) + \mu^* (S \cap M_1 \cap \complement M_2)\\\text{da } M_2 \text{ messbar}} 		} 
	+ \underbrace{\mu^* (S \cap \complement M_1)}_{=\mu^*(S\cap\complement M_1 \cap M_2) +\mu^*(S\cap \complement M_1 \cap \complement M_2)} \\
	& =\mu^*(S\cap M_1 \cap M_2) + \mu^* (S \cap M_1 \cap \complement M_2)+\mu^*(S\cap\complement M_1 \cap M_2) +\mu^*(S\cap \complement M_1 \cap \complement M_2)
	\end{split}
\end{equation*}
Das Lemma (i) liefert, dass \Bild
$$
\mu^*(S\cap (M_1 \cup M_2)) = \mu^*(S\cap M_1 \cap M_2) + \mu^* (S \cap \complement M_1 \cap M_2) + \mu^*(S\cap M_1 \cap \complement M_2).
$$
Daraus folgt
$$
\mu^*(S)= \mu^*(S\cap(M_1\cup M_2))+\mu^*(S\cap \complement (M_1 \cup M_2))
$$
Also $M_1 \cup M_2 \in \AA^*$, also ist $\AA^*$ eine Algebra. Zu zeigen bleibt: $\AA^*$ ist $\sigma$-$\cup$-stabil. Es genügt, für abzählbare disjunkte Vereinigungen zu zeigen (da $\AA^*$ eine Algebra). Sei $(M_n)_{n \in \N}$ eine Folge in $\AA^*$ mit paarweise disjunkten Folgengliedern. Sei $S \subset X$. Setze $\ol{M}_{n}=M_1 \cup ... \cup M_n$ und $\ol{M}_{\infty} = \bigcup\limits_{i\in\N}M_i$. Für alle $n \in \N$ gilt:
\begin{equation*}
\mu^*(S)=\overbrace{\mu^*(\underbrace{S\cap \ol{M}_n}_{\nearrow S \cap \ol{M}_\infty})}^{ \underset{(\text{ii})}{\overset{\text{\tiny Lemma}} {=}}\sum\limits_{i=1}^n\mu^*(S\cap M_i)} + \underbrace{\mu^*(\underbrace{S\cap \complement \ol{M}_n}_{\searrow S \cap \complement \ol{M}_\infty})}_{\geq \mu^*(S\cap \complement \ol{M}_\infty)}
\end{equation*}
Es folgt für $n\longrightarrow \infty$:
\begin{equation*}
	\mu^*(S) \geq \underbrace{\sum_{i=1}^\infty \mu^*(S \cap M_i)}_{\geq \mu^*(S\cap \ol{M}_\infty)} + \mu^*(S\cap \complement \ol{M}_\infty) \geq \mu^*(S)
\end{equation*}
Also überall Gleichheit, d.h.
\marginpar{\tiny{14.11.2019}}
\begin{equation} \label{I.8}
\mu^*(S) = \sum_{n=1}^\infty \mu^*(S\cap M_n) + \mu^*(S\cap \complement \ol{M}_\infty) =\mu^*(S\cap \ol{M}_\infty) + \mu^* (S \cap \complement \ol{M}_\infty)
\end{equation}
dies zeigt, dass $\ol{M}_\infty \in \AA^*$, d.h. $\AA^*$ eine $\sigma$-Algebra. Setze nun $S= \ol{M}_\infty$ in \eqref{I.8}, dann folgt
\begin{equation*}
\mu^*(M_\infty)=\sum_{i=1}^\infty \mu^*(\underbrace{\ol{M}_\infty \cap M_i}_{M_i}),
\end{equation*}
also $\mu^*$ $\sigma$-additiv, also $\mu^*|_{\AA^*}$ Maß. \hfill $\qed$

\subsubsection{Beziehung eines Inhalts zu seinem assoziierten äußerem Maß}
Wir betrachten folgende Situation:
\begin{equation*}
\begin{array}{ccccccccc}
\underset{\text{Halbring}}{\HH} & \subset & \underset{\text{von } \HH \text{ erz. Ring}}{\RR} & \subset &\sigma(\RR)& \subset & \underset{\substack{\sigma \text{ Algebra der} \\ \mu^*\text{-messb. Mengen}}}{\AA^*} & \subset & \PP(X)\\
\mu \text{ Inhalt} && \underset{\text{eind. Forts. von }\mu}{\ol\mu \text{ Inhalt}} &&&& \underset{\mu^* \text{ induziert von } \mu}{\mu^*|_{\AA^*} \text{ Maß}} \\
&& \mu^*|_\RR \leq \ol\mu
\end{array}
\end{equation*}
\begin{beobachtung}
Es gilt $\RR \subset \AA^*$, d.h. für $R \in \RR$ und $S \subset X$ gilt
$$
\mu^*(S) = \mu^*(\underbrace{R \cap S}_{\subset R \in \RR}) + \mu^* \underbrace{(R \cap \complement S)}_{\text{disjunkt zu } R}
$$
Dies folgt unmittelbar aus partieller Additivität von $\mu^*$.
\end{beobachtung}
Aus dem Satz von Carathéodory folgt unmittelbar folgendes Korollar:
\begin{korollar}
\begin{mdframed}
$\sigma(\RR) \subset \AA^*$ und $\mu^*|_{\sigma (\RR)}$ ist Maß (Insbesondere ist $\mu^*|_\RR$ ein Prämaß). Dies ist notwendig dafür, dass $\mu^*$ eine Fortsetzung von $\ol\mu$ ist, dass $\ol\mu$ Prämaß.
\end{mdframed}
\end{korollar}

\subsubsection{Fortsetzung von Prämaßen zu Maßen}
\begin{lemma}
\begin{mdframed}
Sei $\mu$ ein Inhalt auf einem Ring $\RR$. Ist $\mu$ ein Prämaß, so gilt $\mu^*|_\RR = \mu$.
\end{mdframed}
\begin{proof}
Sei $A \in \RR$ und $(A_n)_{n \in \N}$ eine abzählbare Überdeckung von $A$ durch Teilmengen in $\RR$. Setze $\ol{A}_n:=A_1 \cup ... \cup A_n \in \RR$. Da $\mu$ Inhalt ist, folgt
$$
\sum_{i=1}^n \mu(A_i) \geq \mu (\ol{A}_n) \overset{\text{Monot.}}\geq  \overbrace{\mu(\underbrace{\ol{A}_n \cap A}_{\nearrow A})}^{\leq \mu(A)}
$$
Für $n \longrightarrow \infty$ gilt
$$
\sum_{i=1}^\infty \mu(A_i)\geq \lim_{n \to \infty} \mu(\ol{A}_n) = \lim_{n \to \infty} \mu(\ol{A}_n \cap A) \overset{\mu \text{ Prämaß}}= \mu(A)
$$
Jetzt nehmen wir Infimum über $(A_n)$, so folgt
$$ \mu^*(A) \geq \mu(A),$$
also $\mu^*|_\RR \geq \mu$, somit Gleichheit.
\end{proof}
\end{lemma}

\begin{satz}
\begin{mdframed}
Ist $\mu$ ein Prämaß auf dem Ring $\RR$, so wird $\mu$ durch das assoziierte äußere Maß $\mu^*$ zu einem Maß $\mu^*|_{\sigma(\RR)}$ auf dem von $\RR$ erzeugten $\sigma$-Algebra $\sigma(\RR)$ fortgesetzt.
\end{mdframed}
\begin{proof}
Aus $\RR \subset \AA^*$ folgt $\sigma(\RR)\subset \AA^*$, da $\AA^*$ eine $\sigma\text{-Algebra}$ ist. Da $\mu^*|_{\AA^*}$ ein Maß ist, ist $\mu^*|_{\sigma(\RR)} = (\mu^*|_{\AA^*})|_{\sigma(\RR)}$ auch ein Maß. Das obige Lemma liefert dann, dass $\mu^*|_{\sigma(\RR)}$ den Inhalt $\mu$ fortsetzt.
\end{proof}
\end{satz}
Dies zeigt, dass Prämaße auf Ringen \emph{auf mindestens eine Weise} \emph{fortsetzbar zu Maßen} auf den von den Ringen erzeugten $\sigma$-Algebren. 
\subsubsection{Eindeutigkeit fortgesetzter Maße}
Im folgenden betrachten wir
\begin{equation*}
\begin{array}{ccccccc}
\RR	&	 \subset		&	 \BB 		&	\subset		&	 \AA^* 	&	\subset		&	 \PP(X)\\
\text{Ring} & &\text{eine }\sigma\text{-Alg}	&& \substack{\sigma\text{-Alg der} \\ \mu^*\text{-messb. Mengen}}
\end{array}
\end{equation*}
Weiter sei $\nu$ ein Maß, der das Prämaß $\mu$ fortsetzt. Es ist dann natürlich, $\mu^*|_\BB$ mit $\nu$ zu vergleichen.

\begin{lemma}
Es gilt $\boxed{\nu \leq \mu^*|_\BB}$.
\begin{proof}
Sei $M\in \BB$ und $(A_n)_{n \in \N}$ eine abzählbare Überdeckung von $M$, $A_n \in \RR$, $M \subset \bigcup\limits_{n\in \N} A_n$. Dann gilt
$$
\sum_{n \in \N} \mu(A_n) \overset{\nu \text{ Forts.}}= \sum_{n \in \N} \nu(A_n) \overset{\nu \text{ Maß}}\geq \nu \left( \bigcup_{n \in \N} A_n \right)  \overset{\text{monot.}}\geq \nu (M)  
$$
Bilde nun Infimum über $(A_n)$, es folgt dann
$$
\mu^*(M) \geq \nu (M)
$$
\end{proof}
\end{lemma}

\begin{satz}
\begin{mdframed}
Auf der Familie der Teilmengen aus $\BB$, die eine abzählbare Überdeckung durch Teilmengen aus $\RR$ mit endlichem $\mu$-Inhalt zulassen, stimmt $\nu$ mit $\mu^*$ überein.
\end{mdframed}
\begin{proof}
Sei $M \in \BB$. Zunächst zeigen wir unter der Annahme: $\exists R \in \RR$ mit $M \subset R$ und $\mu(R)< \infty$. Unter dieser Annahme gilt
$$
\mu^*(M)+\mu^*(R\setminus M)\overset{\mu^* \text{ Maß}}=\mu^*(R) \overset{\mu \text{ Prämaß}}= \mu (R) \overset{\nu|_{\RR}=\mu} = \nu (R) \overset{\nu \text{ Maß}}= \underbrace{\nu(M)}_{\leq \mu^*(M)}+\underbrace{\nu(R \setminus M)}_{\leq \mu^*(R\setminus M)}
$$
Also muss es gelten $\nu(M)= \mu^*(M)$. \\
Jetzt unter der allgemeinen Annahme: $M$ hat eine abzählbare Überdeckung durch $R_n \in \RR$ mit $\mu(R_n)< \infty$. OBdA dürfen wir annehmen, dass alle $R_i$ paarweise disjunkt sind, da $\RR$ ein Ring ist. Wir haben schon gezeigt: $\nu(\underbrace{M \cap R_n}_{\in \BB}) = \mu^*(M\cap  R_n).$
Da $\nu$ und $\mu^*$ Maße sind, folgt
$$
\nu\bigg( \underbrace{\bigcup_{n \in \N} (M \cap R_n)	}_{=M}	\bigg) = \mu^* \bigg( \underbrace{\bigcup_{n \in \N} (M \cap R_n)}_{=M} \bigg)
$$
\end{proof}
\end{satz}
Dieser Satz zeigt, dass wir Eindeutigkeit unter geeigneten Endlichkeitsvoraussetzungen haben.\footnote{Nichteindeutigkeit schon in einfachen Situationen: Sei $\RR = \{\emptyset\}$. Dann ist $\sigma(\RR)=\{\emptyset, X\} $. Ist $\mu$ ein endlicher Inhalt auf $\RR$, so können wir den Inhalt von $X$ beliebig wählen. Das Problem besteht darin, dass $\mu$ nicht $\sigma$-endlich ist.}

\begin{definition}
\begin{mdframed}
Ein Inhalt $\mu$ auf einem Ring $\RR \subset \PP(X)$ heißt 
\begin{enumerate}[label=(\roman*),topsep=3pt, itemsep=0pt]
	\item \textbf{endlich}, falls $X \in \RR$ mit $\mu(X) < \infty$.
	\item \textbf{$\boldsymbol\sigma$-endlich}, falls $X$ eine abzählbare Überdeckung durch Teilmengen aus $\RR$ mit endlichem $\mu$-Inhalt zulässt.
\end{enumerate}
\end{mdframed}
\end{definition}

\begin{korollar}
\begin{mdframed}
Ein $\sigma$-endliches Prämaß auf einem Ring $\RR \subset \PP(X)$ besitzt eine eindeutige Fortsetzung zu einem Inhalt auf $\sigma(\RR)$.
\end{mdframed}
\end{korollar}

\begin{definition}
\begin{mdframed}
Da das Lebesgue-Prämaß $\lambda^d_{\FF^d}$ auf $\FF^d\subset \PP(\R^d)$ $\sigma$-endlich ist, können wir es eindeutig fortsetzen zu einem Maß $\beta^d$ auf der Borelschen $\sigma$-Algebra $\BB^d = \sigma(\FF^d)$. Wir nennen $\beta^d$ \textbf{das Lebesgue-Borel-Maß}. Wir dehnen dieses Maß weiter aus auf alle (bzgl. des äußeren Maßes) messbaren Teilmengen $(\beta^d)^* \overset{\text{\Ueb}}= (\lambda^d_{\FF^d})^*$. Wir definieren $\BB^d \subset \boldsymbol{\LL^d} \subset \PP(\R^d)$ als \textbf{die $\boldsymbol\sigma$-Algebra der $\boldsymbol{(\beta^d)^*}$-messbaren Teilmengen.} Das Maß $(\beta^d)^*|_{\LL^d} =: \boldsymbol{\lambda^d}$ heißt \textbf{$\boldsymbol{d}$-dimensionale Lebesgue-Maß}.
\end{mdframed}
\end{definition}

Zusammengefasst:
\begin{equation*}
\begin{array}{ccccccc}
\FF^d	& \subset & 
	\underset{\text{Borel-}\sigma\text{-Algebra}}{\BB^d=\sigma(\FF^d)} & \subset  & \underset{(\lambda^d_{\FF^d})^*\text{-messb. Teilmengen}}{\LL^d}	& \subset & \PP(\R^d) \\ 
\underset{\text{Lebesgue-Prämaß}}{\lambda^d_{\FF^d}} & &
\underset{\text{Lebesgue-Borel-Maß}}{\beta^d := \underbrace{(\lambda^d_{\FF^d})^*}_{\lambda^d}|_{\BB^d}} &&
\underset{\text{Lebesgue-Maß}}{\lambda^d := (\lambda^d_{\FF^d})^*|_{\LL^d}}
\end{array}
\end{equation*}

\subsubsection{Approximation messbarer Teilmengen}
Sei $\mu$ ein Prämaß auf dem Ring $\RR \subset \PP(X)$, $\sigma(\RR)$ die von $\RR$ erzeugte $\sigma$-Algebra und $\AA^*$ die Familie der $\mu^*$-messbaren Teilmengen. Wir wollen nun $\sigma(\RR)$ mit $\AA^*$ vergleichen. Erster Erinnerung: $\mu^*$-Nullmengen sind $\mu^*$-messbar, liegen also in $\AA^*$.\\\\
\textit{Sprechweise:} Es seien $M,M'\subset X$ mit $\mu^*(M \triangle M') = 0$. Wir nennen dann $M, M'$ \textbf{Modifikationen} voneinander durch $\mu^*$-Nullmengen. Wir sagen, dass sie ``bis auf $\mu^*$-Nullmengen übereinstimmen''.

\begin{lemma} \label{uebereinstimmen}
\begin{mdframed}
Für $M, M' \subset X$ mit $\mu^*(M \triangle M') =0$ gilt:
\begin{enumerate}[label=(\roman*),topsep=3pt, itemsep=0pt]
	\item $\mu^*(M) = \mu^*(M')$,
	\item $M \in \AA^* \iff M' \in \AA^*$.
\end{enumerate}
\end{mdframed}
\begin{proof}
Zu (i): Aus $M \cup M' = (M \cap M') \cup (M \triangle M')$ folgt
$$
\mu^*(M \cup M' ) \leq \mu^*(M \cap M') + \underbrace{\mu^* (M \triangle M')}_{=0}=\mu^*(M\cap M')
$$ 
Also $\mu^*(M\cup M') = \mu^*(M\cap M')$. Wegen $\mu^*(M\cap M') \leq \mu^*(M) \leq \mu^*(M \cup M')$ folgt Gleichheit überall. Insbesondere gilt $\mu^*(M)=\mu^*(M')$.\\
Zu (ii): Wir erkennen zunächst $M \triangle M' = \complement M \triangle \complement M'$. Sei $S \subset X$. Dann gilt
$$
(S\cap M) \triangle (S \cap M') = S\cap (M \triangle M') \subset M \triangle M'
$$
Es folgt
$$\mu^*((S\cap M)\triangle (S \cap M'))=0$$
und analog für die Komplemente: $\mu^*((S\cap \complement M)\triangle (S \cap \complement M'))=0$.
Es folgt (wegen (i))
$$
\mu^*(S\cap M) + \mu^*(S\cap \complement M) =\mu^*(S \cap M') + \mu^*(S \cap \complement M')
$$
Dies zeigt, dass $\mu^*(S\cap M) + \mu^*(S\cap \complement M) = \mu^*(S)$ genau dann gilt, wenn $\mu^*(S \cap M') + \mu^*(S \cap \complement M')=\mu^*(S)$. Also $M \in \AA^*$ genau dann wenn $M'\in \AA^*$.
\end{proof}
\end{lemma}
\marginpar{\tiny{18.11.2019}}

Wir wollen jetzt (angelehnt an Archimedes-Kompressionsmethode) beliebige Teilmengen von innen/außen im Volumen-Sinne durch Teilmengen aus $\sigma(\RR)$ (möglich ``nah verwandt\footnote{im Sinne der Borel-Hierarchie}'' mit $\RR$) approximieren. Für \emph{messbare} Teilmengen werden wir Approximationen bis auf $\mu^*$-Nullmengen erhalten.

\begin{satz}
\begin{mdframed}
Sei $M\subset X$ mit $\mu^*(M)<\infty$. Dann existieren Teilmengen $I, A\in \sigma (\RR)$ mit $I \subset M \subset A$ sodass $\mu^*(A) = \mu^*(M)$ und $\mu^*(A\setminus I) = \mu^*(A \setminus M)$.\tablefootnote{$I$ steht für innen, $A$ steht für außen.} Diese sind eindeutig bis auf Modifikationen durch $\mu^*$-Nullmengen. Es gilt außerdem: $M \in \AA^* \iff \mu^*(A\setminus I)=0$.\tablefootnote{$M$ ist genau dann $\mu^*$-messbar, wenn $M$ bis auf $\mu^*$-Nullmengen mit der inneren bzw. äußeren Approximation aus $\sigma(\RR)$ übereinstimmt.}
\end{mdframed}
\begin{proof}\
\begin{itemize}
\item {Die \emph{äußere} Approximation:} Aus der Definition von $\mu^*$ folgt, dass es abzählbare Überdeckungen von $M$ durch Teilmengen aus $\RR$ existieren, deren Gesamtvolumen $\mu^*(M)$ beliebig gut approximiert. D.h. Zu jedem $\varepsilon>0$ existiert ein $A_\varepsilon \in \sigma(\RR)$ sodass $M\subset A_\varepsilon$ und $\mu^*(A_\varepsilon) < \mu^*(M)+\varepsilon$. Wir wählen $$A:=\underbrace{\bigcap_{n\in\N}A_{\frac{1}{n}}}_{\supset M} \in \sigma(\RR).$$
Es erfüllt $\mu^*(A)=\mu^*(M)$.
\item Die \emph{innere} Approximation geschieht durch die äußere Approximation von $A\setminus M$. Weil $\mu^*(A\setminus M) \leq \mu^*(M) < \infty$, können wir $A\setminus M$ von außen approximieren. Nämlich gibt es ein $B\in \sigma(\RR)$ so, dass $A\setminus M \subset B$ und $\mu^*(B)=\mu^*(A\setminus M)$. Wir wählen $I:= A \setminus B \subset M$. Weil $ A\setminus I =B$, folgt $\mu^*(A\setminus I) =\mu^*(A \setminus M)$.\\
Das obige Argument zeigt die Existenz der Approximationen.
\item Zur \emph{Eindeutigkeit} bis auf Nullmengen: Sei $A' \in \sigma (\RR)$ mit $M\subset A'$ und $\mu^*(A')=\mu^*(M)$ eine weitere äußere Approximation. Dann gilt $\mu^*(A \cap A') = \mu^*(M) =\mu^*(A) =\mu^*(A')$. Weil $A,A',A\cap A'$ $\mu^*$-messbar sind und endliche Volumina haben, folgt $\mu^*(A \triangle A') = \mu^*(A)+\mu^*(A')-2\mu^*(A\cap A') = 0$. Dies zeigt, dass die äußere Approximation eindeutig bis auf Nullmengen ist. (Analog für die innere Approximation.)
\item Ist $M \in \AA^*$, so gilt $\mu^*(A\setminus I) = \mu^*(A\setminus M) \underset{
\text{endl. Vol.}}{\overset{M \in \AA^*}=} \mu^*(A)-\mu^*(M)=0$. \\
 Ist umgekehrt $\mu^*(A\setminus I)=0$, so gilt $\mu^*(A\triangle I)=0$ (weil $I \subset A$)  und somit stimmen $I,M,A$ bis auf Nullmengen überein. Weil $I,A \in \AA^*$, folgt wegen \hyperref[uebereinstimmen]{des obigen Lemmas} auch $M \in \AA^*$.
\end{itemize}
\end{proof}
\begin{remark}\
%\begin{enumerate}[label=(\roman*)]
%\item
 Die Aussage des Satzes gilt allgemeiner für Teilmengen $M\subset X$, die abzählbare Vereinigungen von Teilmengen endlichen $\mu^*$-Volumens sind. Begründung: Sei $M=\bigcup\limits_{n\in\N}M_n$ mit $\mu^*(M_n)<\infty$. oBdA sei $M_n$ paarweise disjunkt. Der obige Satz liefert, dass es $I_n, A_n \in \sigma(\RR)$ existieren mit $I_n \subset M_n \subset A_n$ und $\mu^*(A_n \setminus I_n)=0$. Die gewünschten Approximationen folgen durch Bildung der Vereinigungen $\bigcup\limits_{n \in \N}I_n \subset M \subset \bigcup\limits_{n\in\N}A_n$, denn aus
$$\bigg( 
\bigcup\limits_{n\in\N} A_n
\bigg) \setminus \bigg(
\bigcup\limits_{n\in \N} I_n
\bigg) =
\bigcup_{n \in \N} \bigg(\underbrace{A_n \setminus \bigcup_{m\in\N}I_m}_{\subset A_n \setminus I_n} \bigg) \subset \bigcup_{n \in \N} (A_n \setminus I_n)$$
folgt $\mu^*\bigg( \bigg( \bigcup\limits_{n\in \N}A_n \bigg)\setminus \bigg( \bigcup\limits_{n \in \N} I_n \bigg)  \bigg)=0$.
%\end{enumerate}
\end{remark}
\end{satz}

Für \emph{Struktur} von \emph{messbaren Teilmengen} folgt:
\begin{korollar}\
\begin{mdframed}
\begin{enumerate}[label=(\roman*)]
\item Jede $\mu^*$-Nullmenge wird von einer $\mu^*$-Nullmenge aus $\sigma(\RR)$ eingeschlossen.
\item Lässt $M \in \AA^*$ eine abzählbare Überdeckung durch (oBdA messbare) Teilmengen endlichen $\mu^*$-Inhalts zu, so ist $M$ die Vereinigung einer Teilmenge\footnote{wähle z.B. die innere Approximation} aus $\sigma(\RR)$ mit einer $\mu^*$-Nullmenge.
\end{enumerate}
\end{mdframed}
\end{korollar}

\begin{korollar} \label{uebereinstimmung-alg}
\begin{mdframed}
Für ein $\sigma$-endliches Prämaß $\mu$ auf einem Ring $\RR \subset \PP(X)$ stimmen alle $\mu^*$-messbaren Teilmengen bis auf Modifikationen durch $\mu^*$-Nullmengen mit Teilmengen aus $\sigma(\RR)$ überein.\footnote{Wir können es so interpretieren, dass die Kompliziertheit messbarer Teilmengen im Vergleich zu Teilmengen aus $\sigma(\RR)$ von Nullmengen ``aufgefangen'' wird. Letztere sind oft vernachlässigbar.}
\end{mdframed}
\end{korollar}

Wir betrachten nun speziell \emph{Lebesgue-messbare} (d.h. bezüglich des äußeren Maßes $(\lambda^d_{\FF^d})^* = (\beta^d)^* = (\lambda^d)^*$) Teilmengen in $\R^d$. Das letzte Korollar liefert einen Vergleich von Borel- und Lebesgue-Messbarkeit für Teilmengen des $\R^d$.

\begin{korollar}
\begin{mdframed}
Jede Lebesgue-messbare Teilmenge stimmt bis auf eine Lebesgue-Nullme-nge (d.h. $(\lambda^d)^*(N)=0$) überein mit einer Borel-messbaren Teilmenge. (Wir nennen eine Borel-messbare Teilmenge kurz eine \textbf{Borel-Menge}.)
\end{mdframed}
\end{korollar}

Um eine bessere geometrische Vorstellung von messbaren Teilmengen des $\R^d$ zu bekommen, vergleichen wir jetzt die \emph{messbare} und \emph{topologische} Struktur auf $\R^d$, d.h. die $\sigma$-Algebra $\BB^d$ mit einem geometrisch natürlichen Erzeuger, nämlich der \emph{Topologie}.

\begin{definition}
\begin{mdframed}
Eine Teilmenge eines topologischen Raums heißt
\begin{enumerate}[label=(\roman*),topsep=3pt, itemsep=0pt]
\item $G_\delta$-Menge, falls sie ein abzählbarer Durchschnitt offener Mengen ist.
\item $F_\sigma$-Menge, falls sie eine abzählbare Vereinigung abgeschlossener Mengen ist.
\end{enumerate}
\end{mdframed}
\end{definition}

\paragraph{Topologische Approximation einer messbaren Teilmenge $M\in \LL^d \subset \PP(\R^d)$}
\begin{itemize}
\item \textbf{1. Schritt:} Sei $M$ beschränkt, d.h. $M\subset B$, wobei $B$ ein offener Ball ist. Es gilt $\lambda^d(M)<\infty$. Weil $M \in \LL^d$, existiert zu jedem $\varepsilon>0$ eine abzählbare Überdeckung von $M$ durch Teilmengen aus $\FF^d$ mit Gesamtvolumen $< \lambda^d(M) +\varepsilon$. Durch offene Aufdickung von Quadern bzw. Figuren existiert ebenfalls ein $O_\varepsilon$ offen mit $$M \subset O_{\varepsilon} \subset B \text{\ \  und \ } \lambda^d(O_\varepsilon) < \lambda^d(M)+\varepsilon.$$ 
Analog (durch äußere Approximation von $\ol{B}\setminus M$) existiert ein $O'_\varepsilon$ offen sodass $$\ol{B}\setminus M\subset O'_\varepsilon \text{\ \  mit \ } \lambda^d(O'_\varepsilon)<\lambda^d(\ol{B}\setminus M)+\varepsilon\overset{M \text{ messbar}}= \lambda^d(\ol{B})-\lambda^d(M)+\varepsilon.$$ 
Es ist $$\ol{B}\subset O_\varepsilon \cup O'_\varepsilon \text{\ \  mit \ } \lambda^d(O_\varepsilon)+\lambda^d(O'_\varepsilon)<\lambda^d(\ol{B})+2\varepsilon.$$
Wir wählen $$K_\varepsilon := \ol{B} \setminus O'_\varepsilon \subset M.$$ Dann ist $K_\varepsilon$ eine kompakte \emph{innere} Approximation, denn $$\lambda^d(K_\varepsilon)=\lambda^d(\ol{B})-\lambda^d(O'_\varepsilon)>\lambda^d(O'_\varepsilon)-2\varepsilon.$$
Also $K_\varepsilon \subset M \subset O_\varepsilon$ mit $\lambda^d(O_\varepsilon \setminus K_\varepsilon) <2\varepsilon$.
\item \textbf{2. Schritt:} Jetzt sei $M\in\LL^d$ beliebig. Dann gibt es eine Darstellung $M=\bigcup\limits_{n\in \N}M_n$ mit $M_n \in \LL^d$ beschränkt, sodass $(M_n)$ eine \emph{lokal endliche} Familie bildet, d.h. jeder Ball in $\R^d$ schneidet nur endlich viele $M_n$. Der erste Schritt liefert, dass es Approximationen $\underset{\text{kompakt}}{K_n} \subset M_n \subset \underset{\text{offen}}{O_n}$ für jedes $M_n$ gibt sodass $\lambda^d (O_n \setminus K_n) < \varepsilon_n$. Die $\varepsilon_n$ sind positiv beliebig vorgebbar. Durch die Vereinigung erhalten wir ${\bigcup\limits_{n \in \N} K_n} \subset M \subset \underset{\text{offen}}{\bigcup\limits_{n \in \N} O_n}$. Weil $(K_n)$ lokal endlich sind und der Schnitt einer abgeschlossenen Menge mit jedem Kompaktum wieder kompakt ist, ist $\bigcup\limits_{n\in \N} K_n$ abgeschlossen. Es gilt
$$
\lambda^d\bigg( \underbrace{\bigg(\bigcup_{n\in\N}O_n \bigg) \setminus \bigg( \bigcup_{n\in\N}K_n\bigg)}_{\subset \bigcup\limits_{n \in \N}(O_n \setminus K_n)}\bigg) < \underset{\substack{\text{positiv beliebig}\\\text{klein wählbar}}}{\sum_{n \in \N}\varepsilon_n}
$$
\end{itemize}

Dies zeigt:

\begin{satz}
\begin{mdframed}
Sei $M \in \LL^d$. Dann existiert zu jedem $\varepsilon>0$ $A_\varepsilon, O_\varepsilon \subset \R^d$ mit $A_\varepsilon$ abgeschlossen und $O_\varepsilon$ offen, sodass
${A_\varepsilon} \subset M \subset {O_\varepsilon}$
und 
$\lambda^d(O_\varepsilon \setminus A_\varepsilon) < \varepsilon$.
Falls $\lambda^d(M)<\infty$, kann $A_\varepsilon$ kompakt gewählt werden.
\end{mdframed}
\end{satz}

Es folgt:
\begin{satz}
\begin{mdframed}
Sei $M\in \LL^d$. Dann existieren eine $F_\sigma$-Menge $F\subset \R^d$ und eine $G_\delta$-Menge\footnote{Sie sind Borel-messbar, ``nah verwandt'' (1.Grades) mit der Topologie.}  $G\subset \R^d$ sodass $F\subset M \subset G$ und $\lambda^d(G\setminus F)=0$.
\end{mdframed}
\end{satz}

Folgerung für das Lebesgue-Maß: für $M\in \LL^d$ gilt
$$
\boxed{\sup\{\lambda^d(K) \mid K \subset M, K \text{ kompakt}\} = \lambda^d(M)=\inf \{\lambda^d(O)\mid O \supset M, O \text{ offen}\}}.
$$
\marginpar{\tiny{21.11.2019}}

\subsubsection{Vervollständigung von Maßen}
Im folgenden sei $\nu$ ein Maß auf einer $\sigma$-Algebra $\AA \subset \PP(X)$.

\begin{definition}
\begin{mdframed}
Eine Menge $N\in \AA$ mit $\nu(N)=0$ heißt \textbf{$\boldsymbol{\nu}$-Nullmenge}.
\end{mdframed}
\end{definition}
Es ist klar, dass abzählbare Vereinigungen von $\nu$-Nullmengen wegen der $\sigma$-(Sub)Additivität von Maßen wieder $\nu$-Nullmengen sind. Zu $\AA$ gehörige Teilmengen von $\nu$-Nullmengen sind $\nu$-Nullmengen.

\begin{definition}[Vollständigkeit von Maßen]
\begin{mdframed}
Das Maß $\nu$ heißt \textbf{vollständig}, wenn alle Teilmengen von $\nu$-Nullmengen zu $\AA$ gehören, also $\nu$-Nullmengen sind.
\end{mdframed}
\end{definition}

\begin{example}
Die Einschränkung $\mu^*|_{\AA^*}$ eines äußeren Maßes auf die $\sigma$-Algebra der $\mu^*$-messbaren Teilmengen ist vollständig.
\end{example}

\begin{proposition}
\begin{mdframed}
Ist $\NN \subset \PP(X)$ die Familie aller Teilmengen von $\nu$-Nullmengen, so ist
\begin{equation*}
	\begin{split}
		\widetilde{\AA} & := \{ A \cup N \mid A \in \AA, N \in \NN\}	\\
								& = \{\widetilde{A} \subset X \mid \exists A \in \AA \text{ mit } \widetilde{A} \triangle A \in \NN \}
	\end{split}
\end{equation*}
eine $\sigma$-Algebra und $\nu$ lässt sich in eindeutiger Weise zu einem Maß $\widetilde{\nu}$ auf $\widetilde{\AA}$ fortsetzen. 
Für $\widetilde{A} \in \widetilde{\AA}$ und $A \in \AA$ mit $\widetilde{A} \triangle A \in \NN$ gilt $\widetilde{\nu}(\widetilde{A})=  \nu (A)$.
\end{mdframed}

\begin{proof}\
\begin{itemize}[itemsep=0pt,topsep=3pt]
\item Ist $A\in \AA, N \in \NN$, so ist $(A \cup N) \triangle A = N\in \NN$, d.h. $\subseteq$.
\item Sei $\widetilde{A}\subset X$ so, dass es ein $A \in \AA$ existiert mit $A \triangle \widetilde{A} \in \NN$. Dann ist $\widetilde{A}= A \cup (\widetilde{A} \triangle A)$, also $\supseteq$.
%Sei $A \in \AA$, $N$ eine $\nu$-Nullmenge, $\widetilde{A}\subset X$ sodass $\widetilde{A} \triangle A \subset N$. Dann gilt $A \setminus N \subset \widetilde{A} \subset A \cup N$. Es folgt $\widetilde{A} \subset (A \setminus N) \cup N$. Wir setzen $A':=A \setminus N$, wegen Definition ist $A' \in \widetilde{\AA}$. Es folgt $\widetilde{A} \subset A' \subset N'$.
\end{itemize}
Also stimmen beide Familien überein. Die erste Beschreibung der Familie zeigt, dass sie $\sigma$-$\cup$-stabil ist. Die zweite Beschreibung zeigt, dass sie $\complement$-stabil ist. (da $\complement \widetilde{A} \triangle \complement A  = \widetilde{A} \triangle A$.) Also ist sie eine $\sigma$-Algebra. Ein $\nu$ auf $\widetilde{A}$ fortsetzendes Maß $\widetilde{\nu}$ erfüllt
\begin{equation}\label{vervollst}
\widetilde{\nu}(\widetilde{A})=\nu(A) \text{ für } \widetilde{A} \in \widetilde{\AA} \text{ mit } \widetilde{A}\triangle A \in \NN \text{ und } A \in \AA.
\end{equation}
Wir definieren nun die Fortsetzung $\widetilde{\nu}$ von $\nu$ durch \eqref{vervollst}.
\begin{itemize}[itemsep=0pt,topsep=3pt]
\item
\emph{Wohldefiniertheit:} Wir wollen zeigen, dass $\widetilde{\nu}(\widetilde{A})$ nicht von der Auswahl von $A\in \AA$ abhängt. Seien hierzu $A,A' \in \AA$ und $N,N'$ zwei $\nu$-Nullmengen sodass $\widetilde{A} \triangle A \subset N$ und $\widetilde{A} \triangle A' \subset N'$. Weil $A \triangle A' \subset (A \triangle \widetilde{A}) \cup (A' \triangle \widetilde{A})$, ist $A \triangle A' \in \AA$ eine $\nu$-Nullmenge. Es folgt $\nu(A)=\nu(A')$. Also gibt \eqref{vervollst} wohldefiniertes $\widetilde{\nu}$.
\item Es bleibt zu überprüfen, dass $\widetilde{\nu}$ tatsächlich ein Maß ist, d.h. \emph{$\sigma$-additiv} ist. Sei $(A_n)_{n \in \N}$ eine Folge in $\AA$ und $(N_n)_{n\in\N}$ eine Folge in $\NN$ sodass die $\underbrace{A_n \cup N_n}_{\in \widetilde{\AA}}$ paarweise disjunkt sind. Es gilt
$$
\sum_{n \in \N} \widetilde{\nu}(A_n \cup N_n) \overset{\text{Def}}= \sum_{n \in \N} \widetilde\nu (A_n) \underset{\nu \text{ Maß}}{\overset{A_n \text{ disj}}=} \nu\Big(\bigcup_{n\in\N} A_n\Big) \overset{\text{Def}}= \widetilde{\nu} \bigg( \underbrace{ \overbrace{\Big( \bigcup_{n\in \N}A_n}^{\in \AA} \Big) \cup \overbrace{\Big(\bigcup_{n\in\N} N_n \Big)}^{\in \NN}}_{\bigcup\limits_{n\in\N}(A_n \cup M_n)} \bigg)
$$
Also ist $\widetilde{\nu}$ auch $\sigma$-additiv.
\end{itemize}
\end{proof}
\end{proposition}

\begin{definition}
\begin{mdframed}
Das Maß $\widetilde{\nu}$ heißt die \textbf{Vervollständigung} des Maßes $\nu$. 
\end{mdframed}
\end{definition}

Wir erinnern uns an \hyperref[uebereinstimmung-alg]{ein früheres Korollar}: Ist $\mu$ ein $\sigma$-endliches Prämaß auf einem Ring $\RR \subset \PP(X)$, so stimmen alle $\mu^*$-messbaren Teilmengen bis auf Nullmengen mit Teilmengen aus $\sigma(\RR)$ überein. Daraus folgt folgendes Korollar:

\begin{korollar}
\begin{mdframed}
Ist $\mu$ ein $\sigma$-endliches Prämaß auf einem Ring $\RR \subset \PP(X)$, so ist $\mu^*|_{\AA^*}$ die Vervollständigung von $\mu^*|_{\sigma(\RR)}$.
\end{mdframed}
\end{korollar}

\begin{hauptbsp}
\begin{mdframed}
$\lambda^d$ ist die Vervollständigung von $\beta^d$.
\end{mdframed}
\end{hauptbsp}

\subsubsection{Bildmaße}
Wir kommen kurz zurück auf die messbare Kategorie.
\begin{definition}
\begin{mdframed}
Ein \textbf{Maßraum} ist ein Tripel $(X,\AA,\mu)$, besteht aus einer Menge $X$, einer $\sigma$-Algebra $\AA\subset \PP(X)$ und einem Maß $\mu$ auf $\AA$.
\end{mdframed}
\end{definition}

Maße können mittel Abbildungen ``nach vorne'' (kovariant im Sinne der Kategorientheorie) transportiert werden:

\begin{definition}
\begin{mdframed}
Sei $f:(X,\AA)\longrightarrow (Y,\BB)$ eine \emph{messbare} Abbildung zwischen Messräumen. Ist $\mu$ ein Maß auf $\AA$, so ist das \textbf{Bildmaß} $f_*\mu$ auf $\BB$ gegeben durch für $B\in \BB$
$$
(f_*\mu)(B):= \mu (\underbrace{f^{-1}(B)}_{\in \AA}) 
$$
\end{mdframed}
\end{definition}
Klar ist, dass $f_*\mu$ wieder $\sigma$-additiv ist, also tatsächlich ein Maß.
Wir betrachten
$$(X,\AA)\xrightarrow[\text{messbar}]{f} (Y,\BB) \xrightarrow[\text{messbar}]{g} (Z,\CC).$$
Es gilt die \emph{Transitivität} $(g\circ f)_* \mu = g_*(f_*\mu)$, denn für $C \in \CC$ gilt:
\begin{equation*}
\begin{split}
	((g\circ f)_* \mu) (C) = \mu ((g\circ f)^{-1}(C))	
								 = \mu(f^{-1}(g^{-1}(C))	
								 = (f_*\mu)(g^{-1}(C))		
								  = (g_*(f_*\mu))(C).
\end{split}
\end{equation*}
Eine messbare Abbildung $f:(X,\AA,\mu)\longrightarrow (Y,\BB,\nu)$ heißt \textbf{maßerhaltend}, falls $\boxed{f_*\mu=\nu}$. Ist $f$ bijektiv, maßerhaltend und $f^*\BB=\AA$, so ist $f$ ein \textbf{Isomorphismus von Maßräumen}. Ist eine Selbstabbildung $f:(X, \AA,\mu) \longrightarrow (X, \AA, \mu)$ maßerhaltend, so sagt man, dass das Maß $\mu$ unter der messbaren Abbildung $f$ \textbf{invariant} ist. Die maßerhaltenden bijektiven Selbstabbildungen eines Maßraums nennt man seine \textbf{Automorphismen}. Sie bilden eine Gruppe. 

\begin{remark}\
\begin{itemize}
\item
Eine bijektive Abbildung von Mengen $f:X \longrightarrow Y$ induziert eine ebenfalls bijektive Abbildung $f_*:\PP(Y) \longrightarrow \PP(X), M \longmapsto f^{-1}(M)$ mit der Umkehrabbildung $M\mapsto f(M)$. 
\item Ist  $f:(X,\TT_X) \longrightarrow (Y,\TT_Y)$ eine Bijektion topologischer Räume, so ist $f$ genau dann Homöomorphismus, wenn $f^*\TT_Y=\TT_X$.  In diesem Fall gilt auch $f^*(\underbrace{\BB(Y)}_{\sigma(\TT_Y)}) = \sigma (\underbrace{f^*\TT_Y}_{\TT_X}) = \BB(X)$. D.h. $f$ und $f^{-1}$ messbar, also ist $f:(X,\BB(X))\longrightarrow (Y,\BB(Y))$ ein Isomorphismus von Messräumen.
\end{itemize}
\end{remark}


\subsubsection{Produktmaße}
Wir konstruieren nun endliche Produkte von Maßen bzw. Maßräumen.
\vspace{0.5pc}
\\Seien $(X_i,\AA_i,\mu_i)$ Maßräume. Dann wird die Produkt-$\sigma$-Algebra $\AA_1 \otimes ... \otimes \AA_n \subset \PP(X_1\times ... \times X_n)$ erzeugt von dem Halbring $\QQ =\AA_1 \ast ... \ast \AA_n$ der Quander sowie dem Ring $\FF=\AA_1 \boxtimes ... \boxtimes \AA_n$ der Figuren. Der Produktinhalt $\mu_1 \times ... \times \mu_n$ auf $\QQ$ wird eindeutig fortgesetzt zu dem Produktinhalt $\mu_1 \boxtimes ... \boxtimes \mu_n$ auf $\FF$.
\vspace{0.5pc}
\\ Weil $\mu_i$ Maße (insb. Prämaße) sind, ist $\mu_1 \boxtimes ... \boxtimes \mu_n$ auch ein Prämaß und wird fortgesetzt zu einem Maß auf der Produkt-$\sigma$-Algebra
$$
\mu_1 \otimes ... \otimes \mu_n =  (\mu_1 \boxtimes ... \boxtimes \mu_n)^*|_{\AA_1 \otimes ... \otimes \AA_n}
$$
Sind alle $\mu_i$ $\sigma$-endlich, so ist $\mu_1 \boxtimes ... \boxtimes \mu_n$ auch $\sigma$-endlich. Somit ist $\mu_1 \otimes ... \otimes \mu_n $ die \emph{eindeutige} Fortsetzung des Prämaßes $\mu_1 \boxtimes ... \boxtimes \mu_n$ zu einem Maß auf $\AA_1 \otimes ... \otimes \AA_n$. In diesem Fall heißt $\mu_1 \otimes ... \otimes \mu_n$ das \textbf{Produktmaß} (der $\mu_i$) und 
$$
(X_1 \times ... \times X_n, \AA_1 \otimes ... \otimes \AA_n, \mu_1 \otimes ... \otimes \mu_n)
$$
das Produkt der Maßräume $(X_i,\AA_i,\mu_i)$.


\subsubsection{Charakterisierung des Lebesgue-Maßes}
Wir wollen zeigen, dass das Lebesgue-Maß \emph{eine natürliche Lösung} des eingangs diskutierten ``(post-paradoxalen) Maßproblems\footnote{Man definiere eine Volumenfunktion $\FF \longrightarrow [0,\infty]$ mit geforderten Eigenschaften ($\sigma$-additiv, Bewegungsinvarianz und Normierung) auf einer möglich flexiblen und reichhaltigen Familie (jetzt $\sigma$-Algebra) $\FF\subset \PP(\R^d)$.}"" ist und dadurch \emph{charakterisiert} wird.

\paragraph{Geometrische Transformationen des $\R^d$}
\begin{itemize}
\item \textbf{Affine Transformationen}  des $\R^d$ sind Abbildung $\Phi_{A,v}$ der Form
$$
\Phi_{A,v}(x)=Ax + v	\quad \text{ mit }	\underset{\text{linearer Anteil}}{A\in \GL (d,\R)}, v\in\R^d
$$
Sie sind geometrisch charakterisierbar als injektive Selbstabbildungen, welche 
	\begin{itemize}
		\item Geraden erhalten\footnote{D.h. Geraden werden auf Geraden abgebildet.},
		\item Teilverhältnisse kollinearer\footnote{Punkte, die auf einer Geraden liegen, sind kollinear.} Triple werden erhalten.
	\end{itemize}
	Affine Transformationen sind \emph{Homöomorphismen}, also auch Automorphismen\footnote{Ein Automorphismus ist ein \emph{Isomorphismus} auf sich selbst.} von $(\R^d,\BB^d,\beta^d)$. Die affinen Transformationen des $\R^d$ bilden die \emph{affine Gruppe} (Lie-Gruppe) $\operatorname{Aff}(\R^d) \subset \operatorname{Homöos} (\R^d) \subset \operatorname{messb. Autom.}$
	\item \textbf{Bewegungen} bzw. \textbf{Isometrien} des $\R^d$ sind die Selbstabbildungen $\R^d \longrightarrow \R^d$, die die metrische Struktur, d.h. den \emph{euklidischen Abstand} erhalten. Sie sind genau die affinen Transformationen $\Phi_{A,v}$, für die gilt $A \in \O(n)$\footnote{$\mathrm{O}(d) := \{ A \in \operatorname{GL}(d,\R) \mid AA^t = \operatorname{id}_{\R^d}\} \subset \End(\R^d)$, ist (überall) eine $\mathcal{C}^\infty$-Untermannigfaltigkeit von $\End(\R^d)\cong \R^{d^2}$.}. Sie bilden ebenfalls eine Bewegungsgruppe und es gilt $\operatorname{Trans}(\R^d) \subset \operatorname{Isom(\R^d)}  \subset \operatorname{Aff}(\R^d)$.
\end{itemize}
\marginpar{\tiny{25.11.2019}}
Zusammengefasst:
\begin{equation*}
	\begin{array}{ccccc}
			\Aff(\R^d) & \supset &  \Isom(\R^d)  & \supset  & \Trans (\R^d)	\vspace{0.5pc}\\
			\te{affine Transformationen} && \te{Bewegungen} && \te{Translationen}\\
			&& \te{(Isometrien)} \vspace{0.5pc}\\
			\Phi_{A,v} = Ax+v && A \te{ orthogonal} && \te{trivialer lin. Anteil} \\
			A \in \GL(d,\R) && \te{(Rotationsanteil)} && A= \id_{\R^d}
	\end{array}
\end{equation*}

\begin{remark}
Die lineare Abbildung $\operatorname{lin}:\Aff(\R^d) \overset{\operatorname{}}\longrightarrow \GL(d,\R), \Phi_{A,v} \longmapsto A$ ist ein Epimorphismus. Ihr Kern sind die Translationen, also ein  Normalteiler\footnote{Eine Untergruppe $H$ einer Gruppe $G$ heißt Normalteil, falls für jedes $g \in G$ gilt $gH=Hg$. D.h. für jedes $g\in G$ stimmt die Linksnebenklasse mit der Rechtsnebenklasse überein. Man kann zeigen, dass die folgenden Aussagen äquivalent sind:
\begin{enumerate}[topsep=0pt,itemsep=0pt]
\item $H$ ist ein Normalteil von $G$.
\item Es existiert ein Gruppenhomomorphismus aus $G$, dessen Kern $H$ ist.
\item Für jedes $h\in H$ und $g \in G$ gilt $g\circ h \circ g^{-1} \in H$.
\end{enumerate}\vspace{-0.7pc}} in $\Aff(\R^d)$, also $\Trans(\R^d) \triangleleft \Aff(\R^d)$. Im Sinne der Algebra ist 
$$
\Trans (\R^d) \xrightarrow[\te{inj.}]{\te{Inklusion}} \Aff(\R^d) \xrightarrow[\te{surj.}]{\te{linear}} \GL(d,\R)
$$
 eine exakte Sequenz.\footnote{Eine Sequenz $A' \to A \to A''$ von Objekten und Morphismen in einer geeigneten Kategorie heißt exakt an der Stelle $A$, wenn $\Im(A'\to A) = \ker(A \to A'')$.} Bezüglich der Einschränkung auf den Isometrien $$\Isom(\R^d)\xrightarrow[\te{Epimorphismus}]{\operatorname{lin}|_{\Isom(\R^d)}=:\operatorname{rot}} \O(d)$$ ist
$$
\Trans(\R^d) \longrightarrow \Isom(\R^d) \longrightarrow \O(d)
$$
eine exakte Sequenz.
\end{remark}

Wir untersuchen nun den Effekt affiner Transformationen $\Phi:\R^d\longrightarrow \R^d$ auf Lebesgue-Maß $\lambda^d$.\vspace{0.5pc}

Weil $\Phi$ ein Homöomorphismus ist, erhält $\Phi$ die Borelsche-$\sigma$-Algebra $\BB^d$. Es gilt nämlich $\Phi^*\BB^d= \Phi^*(\sigma(\TT_{\R^d})) = \sigma (\Phi^*(\TT_{\R^d})) = \sigma (\TT_{\R^d})=\BB^d$, d.h. $\Phi$ ist Borel-messbar.\footnote{Vorsicht: Nicht alle Homöomorphismen erhalten $\LL^d$! Dies ist äquivalent dazu, dass Nullmengen erhalten werden (von $\Phi$ und $\Phi^{-1}$).} Affine Transformationen sind auch Lebesgue-messbar, d.h. $\Phi^* \LL^d = \LL^d$. Dies folgt sowohl aus der Lipschitz-Stetigkeit (und damit werden Nullmengen erhalten\Ueb), als auch aus dem Skalierungsverhalten von $\lambda^d$ unter $\Aff(\R^d)$ (siehe spätere Diskussion).\vspace{0.5pc}

Wir untersuchen zunächst den Effekt von Translationen auf $\lambda^d$. Weil die Translationen Volumina von Quadern erhalten, also $\lambda^d_{\QQ^d}$ (der Elementarinhalt) invariant unter Translationen, folgt, dass $\beta^d$ translationsinvariant ist. Außerdem ist $\BB^d = \sigma (\QQ^d)$ translationsinvariant, weil $\QQ^d$ translationsinvariant ist.

\begin{lemma}
\begin{mdframed}
Die $\sigma$-Algebren $\BB^d$ und $\LL^d$ sowie die Maße $\beta^d$ und $\lambda^d$ sind invariant unter Translationen, ebenso das äußere Maß $(\beta^d)^* = (\lambda^d)^*$.
\end{mdframed}
\begin{proof}
Seien $Q = [a,b) \in \QQ^d$, $a<b$ (d.h. $a_i<b_i \ \forall i$) und $v \in \R^d$. Es ist
$$
\underbrace{T^{-1}_v}_{T_{-v}}(Q) = [a-v, b-v) \subset \QQ^d
$$
und
$$
((T_v)_* \lambda^d_{\QQ^d})(Q) = \lambda^d_{\QQ^d} (T^{-1}_v(Q)) = \prod_{i=1}^d ((b_i-v_i)-(a_i-v_i))=\prod_{i=1}^d (b_i-a_i)=\lambda^d_{\QQ^d}(Q).
$$
Also ist $T^*_{v}\QQ^d=\QQ^d$ und $(T_v)_* \lambda^d_{\QQ^d} = \lambda^d_{\QQ^d}$.  Da Figuren endliche Vereinigungen von Quadern sind, folgt die Translationsinvarianz des von $\QQ^d$ erzeugten Rings $\FF^d$ sowie die Translationsinvarianz des den Inhalt $\lambda^d_{\QQ^d}$ eindeutig fortsetzenden Inhalts $\lambda^d_{\FF^d}$ (Lebesgue-Prämaß) $T^*_{v} \FF^d = \FF^d$ und $(T_v)_*\lambda^d_{\FF^d}\footnote{Inhalt, der $(T_v)_* \lambda^d_{\QQ^d}$ auf $\QQ^d$ eindeutig fortsetzt.} = \lambda^d_{\FF^d}$.\\\\
Weiter folgt: die von $\QQ^d$ erzeugte $\sigma$-Algebra $\BB^d$ ist invariant unter Translationen, da $T^*_v \BB^d = T^*_v (\sigma(\FF^d)) = \sigma(T^*_v(\FF^d)) = \sigma(\FF^d)=\BB^d$. Ebenfalls ist das äußere Maß $(\lambda^d)^*=(\beta^d)^*=(\lambda^d_{\FF^d})^*=(\lambda^d_{\QQ^d})^*$ translationsinvariant:
$$
(T_v)_*(\lambda^d_{\FF^d})^*=
((T_v)_* \lambda^d_{\FF^d})^* = (\lambda^d_{\FF^d})^*%\ \ \footnote{Äußeres Maß und Bildmaß vertauschbar}
$$
Es folgt, dass die $\sigma$-Algebra der $(\lambda^d_{\FF^d})^*$-messbaren Teilmengen invariant (unter Translationen) ist, damit auch $\lambda^d=(\lambda^d_{\FF^d})^*|_{\LL^d}$. Also $T^*_v \LL^d = \LL^d$ und $(T_v)_* \lambda^d = \lambda^d$. Es folgt, dass $\beta^d=\lambda^d|_{\BB^d}$ ebenfalls translationsinvariant ist, denn 
$$
(T_v)_*\beta^d=(T_v)_*(\lambda^d|_{\BB^d}) = (T_v)_*\lambda^d|_{T^*_{-v}\BB^d}= \lambda^d|_{\BB^d}=\beta^d
$$
\end{proof}
\end{lemma}
Es war ein großer Schritt zur Lösung des Maßproblems, denn $\beta^d$ ist translationsinvariant und normiert durch $\beta^d([0,1)^d)=\lambda^d_{\FF^d}([0,1)^d)=1$. Diese Eigenschaften charakterisieren das Lebesgue-Borel-Maß bereits:

\begin{satz}
\begin{mdframed}
Ist $\mu$ ein translationsinvariant Maß auf $\BB^d$ mit $\mu([0,1)^d)=c<\infty$, so ist $\mu$ ein Vielfaches des Lebesgue-Borel-Maßes, d.h. $\mu=c\cdot \beta^d$.
\end{mdframed}
\begin{proof}
Weil $\mu$ translationsinvariant ist, hängt das $\mu$-Volumen von Quadern in $\QQ^d$ nur von deren euklidischen Kantenlängen ab. Wir betrachten einen kleineren Erzeuger von $\BB^d$, nämlich 
$$
\QQ^d_{\Q} := \{ \te{Quader aus }\QQ^d \te{ mit Ecken in } \Q^d\} \subset \QQ^d.
$$
Es folgt unmittelbar, dass die Kantenlängen der Quader aus $\QQ^d_\Q$ rational sind.\\
Zu $Q,Q' \in \QQ^d_\Q$ existieren $n,n'\in\N$ und ein weiterer größerer Quader $\in \QQ^d_\Q$, der disjunkt sowohl in $n$ Translate von $Q$ zerlegt werden kann als auch disjunkt in $n'$ Translate von $Q'$. Es folgt
\begin{equation*}
\begin{cases}
	n \cdot \mu(Q) = n' \cdot  \mu(Q') \\
	n \cdot \underbrace{\beta^d(Q)}_{>0} = n' \cdot \underbrace{\beta^d(Q')}_{>0}
\end{cases}
\end{equation*}
Also ist $\dfrac{\mu(Q)}{\beta^d(Q)}$ unabhängig von Auswahl $Q\in \QQ^d_\Q$. Es ist $\dfrac{\mu(Q)}{\beta^d(Q)} = \dfrac{\mu([0,1)^d)}{\beta^d([0,1)^d} =c \in [0,\infty)$. Es folgt $\mu|_{\QQ^d_\Q}= c \cdot \beta^d|_{\QQ^d_\Q}$. Weil die Fortsetzung von Inhalten von Halbringen auf Ringen eindeutig ist, folgt $\mu|_{\FF^d_\Q} = c \cdot \beta^d|_{\FF^d_\Q}$. $\mu|_{\FF^d_\Q}$ ist ein Prämaß (da Einschränkung von Maß $\beta^d$) und $\sigma$-endlich. Wegen der Eindeutigkeit der Fortsetzung von $\sigma$-endlichen Prämaßen auf Ringen $\RR$ zu Maßen auf $\sigma(\RR)$ folgt $\mu = c \cdot \beta^d$, denn $\BB^d=\sigma(\FF^d_\Q)$.
\end{proof}
\end{satz}

\begin{korollar}[Charakterisierung des Lebesgue-Borel-Maßes]
\begin{mdframed}
$\beta^d$ ist das eindeutig bestimmte translationsinvariante Maß auf $\BB^d$ mit $\beta([0,1)^d)=1$.
\end{mdframed}
\end{korollar}

Wir nutzen diese Charakterisierung jetzt, um die \emph{Bewegungsinvarianz} von $\beta^d$ einzusehen. Dies basiert auf der Struktur der affinen Gruppen, nämlich dass $\Trans(\R^d) \triangleleft \Aff(\R^d)$. Die natürliche Operation $\Aff(\R^d) \curvearrowright \{\te{Maße auf }\BB^d\}$ erhält die Fixpunktemenge des Normalteilers $\Trans(\R^d)$, d.h. die translationsinvarianten Maße $\{c\cdot\beta^d \mid c\geq 0\}$

\begin{lemma}
\begin{mdframed}
Ist $\mu$ ein translationsinvariantes Maß auf $\BB^d$ und $\Phi \in \Aff(\R^d)$, so ist das Bildmaß $\Phi_*\mu$ auch translationsinvariant.
\end{mdframed}
\begin{proof}
Weil $\Trans(\R^d) \triangleleft \Aff(\R^d)$, gilt
$$
\Phi^{-1} T_v \Phi \in \Trans(\R^d)
$$
Die Invarianz unter Translationen von $\mu$ liefert
$$
(T_v)_*\Phi_*\mu=  \Phi_* \underbrace{(\Phi^{-1})_*(T_v)_*\Phi_*}_{\substack{(\Phi^{-1}T_v\Phi)_*\\ \te{Translation}}}\mu = \Phi_* \mu
$$
Also ist $\Phi_*\mu$ translationsinvariant.
\end{proof}
\end{lemma}

Weil laut Charakterisierung alle translationsinvarianten Maße zueinander proportional sind, folgt
\begin{korollar}
\begin{mdframed}
Es existiert einen Gruppenhomomorphismus $c:\Aff(\R^d)\longrightarrow \R^+$ sodass 
$\Phi_*\beta^d = c(\Phi)\beta^d$ für $\Phi \in \Aff(\R^d)$.
\end{mdframed}
\begin{proof}
Das letzte Lemma zusammen mit dem Charakterisierungssatz liefert eine Funktion $c:\Aff(\R^d)\longrightarrow \R^+$. Seien $\Phi,\Psi \in \Aff(\R^d)$. Ihre Multiplikativität folgt aus
\begin{equation*}
\begin{split}
c(\Psi \circ \Phi) \cdot \beta^d = \underbrace{(\Psi \circ \Phi)_*}_{\Psi_*\circ\Phi_*} \beta^d= \Psi_*(\underbrace{\Phi_*\beta^d}_{c(\Phi)\beta^d}) = c(\Psi) \cdot c(\Phi) \cdot \beta^d
\end{split}
\end{equation*}
Weil $\beta^d \neq 0$ folgt $c(\Psi \circ \Phi)=c(\Psi)\cdot c(\Phi)$.
\end{proof}
\end{korollar}


\marginpar{\tiny{28.11.2019}}

%\appendix
\newpage
\subsubsection*{Wiederholung: Integrale stetiger Funktionen (einer reellen Variable)} %\small (einer reellen Variable)}
\marginpar{\tiny{23.10.2019}}
Wir unterscheiden zwischen
\begin{enumerate}[- ,topsep =-3pt]
	\item dem \emph{bestimmten Integral}
				$$ \int\limits_a^b f(x)\D x$$
	\item und dem \emph{unbestimmten Integral}, d.h. die Menge der Funktionen dieser Art
	$$x \mapsto \int\limits_a^x f(\xi) \D \xi + \underset{\text{\tiny const}}C$$
	Notation: $\int f \D x$.
\end{enumerate}

\begin{satz}[\textbf{Hauptsatz der Differential- und Integralrechnung}]  \begin{mdframed} \
\begin{enumerate}[(\roman*), topsep = -1 pt]
	\item Ist $f$ von der Klasse $\mathcal{C}^0$ (d.h. stetig), so ist 
		$$\left(\int f \D x\right)' = f$$
		d.h. die Repräsentanten des unbestimmten Integrals sind Stammfunktionen.
	\item Ist $f$ von der Klasse $\mathcal{C}^1$ (d.h. stetig differenzierbar), so ist
	$$ \int F' \D x =F$$
	(zu lesen: $F$ repräsentiert $\int F' \D x)$ bzw.
	$$ \int\limits_a^x F'(\xi) \D x = F(x) -F(a)$$
\end{enumerate}
\end{mdframed}
\end{satz}

\paragraph{Rechenregeln für Differentialrechnung $\leadsto$ Rechenregeln für Integralrechnung}z.B.
\begin{equation*}
	\begin{split}
		(\ln |x|)' = \frac{1}{x} & \implies \int \frac{\D x}{x} = \ln |x| \text{ auf } \R\setminus\{0\}\\
		\arcsin' x {=\frac{1}{\sqrt{1-x^2}}} & \implies \int \frac{\D x}{\sqrt{1-x^2}} = \arcsin x \text{ auf } (-1,1) \\ 		
		\arctan' x = \frac{1}{1+x^2} & \implies \int \frac{\D x}{1+x^2} = \arctan x \text{ auf }\R
	\end{split}
\end{equation*}
\paragraph{Kettenregel $\leadsto$ Substitutionsregel} Aus der Kettenregel
$$(F \circ \varphi)' (u) = F'(\varphi (u)) \cdot \varphi'(u)$$
folgt mit dem Hauptsatz der Differential- und Integralrechnung: Sei $f := F'$ ($\mathcal{C}^0$),
$$ \int\limits_a^b f(\varphi(u)) \cdot \varphi'(u) \D u = \int\limits_a^b (F \circ \varphi) (u) \D u= F \circ \varphi \big\vert_a^b = F\big\vert^{\varphi(b)}_{\varphi(a)} = \int\limits_{\varphi(a)}^{\varphi(b)} f \D x
$$
Also die \textbf{Substitutionsregel} (Bezeichne $I:= (a,b), J = (\phi(a),\phi(b))$)
$$
\boxed{
\int\limits_a^b f(\varphi(u)) \varphi'(u) \D u = \int_{\phi(a)}^{\phi(b)} f(x) \D x 
}
$$
und die \textbf{Version für unbestimmtes Integral}
$$ \int f(\phi(u)) \phi'(u) \D u = \underbrace{\int f(x) \D x \Big \vert_{x =\phi(u)}}_{\substack{\text{die Komposition }\phi \\ \text{ mit } \int f(x)\D x}} $$
\begin{example} \
\begin{enumerate}
 \item \textbf{Lineare Substitution} mit $x = u+\alpha, \alpha \in \R$ 
 $$
	\int_a^b f(u+\alpha) \D x = \int_{a + \alpha}^{b+ \alpha} f(x) \D x
 $$
 bzw. 
 $$
 	\int f( u +\alpha) \D u = \int f(x) \D x \Big\vert_{x = u + \alpha}
 $$
 z.B. $f(x) = \frac{1}{x}$ auf $\R \setminus \{0\}$
 $$ \int_a^b \frac{\D u}{u+\alpha} = \int_{a+\alpha}^{b+\alpha} \frac{\D x}{x} = \ln |x| \big\vert^{b+\alpha}_{a+\alpha} = \ln \left| \frac{b+\alpha}{a+\alpha} \right|$$
 bzw.
 $$ \int \frac{\D u}{u+\alpha} = \int \frac{\D x}{x} \Big\vert_{x= u+\alpha}= \ln |x| \big\vert_{x = u + \alpha} = \ln |u + \alpha|$$
 \item[(i')] \textbf{(Multiplikative) lineare Substitution} mit $x= \lambda u (\lambda \in \R \setminus \{0\})$
 	$$ \int_a^b f(\lambda u) \D u = \frac{1}{\lambda} \int_{\lambda a}^{\lambda b} f(x) \D x$$ bzw. 
 	$$ \int f(\lambda u) \D u = \frac{1}{\lambda} \int f(x) \D x \Big\vert_{x = \lambda u}$$
 	z.B.
 $$\int \cos \lambda u \D u = \frac{1}{\lambda} \int \underbrace{\cos x}_{\sin' x} \D x \Big\vert_{x= \lambda u} = \frac{1}{\lambda} \sin \lambda u$$
 \item \textbf{Quadratische Substitution} mit $x = u^2$
 $$ \int_a^b f(u^2) u \D u =\frac{1}{2} \int_{a^2}^{b^2} f(x) \D x$$
 bzw.
 $$ f(u^2) u \D u = \frac{1}{2} f(x) \D x \Big\vert_{x=u^2}$$
 z.B. $f(x) = e^x$:
 $$ \int u  e^{u^2} \D u = \frac{1}{2} \int e^x \D x \Big\vert_{x=u^2} = \frac{1}{2} e^{u^2}$$
 \item Mit $f(x) = \frac{1}{x}$, (falls $\phi|_J$ keine Nullstelle hat)
 $$ \int_a^b \frac{\phi'(u)}{\phi(u)} \D u = \int_{\phi(a)}^{\phi(b)}\frac{\D x}{x} = \ln |x| \big\vert^{\phi(b)}_{\phi(a)} = \ln |\phi(u)| \big\vert_{a}^b$$
 bzw.
 $$ \int \frac{\phi'(u)}{ \phi(u)} \D u = \int \frac{\D x}{x} \Big\vert_{x=\phi(u)} = \ln | \phi(u) |$$
 z.B. $\phi(u)= \cos u$ auf $(-\frac{\pi}{2},\frac{\pi}{2})$
 $$\int \tan u \D u  = \int -\frac{\cos' u}{\cos u} \D u= -\ln |\cos u|$$
 Berechne $\int \frac{\D x}{\sqrt{1+x^2}}$ auf $\R (=I =J)$. Substituiere $x=\sinh u$ mit der Umkehrfunktion $u = \operatorname{arsinh} x$.
 $$\int \frac{\D x}{\sqrt{1+x^2}} = \int \frac{\sinh' u}{\sqrt{1+\sinh^2 u}} \D u \Bigg\vert_{u = \operatorname{arsinh}x}= \int \frac{\cosh' u}{\cosh' u} \D u\Big\vert_{u = \operatorname{arsinh}x} = \operatorname{arsinh}x$$
\end{enumerate}
\rule{\textwidth}{0.4pt}
\marginpar{\tiny{30.10.2019}}
Die \emph{Produktregel} für die Ableitung führt zur Methode der \emph{partiellen} Integration.
\paragraph{Partielle Integration:} Für $\mathcal{C}^1$ Funktionen $f,g:I\to \C$ auf einem offenen Intervall $I \subset \R$ gilt:
\begin{equation*}
	\boxed{
	\int_a^b f' \cdot g \D x = f\cdot g \big\vert^b_a - \int^b_a f \cdot g' \D x	
	}
\end{equation*}
für $a,b \in I$ Mann nennt $f\cdot g \vert^b_a$ \emph{Randterm}. \\\\
Für unbestimmte Integrale schreibt man
\begin{equation*}
	\boxed{
	\int f' g\D x = f \cdot g \big\vert -  \int f \cdot g' \D x
	}
\end{equation*}
Man kann diese Gleichung lesen als eine Gleichheit von Funktionenmengen oder so, dass jeder Repräsentant der rechten Seite $f \cdot g \big\vert -  \int f \cdot g'$ ein Repräsentant der linken Seite $\int f' g$ ist.
\begin{proof}
Nach der Produktregel ist $f \cdot g$ Stammfunktion von $f'g+fg'$. Der Hauptsatz der Differential- und Integralrechnung liefert dann 
$$ \int (f' g + fg') = fg \big\vert$$
\end{proof}

\begin{example}
\begin{enumerate}
	\item Berechnung von $\int \ln x \D x$ auf $(0,\infty)$. Dort gilt wegen $\ln' x = \frac{1}{x}$
	\begin{equation*}
		\int \ln x \D x = \int (x)'\ln x \D x = x\ln x \big\vert - \int x \cdot \frac{1}{x} \D x = x (\ln x -1)\big\vert
	\end{equation*}
	\item Berechnung von $\int x e^x \D x$ auf $\R$.
	\begin{equation*}
		\int x e^x \D x = \int x (e^x)' \D x = x e^x\big\vert - \int \underbrace{(x)'}_{=1} e^x \D x =e^x (x-1)
	\end{equation*}
	\item[(ii)'] Berechnung von $\int x^n e^x \D x$ auf $\R$.
	\begin{equation*}
		I_n(x) := \int x^n e^x \D x  = \int x^n (e^x)' \D x = x^ne^x - n \underbrace{\int x^{n-1} e^x \D x}_{I_{n-1}(x)}
	\end{equation*}
	Wir erhalten die Rekursionsformel 
	$$I_n (x) = x^n e^x \big\vert - n I_{n-1}(x)$$
	\item Berechnung von $\int\sqrt{1-x^2}\D x $ auf $(-1,1)$.
	\begin{equation*}
	\begin{split}
		\int \sqrt{1-x^2} \D x= \int (x') \sqrt{1-x^2} \D x & = x \sqrt{1-x^2} \big\vert + \int x  \frac{-x}{\sqrt{1-x^2}} \D x \\
		& = x \sqrt{1-x^2}\big\vert + \int \underbrace{\frac{1-x^2}{\sqrt{1-x^2}}}_{\sqrt{1-x^2}}\D x + \underbrace{\int \frac{1}{\sqrt{1-x^2}} \D x}_{\arcsin x} \\
		& = \frac{1}{2}\cdot \Big( x \sqrt{1-x^2} + \arcsin (x) \Big) \Big\vert
	\end{split}
	\end{equation*}
	\begin{remark}
	Die Regel für die Berechnung der Ableitung von Umkehrfunktion ist
	$$ (f^{-1})' = \frac{1}{f'(f^{-1})} $$
	\end{remark}
	und somit haben wir die Ableitung von $\arcsin$:
	$$\arcsin' (x) = \frac{1}{\cos (\arcsin (x))} = \frac{1}{\sqrt{1-\sin^2\big(\arcsin(x)\big)}} = \frac{1}{1-x^2}$$
	Insbesondere erhalten wir durch Grenzübergang \begin{small} (Hier ist der Grenzübergang nötig, da $\sqrt{1-x^2}$ nicht stetig differenzierbar in Punkten $-1$ und $1$ sind) \end{small} für das abgeschlossene Intervall $[-1,1]$:
	\begin{equation*}
		\begin{split}
		\int_{-1}^1 \sqrt{1-x^2} \D x = \lim\limits_{\varepsilon \to 0} \int_{-1+\varepsilon}^{1-\varepsilon} \sqrt{1-x^2} \D x & = \lim\limits_{\varepsilon \to 0} \frac{1}{2} \big(x \sqrt{1-x^2} + \arcsin x\big) \Big\vert_{-1+\varepsilon}^{1-\varepsilon} \\
		& = \frac{1}{2} \left( \frac{\pi}{2} - \left( - \frac{\pi}{2} \right) \right) \\ 
		& = \frac{\pi }{2}
		\end{split}
	\end{equation*}
	Dies zeigt insbesondere, dass die Fläche der Einheitsscheibe $\pi$ ist.  \\ Außerdem können wir das Integral auch mit Substitution berechnen: 
	\begin{equation*}
	\begin{split}
	 \int_{-1}^1 \sqrt{1-x^2} \D x & \overset{x= \sin u}= \int^{\pi/2}_{-\pi/2} \sqrt{1-\sin^2(u)} \cos u \D u  = \int_{-\pi/2}^{\pi/2} \underbrace{\cos^2 u}_{ \frac{1+\cos 2u}{2}} \D u \\
		& = 	\Big(\frac{u}{2}+\frac{1}{4}\sin 2u\Big) \Big\vert_{-\pi/2}^{\pi/2} = \frac{\pi}{2}
	 \end{split}
	\end{equation*}
	\item Berechnung von $\int \arctan$: Wir bemerken, dass $\arctan ' (x)=\frac{1}{1+x^2}$, denn $\tan'(x)=1+\tan^2x $.
	\begin{equation*}
	\begin{split}
	  \int \arctan(x) \D x &= \int (x')\arctan(x) \D x = x \cdot \arctan x \big\vert - \int \frac{1}{2} \frac{2x}{1+x^2} \D x \\
	  &  \overset{t = x^2}= x \cdot \arctan x - \int \frac{1}{2} \cdot \frac{1}{1+t} \D t \Big\vert_{t=x^2} \\
	  & = x \cdot \arctan x - \frac{1}{2} \ln (1+x^2)
	  	\end{split}
	\end{equation*}
	\item Berechnung von $\int \arcsin(x)$ auf $(-1,1)$
	\begin{equation*}
	\begin{split}
		\int \arcsin(x) \D x = \int (x)' \arcsin(x) \D x & = x\arcsin(x) \big\vert - \int \frac{1}{2} \frac{2 x \D x}{\sqrt{1-x^2}} \\
		& \overset{t:= x^2}= x\arcsin (x) \big\vert - \int \frac{1}{2} \frac{\D t}{\sqrt{1-t}}  \Big\vert_{t=x^2}	 \\
		& = x\arcsin x + \sqrt{1-t}\big\vert_{t=x^2} \\
		& = x\arcsin x + \sqrt{1-x^2} \big\vert
	\end{split}
	\end{equation*}
	\item Berechnung von $\int \sin^2x \D x$. Da
	\begin{equation*}
		\int \sin^2 x \D x = \int (-\cos(x))' \sin x \D x = -\cos x \sin x + \int \underbrace{\cos^2 x}_{1-\sin^2 x} \D x
	\end{equation*}
	erhalten wir
	$$
	\int \sin^2 x \D x = \frac{1}{2} (-\cos x \sin x + x )	
	$$
	\item[(vi)'] Berechnung von $\int \sin^n(x) \D x$ für $n \in \N$.
	\begin{equation*}
		\begin{split}
		I_n (x) := \int \sin^n (x) \D x & = \int (-\cos(x))' \sin^{n-1}(x)\D x \\
				& = -\cos x \sin^{n-1} x + \int 	\underbrace{\cos^2 (x)}_{1-\sin^2(x)} (n-1) \sin^{n-2} (x) \D x \\
				&  = -\cos x \sin^{n-1} x +(n-1) \left( \underbrace{\int \sin^{n-2} (x) \D x}_{I_{n-2}(x)} - \underbrace{\int \sin^n(x)\D x}_{I_n(x)} \right)
		\end{split}
	\end{equation*}
	Wir erhalten:
	$$n \cdot I_n(x) = -\cos (x)\sin^{(n-1)}(x) + (n-1) I_{n-2}(x)$$
	Zum Beispiel gilt $I_0(x)=x$, $I_1(x)=-\cos x $, $I_2(x)=\frac{1}{2}(x-\sin x\cos x)$, $I_3(x)=\frac{1}{3} \cos^3 x - \cos x$.
\end{enumerate}
\end{example}
\newline
\rule{\textwidth}{0.4pt}

\paragraph{Rationale Funktionen:} $\int \frac{p(x)}{q(x)} \D x$ mit $p,q$ Polynome, $q \neq 0$.Zunächst $p,q \in \C[x]$:
\begin{satz}[\textbf{Fundamentalsatz der Algebra}] \begin{mdframed} Jedes Polynom $p \in \C[x]$ vom Grad $n= \deg (p)$ besitzt eine Zerlegung in Linearfaktoren:
$$ p(x) = c \cdot \prod_{n=1}^m (x-\alpha_k)^{n_k}$$
mit $m \leq n, c \in \C \setminus \{0\}, \alpha_1,...,\alpha_m \in \C, \sum_{k=1}^m n_k = n$. Hierbei sind $\alpha_1,...,\alpha_m$ die verschiedenen Nullstellen von $p$ und $n_k$ die Vielfachheit von $\alpha_k$. Die Linearfaktorzerlegung ist bis auf Vertauschung von Faktoren eindeutig.
\end{mdframed}
Polynomdivision führt zu:
$$\frac{p(x)}{q(x)} = s(x) + \frac{r(x)}{q(x)} \quad \text{mit } s\in \C [x] \text{ und } \deg r < \deg q$$
\end{satz}

\paragraph{Partialbruchzerlegung:} Sei $q(x) \in \C [x]$ mit $q(x) = c \cdot \prod_{n=1}^m (x-\alpha_k)^{n_k} $. \linebreak Für jedes komplexe Polynom $r(x) \in \C[x]$ mit $\deg r < \deg q =n $ existiert eine eindeutige Zerlegung:
$$
\frac{r(x)}{q(x)} = \sum_{k=1}^m \sum_{j=1}^{n_k} \frac{c_{kj}}{(x-x_k)^j}, \quad c_{jk} \in \C
$$
d.h.
$$r(x) = \sum_{k=1}^m \sum_{j=1}^{n_k} b_{kj}(x) c_{kj}$$
mit 
$$b_{kj} = (x-\alpha_k)^{n_k-j} \cdot \prod_{\substack{l=1 \\ l \neq k}}^m (x-\alpha_l)^{n_l}$$
Anderes gesagt, $\{b_{kj}\}$ für $k=1,...,m$ und $j=1,...,n_k$ bilden Basis von $\C_{\deg<n}[x]$. Insbesondere $\dim( \C_{\deg <n}[x])=n$.

\begin{example}
$$\int \frac{1}{1-x^2} \D x = \frac{1}{2} \int \frac{1}{1-x} \D x + \frac{1}{2} \int \frac{1}{1+x} \D x = \frac{1}{2} \ln \left| \frac{1+x}{1-x} \right|$$
\end{example}
\end{example}
\end{document}
