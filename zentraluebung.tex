\documentclass[12pt,a4paper]{article}
\usepackage[T1]{fontenc}
\usepackage[utf8]{inputenc}
\usepackage{lmodern}
\usepackage[left=28mm,top=28mm,right=28mm,bottom=28mm] {geometry}
\usepackage{amsfonts}
\usepackage{mathrsfs}
\usepackage[intlimits]{amsmath}
\usepackage{stmaryrd}
\usepackage{relsize}
\usepackage{etoolbox}
\usepackage[ngerman]{babel}
\usepackage[utf8]{inputenc}
\usepackage[T1]{fontenc}
\usepackage{marvosym}
\usepackage[shortlabels]{enumitem}
\usepackage{mathtools}
\usepackage{amssymb}
\usepackage{cancel}
\usepackage{mdframed}
\usepackage{framed}
\usepackage{mathtools}
\usepackage{tablefootnote} 
\usepackage{listings}
\usepackage{amsthm}
\usepackage{xcolor}
\usepackage{etoolbox}
\usepackage[all]{xy}
\usepackage{tikz}
\usepackage{thmtools}
\usepackage[
   pdfpagelabels=true,
   pdftitle={Analysis III: Maßtheorie und Integralrechnung mehrerer Variablen},
   pdfauthor={MT},
 ]{hyperref}
\usepackage{bookmark}
\usetikzlibrary{cd}
\usetikzlibrary{calc}
\theoremstyle{definition}
\newtheorem*{definition}{Definition}
\newtheorem*{satz}{Satz}
\newtheorem*{lemma}{Lemma}
\newtheorem*{proposition}{Proposition}
\newtheorem*{korollar}{Korollar}
\newtheorem*{folgerung}{Folgerung}
\newtheorem*{example}{Beispiel}
\newtheorem*{hauptbsp}{Hauptbeispiel}
\theoremstyle{remark}
\newtheorem*{remark}{Bemerkung}
\newtheorem*{remark'}{Nebenbemerkung}
\newtheorem*{beobachtung}{Beobachtung}
\AfterEndEnvironment{lemma}{\noindent\ignorespaces}
\AfterEndEnvironment{definition}{\noindent\ignorespaces}
\AfterEndEnvironment{example}{\noindent\ignorespaces}
\AfterEndEnvironment{theorem}{\noindent\ignorespaces}
\AfterEndEnvironment{satz}{\noindent\ignorespaces}
\AfterEndEnvironment{korollar}{\noindent\ignorespaces}
\AfterEndEnvironment{remark}{\noindent\ignorespaces}
\AfterEndEnvironment{remark'}{\noindent\ignorespaces}
\AfterEndEnvironment{proposition}{\noindent\ignorespaces}
\AfterEndEnvironment{proof}{\noindent\ignorespaces}
\let\existstemp\exists
\let\foralltemp\forall
\newcommand{\tikzmark}[1]{\tikz[overlay,remember picture] \node (#1) {};}
\newcommand{\vsubset}{\rotatebox[origin=c]{90}{$\subset$}}
\newcommand{\vphi}{\phi}
\newcommand{\ol}{\overline}
%Differentiation
\newcommand{\D}{\, \mathrm{d}}
%Bold Symbols
\newcommand{\R}{\mathbb{R}}
\newcommand{\C}{\mathbb{C}}
\newcommand{\N}{\mathbb{N}}
\newcommand{\Q}{\mathbb{Q}}
%Calligraphic Symbols
\newcommand{\II}{\mathcal{I}}
\newcommand{\FF}{\mathcal{F}}
\newcommand{\QQ}{\mathcal{Q}}
\newcommand{\EE}{\mathcal{E}}
\newcommand{\PP}{\mathcal{P}}
\newcommand{\TT}{\mathcal{T}}
\newcommand{\HH}{\mathscr{H}}
\newcommand{\RR}{\mathscr{R}}
\newcommand{\BB}{\mathscr{B}}
\newcommand{\CC}{\mathscr{C}}
\newcommand{\DD}{\mathscr{D}}
\renewcommand{\AA}{\mathscr{A}}
% Operator names
\newcommand{\Hom}{\operatorname{Hom}}
\newcommand{\del}{\partial}
\newcommand{\vol}{\operatorname{vol}}
\newcommand{\Var}{\operatorname{Var}} 
\newcommand{\Cov}{\operatorname{Cov}}
\newcommand{\End}{\operatorname{End}}
\newcommand{\SL}{\operatorname{SL}}
\newcommand{\Bild}{\begin{tiny}(Bild hier)\end{tiny}}
\renewcommand*{\exists}{\existstemp\mkern2mu}
\renewcommand*{\forall}{\foralltemp\mkern2mu}
\renewcommand{\emptyset}{\varnothing}
\renewcommand{\Re}{\operatorname{Re}}
\renewcommand{\Im}{\operatorname{Im}}
\renewcommand{\qedsymbol}{$\blacksquare$}
\renewcommand{\phi}{\varphi}
\renewcommand{\thesection}{\Roman{section}}
\makeatletter 
\AfterEndEnvironment{mdframed}{%
 \tfn@tablefootnoteprintout% 
 \gdef\tfn@fnt{0}% 
}
\numberwithin{equation}{section}

\setlist[enumerate,1]{label={(\roman*)}}

\title{Zentralübungen der Vorlesung Analysis III}
\begin{document}
\appendix
\section{Zentralübungen}
\addtocontents{toc}{\protect\setcounter{tocdepth}{0}}
\subsection{1. Zentralübung}
\marginpar{\tiny{16.10.2019}}
\subsubsection{Trigonometrische Funktionen}
Wir betrachten die Kurve, die mit der Geschwindigkeit 1 durch den Einheitskreis läuft. Sie erfüllt die Gleichung
\begin{equation*}
\begin{cases}
x^2+y^2=1 \\
x'^2 +y'^2 =1
\end{cases}
\end{equation*}
und die Differentialgleichungen
\begin{equation*}
\begin{cases}
x'(t)=-y(t)\\
y'(t)=x(t)
\end{cases}
\end{equation*}
Die komplexe Exponentialfunktion ist definiert durch:
\begin{equation*}
\exp: \C \to \C, z\mapsto e^z := \sum_{n=0}^\infty \frac{z^n}{n!}
\end{equation*}  
Sie erfüllt die funktionale Gleichung
\begin{equation*}
e^{z_1+z_2}=e^{z_1} \cdot e^{z_2}
\end{equation*}
und hat die Ableitung
\begin{align*}
(e^x)' & = e^x	\quad \text{($x$ reell)} \\
(e^{ix})' & = ie^{ix}
\end{align*}
Die erste Gleichung folgt aus
$$ \frac{e^{x+h}-e^x}{h} = e^x \cdot \underbrace{\frac{e^h-1}{1}}_{\longrightarrow 1 \text{ für } h \to 0} $$
Die Kreis -bzw. trigonometrischen Funktionen in $\C \cong \R^2$ sind eng mit der Exponentialfunktion verbunden, denn
\begin{equation*}
e^{it} =: \cos t +i\sin t \tag{eulersche Formel}
\end{equation*}
Dies ist äquivalent zu
$$ \begin{cases}
\cos t = \Re e^{it} \overset{(*)}= \frac{e^{it} + e^{-it}}{2}\\
\sin t = \Im e^{it} = \frac{e^{it}-e^{-it}}{2}
\end{cases}$$
($*$) folgt aus
$$
e^{\ol{z}} = \ol{e^z}, \quad \text{also } e^{-it} = \ol{e^{it}} = \cos t - i \sin t
$$
$e^{it}$ bewegt sich mit Geschwindigkeit $\equiv 1$ entlang Einheitskreis, denn
$$1 = e^0 \overset{\text{Fkt Gl}}=e^{it} \cdot e^{-it} = \cos^2 t + \sin^2 t$$
und somit
$$\left|\frac{d}{dt}e^{it}\right| = |ie^{it}| = |i(\cos t + i \sin t)|=1$$
Es folgt außerdem
\begin{equation*}
\begin{cases}
\cos ' = -\sin \\
\sin' = \cos
\end{cases}
\end{equation*}
da
$$
\underbrace{(e^{it})'}_{\cos' t + i\cdot \sin' t} =  \underbrace{i e^{it}}_{-\sin t+ i \cos t}
$$
\paragraph{Additionstheorem für $\sin$ und $\cos$}
\begin{equation*}
\begin{cases}
\cos (\alpha + \beta ) = \cos \alpha \cos \beta - \sin \alpha \sin \beta \\
\sin (\alpha + \beta) = \sin \alpha \cos \beta + \cos \alpha \sin \beta
\end{cases}
\end{equation*}
Dies folgt aus
$$e^{i(\alpha + \beta)} = e^{i\alpha}\cdot e^{i\beta}$$
Jetzt wollen wir überprüfen, ob die Fläche = $\frac{1}{2}$ Grundseite $*$ Höhe ist. Dazu zerlegen wir die Fläche zu zwei Teilen und berechnen das Integral  \Bild
\begin{align*}
\int\limits_{\cos t}^1 \sqrt{1-x^2} \D x
\end{align*}
\end{document}